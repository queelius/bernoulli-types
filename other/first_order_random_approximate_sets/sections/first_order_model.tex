\documentclass[ ../main.tex]{subfiles}
\providecommand{\mainx}{..}
\begin{document}
\section{First-order random approximate set model}
%A random variable $\RV{W} \colon \Sigma \mapsto \PlainSet{Y}$ is a function that maps outcomes in the $\sigma$-algebra to a measurable space 
%$\PlainSet{Y}$.
%The probability that $\RV{W}$ realizes some measureable subset $\PlainSet{Z} \subseteq \PlainSet{Y}$ is given by $\Prob{\RV{W} \in 	\PlainSet{S}} = %\Prob{\{w \Given \RV{W}(w) \in \PlainSet{Z}\}}$.
In the \emph{random} approximate set model, we do not describe any particular approximation, but rather we describe the statistical properties of processes that \emph{generate} approximations.

Suppose we have a universal set $\Set{X}$ with a total order where $x_{(j)}$ denotes the $j$-th ranked element
and we serialize subsets of $\Set{X}$ with bit strings of length $\Card{\Set{X}}$ where the $j$-th bit is $1$ if the $j$-th ranked element of $\Set{X}$ is a member of the set and otherwise $0$.
If we have a communications channel which transmits $1$s and $0$s and due to \emph{noise} or \emph{rate-distortion} flips bits with probability $\epsilon$, denoted its \emph{error rate}, then the channel induces a random approximation of any serialized sets transmitted over it.
This channel in particular induces \emph{first-order} approximations, denoted the \emph{first-order random approximate set} model.
Typically, the communications channel is a storage medium and $\epsilon$ is a rate distortion caused by \emph{lossy} compression algorithms that construct approximations of input sets.

We may generalize this result to higher-order random approximate sets.
The \emph{order} of a random approximate set model is the number of partitions of the universal set $\Set{X}$ such that each block of the partition has an identically distributed error rate, i.e., each block has a Bernoulli distributed error rate where an \emph{error} is a relation between an objective set and its approximation.

Suppose we have an objective set $\Set{A}$ and we model this set with an approximation $\Set{B}$.
Then, for any element $x \in \Set{X}$, an error occurs if $\SetIndicator{\Set{B}}(x) \neq \SetIndicator{\Set{A}}(x)$.
In the first-order model, there is only one element in the partition, the universal set $\Set{X}$, and thus each element has an identically distributed error,
\begin{equation}
	[\SetIndicator{\ASetFO{A}}(a) \neq \SetIndicator{\Set{A}}(a)] \sim \berdist(\epsilon)\,,
\end{equation}
for all $a \in \Set{X}$.

Since the first-order random approximate set is \emph{random}, properties like its \emph{error rate} are also random, modeled by the random error rate $\Delta$.
The \emph{zero-th order} generative model for sets is not generally known, but we denote the zero-th order model by $\RV{R}$.
We denote the \emph{first-order} random approximate set generative model by $\AT{\RV{R}}$.
The joint distribution of $\AT{\RV{R}}$, $\Delta$, and $\RV{R}$ given a unviersal set $\Set{U}$ has a probability density
\begin{equation}
\PDF{\Set{Y}, \epsilon, \Set{X} \Given \Set{U}}[\AT{\RV{R}}, \Delta, \RV{R}]\,.
\end{equation}
By the axioms of probability theory, this may be decomposed into
\begin{equation}
\PDF{\Set{Y},\epsilon, \Set{X} \Given \Set{U}}[\AT{\RV{R}}, \Delta, \RV{R}] =
\PDF{\Set{Y} \Given \epsilon, \Set{X}, \Set{U}}[\AT{\RV{R}} \Given \Delta, \RV{R}]
\PDF{\epsilon \Given \Set{X}}[\Delta \Given \RV{R}]
\PDF{\Set{X} \Given \Set{U}}[\RV{R}]\,.
\end{equation}
We typically omit the explicit reference to $\Set{U}$, since it may usually be understood as implicit to the model.

The object of central interest is the distribution of $\AT{\RV{R}}$ given $\RV{R}$.
The conditional distribution of $\AT{\RV{R}}$ given $\RV{R} = \Set{X}$ is denote by $\ASet{X}$.
By the axioms of probability,
\begin{equation}
\PDF{\Set{Y},\epsilon}[\ASet{X}, \Delta] =
\PDF{\Set{Y} \Given \epsilon}[\ASet{X} \Given \Delta]
\PDF{\epsilon \Given \Set{X}}[\Delta \Given \RV{R}]\,.
\end{equation}


\begin{equation}
\Fun{p}_{\AT{\RV{R}}}(\Set{Y} \Given \epsilon, \Set{U}) = \sum_{\Set{X} \in \PS{\Set{U}}} \PDF{\Set{Y}, \Set{X} \Given \epsilon, \Set{U}}[\AT{\RV{R}}, \RV{R} \Given \Delta]\,.
\end{equation}

\begin{equation}
\Fun{p}_{\AT{\RV{R}}}(\Set{Y} \Given \epsilon, \Set{U}) = \sum_{\Set{X} \in \PS{\Set{U}}} \PDF{\Set{Y} \Given \epsilon, \Set{X}, \Set{U}}[\AT{\RV{R}} \Given \Delta, \RV{R}] \PDF{\epsilon \Given \Set{X}}[\Delta \Given \RV{R}] \PDF{\Set{X} \Given \Set{U}}[\RV{R}]\,.
\end{equation}



The random error rate conditioned on $\RV{R} = \Set{X}$ is given by
\begin{equation}
	\Delta = \frac{1}{\Card{\Set{U}}} \sum_{x \in \Set{U}} [ \SetIndicator{\ASet{X}}(x) \neq \SetIndicator{\Set{X}}(x) ]\,.
\end{equation}

$\ASet{A}$ conditioned on $\Delta = a$ is a random approximate set with the indicated false positive and false negative rates.
If the rates happen to pick out a specific set in the support, then the result is a degenerate distribution, e.g., $\ASet{A}$ given $\Delta = 0$ is degenerate where all probability mass is assigned to $\Set{A}$.

An object of central interest is the distribution of $\ASet{X}$ given $\Expect{\Delta} = \epsilon$, denoted by
\begin{equation}
\ASet{X}[\epsilon]\,.
\end{equation}

If we \emph{sample} from ${}^{\epsilon} \! \! \Set{A}$, some set $\Set{Y} \in \PS{\Set{U}}$ with an error rate $a$ will be realized with probability $\PDF{\Set{Y} \Given a}[\ASet{A}[\epsilon] \Given \Delta]$.
However, as the number of samples goes to infinity, the mean error rate goes to $\epsilon$.

The complement of a first-order approximate set is a first-order approximate set.
The first-order random approximate set model makes the following set of assumptions.
\begin{axiom}
\label{asm:err_rate}
Given $\ASet{A}[\epsilon]$, the outcome of a membership test on any element in the universe is an independent and identically distributed Bernoulli trial with a mean $\epsilon$,
\begin{equation}
\label{eq:axiom_err_rate}
    \Prob{\SetIndicator{\ASet{A}[\epsilon]}(x) \neq \SetIndicator{\Set{A}}(x) \Given \SetIndicator{\Set{U}}(x)} = \epsilon\,.
\end{equation}
\end{axiom}

Given a universal set $\Set{X}$, we may partition the universe as $\left\{\Set{X}[1],\ldots,\Set{X}[k]\right\}$ such that
\begin{equation}
\Prob{\SetIndicator{\ASet{A}}(x) \neq \SetIndicator{\Set{A}}(x) \Given \SetIndicator{\Set{X}[i]}(x)} = \epsilon_i
\end{equation}
for $i \in \{1,\ldots,k\}$ where $\epsilon_1,\ldots,\epsilon_k$ are different.
We denote the order of such a random approximate set model the $k$-th order model.

The first-order model has only one such partition, the universal set itself.

The second-order random approximate set model is conditioned on two partitions.

A very useful constraint on the partition is given by
specifically
\begin{align}
    \Prob{\SetIndicator{\ASet{A}[\fprate]}(x) \neq \SetIndicator{\Set{A}}(x) \Given \SetIndicator{\SetComplement[\Set{A}]}(x)} &= \fprate\\
    \Prob{\SetIndicator{\ASet{A}[\fnrate]}(x) \neq \SetIndicator{\Set{A}}(x) \Given \SetIndicator{\Set{A}}(x)} &= \fnrate\,,
\end{align}
which we denote the \emph{natural} second-order approximation.
A common case is when $\fnrate$ ($\fprate$) is $0$, in which case we have a \emph{positive} (negative) second-order random approximate set.
Examples of the positive second-order random approximate set is the Bloom filter, 

In this case, the expected error rate is dependent on the distribution of $\RV{R}$, i.e.,
\begin{equation}
	\epsilon = \sum_{\Set{A} \in \PS{\Set{U}}} \fnrate \Fun{p}_{\RV{R}}(\Set{A}) + \fprate \Fun{p}_{\RV{R}}(\SetComplement[\Set{A}])\,.
\end{equation}
If we assume that $\RV{R}$ is uniformly distributed over its support $\PS{\Set{U}}$, then the total expected error given a proportion $\alpha$ of positives is the linear combination
\begin{equation}
	\epsilon = \alpha \fnrate + (1-\alpha) \fprate\,.
\end{equation}

By \cref{asm:err_rate} and by the axioms of probability,
\begin{equation}
\PDF{\Set{Y},\epsilon}[\ASet{X}, \Delta] =
\PDF{\Set{Y} \Given \epsilon}[\ASet{X} \Given \Delta]
\PDF{\epsilon \Given \Set{X}}[\Delta \Given \RV{R}]
\end{equation}

Every statistical property of the first-order random approximate set model is entailed by \cref{asm:err_rate}.

Suppose the first-order random approximate sets are over the universal set $\Set{U}$.
Compositions of first-order random approximate sets over the Boolean algebra $(\PS{\Set{U}},\SetUnion,\SetIntersection,\SetComplement,\EmptySet,\Set{U})$, or random approximate sets of random approximate sets, are not closed over the \emph{first-order} model.
%They are transformed into \emph{second-order} random approximate sets, which is a subject beyond the scope of this paper.


Consider the following example.
\begin{example}
	Suppose the universal set is $\{ x_1,x_2 \}$ and consider the distribution of the first-order random approximate set $\AT{\{x_1\}}[\epsilon]$.
	The probability mass function $\Fun{p}_{\AT{\{x_1\}}[\epsilon]}$ is given by
	\begin{equation}
	\Fun{p}_{\AT{\{x_1\}}[\epsilon]}(\Set{X}) =
	\begin{cases} 
	\epsilon (1-\epsilon) 	& \Set{X} = \EmptySet\,,\\
	\epsilon^2     			& \Set{X} = \{x_2\}\,,\\
	(1-\epsilon)^2     & \Set{X} = \{x_1\}\,,\\
	(1-\epsilon) \epsilon         & \Set{X} = \{x_1,x_2\}\,.
	\end{cases}
	\end{equation}
\end{example}
\end{document}