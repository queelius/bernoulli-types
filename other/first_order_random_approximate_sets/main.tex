\documentclass[10pt,final,hidelinks]{article}
\usepackage{lmodern}
\usepackage[T1]{fontenc}
\usepackage[english]{babel}
\usepackage[utf8]{inputenc}
\usepackage[activate={true,nocompatibility},final,tracking=true,kerning=true,
    spacing=true,factor=1100,stretch=10,shrink=10]{microtype}
\microtypecontext{spacing=nonfrench}
\usepackage[margin=.5in]{geometry}
\usepackage{graphicx}
\graphicspath{{img/}} 
\usepackage[noend]{algorithm2e}
\usepackage{caption}
\usepackage{subcaption}
\usepackage{booktabs}
\usepackage{array}
\usepackage{mathtools}
\usepackage{subfiles}
\usepackage{enumitem}
\usepackage{commath}
\usepackage{appendix}
\renewcommand\appendixtocname{Appendices}
\usepackage{amsmath}
\usepackage{amsthm}
\usepackage{amssymb}
%\usepackage[intoc,english]{nomencl}
%\makenomenclature
\usepackage{pgfplots}
\usepackage{tabu} 
\usepackage{tikz}
\usepackage{tikzscale}
\usepackage[square,numbers]{natbib}
\bibliographystyle{plain}
\usepackage{siunitx}
\numberwithin{equation}{section}
\usepackage{hyperref}
\usepackage{cleveref}
\usepackage{nomencl}
\makenomenclature

\usepackage{functionnotation}
\usepackage{algorithmnotation}
\usepackage{matrixnotation}
\usepackage[fancy]{setnotation}
\usepackage[fancy]{relationnotation}
\usepackage{approxsetnotation}
\usepackage{approxrelationnotation}
\usepackage[section]{envnotation}

\usepackage{custom}

\title
{
    First-order random approximate set model\\
    \large and corresponding random binary classification measures.
}
\author
{
    Alexander Towell\\
    \texttt{atowell@siue.edu}
}
\date{}

\begin{document}
\maketitle
\begin{abstract}
We define a \emph{random approximate set} model and the probability space that 
follows.
A random approximate set is a \emph{probabilistic} set generated to \emph{approximate} another set of objective interest.
We derive several properties that follow from this definition, such as the expected \emph{precision} in information retrieval.
Finally, we demonstrate an application of approximate sets, approximate Encrypted Search with queries as a Boolean algebra, which generates random approximate result sets.
\end{abstract}

\microtypesetup{protrusion=false}
\tableofcontents
\microtypesetup{protrusion=true}
\listoftables
\listoffigures

\include{nom}
\printnomenclature

\subfile{sections/intro}
\subfile{sections/algebra_of_sets}
\subfile{sections/aset_model}
\subfile{sections/first_order_model}
\subfile{sections/derived_distributions}
\subfile{sections/relational}
\subfile{sections/aset_theory}
\subfile{sections/interval}
\subfile{sections/aset_adt}
\subfile{sections/appendix}

\bibliography{references}

\end{document}
