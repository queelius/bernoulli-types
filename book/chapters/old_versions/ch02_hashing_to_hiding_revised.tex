\chapter{From Hashing to Hiding}
\label{ch:hashing}

\begin{quote}
\textit{``The secret to hiding information is not to hide it, but to make it indistinguishable from noise.''}
\end{quote}

\section*{Learning Objectives}
By the end of this chapter, you will:
\begin{itemize}
\item Understand the difference between cryptographic and non-cryptographic hash functions
\item Learn how deterministic hashing becomes probabilistic encoding
\item Recognize the avalanche effect and its importance
\item Implement Bernoulli encoding for uniform representations
\item See how hashing enables both efficiency and privacy
\end{itemize}

\section{The Dual Nature of Hashing}

Hash functions serve two masters: efficiency and security. Understanding both faces is crucial for building privacy-preserving systems.

\subsection{Non-Cryptographic Hash Functions}

For data structures like hash tables and Bloom filters, we need fast hash functions with good distribution properties:

\begin{definition}[Non-Cryptographic Hash Function]
A function $h: \{0,1\}^* \to \{0,1\}^m$ is a non-cryptographic hash if:
\begin{enumerate}
\item \textbf{Deterministic}: Same input always produces same output
\item \textbf{Uniform distribution}: Outputs are evenly distributed
\item \textbf{Fast computation}: Optimized for speed (often < 10 CPU cycles)
\item \textbf{Good avalanche}: Small input changes cause large output changes
\end{enumerate}
\end{definition}

Examples include MurmurHash3, xxHash, and CityHash. These prioritize speed over security.

\subsection{Cryptographic Hash Functions}

For security applications, we need stronger guarantees:

\begin{definition}[Cryptographic Hash Function]
A function $H: \{0,1\}^* \to \{0,1\}^n$ is cryptographically secure if:
\begin{enumerate}
\item \textbf{Pre-image resistance}: Given $y$, computationally infeasible to find $x$ where $H(x) = y$
\item \textbf{Second pre-image resistance}: Given $x_1$, hard to find $x_2 \neq x_1$ where $H(x_1) = H(x_2)$
\item \textbf{Collision resistance}: Hard to find any pair $(x_1, x_2)$ where $x_1 \neq x_2$ and $H(x_1) = H(x_2)$
\end{enumerate}
\end{definition}

Examples include SHA-256, SHA-3, and BLAKE2. These sacrifice speed for security.

\section{Hashing as Information Compression}

Every hash function is fundamentally a compression function—mapping a large input space to a smaller output space. This compression is lossy and irreversible.

\begin{example}[The Pigeonhole Principle in Action]
SHA-256 maps arbitrary-length inputs to 256-bit outputs. There are $2^{256}$ possible outputs but infinitely many possible inputs. Therefore:
\begin{itemize}
\item Multiple inputs must map to the same output (collisions exist)
\item Information is necessarily lost
\item The mapping cannot be reversed
\end{itemize}
\end{example}

This information loss, usually seen as a limitation, becomes our privacy advantage.

\section{The Avalanche Effect}

A crucial property for both efficiency and security is the avalanche effect:

\begin{definition}[Avalanche Effect]
A hash function exhibits the avalanche effect if changing a single bit in the input causes, on average, half of the output bits to change.
\end{definition}

\begin{example}[SHA-256 Avalanche]
\begin{verbatim}
Input: "hello"  → SHA-256: 2cf24dba5fb0a30e26e83b2ac5b9e29e...
Input: "Hello"  → SHA-256: 185f8db32271fe25f561a6fc938b2e26...
\end{verbatim}
Changing one bit (h→H) completely changes the output.
\end{example}

The avalanche effect ensures:
\begin{itemize}
\item Similar inputs produce dissimilar outputs (good for hash tables)
\item Input patterns don't leak through outputs (good for privacy)
\item Small changes are amplified (good for integrity checking)
\end{itemize}

\section{From Deterministic to Probabilistic}

Traditional hashing is deterministic—the same input always produces the same output. This is a problem for privacy: repeated queries reveal patterns. We need probabilistic encoding.

\subsection{The Birthday Paradox Lottery}

Imagine a lottery where:
\begin{itemize}
\item Your ticket number is your birthday (1-365)
\item You can buy multiple tickets with the same birthday
\item The winning number is drawn randomly each time
\end{itemize}

If you always use your actual birthday, an observer learns it after seeing enough tickets. But if you randomly select from \textit{plausible} birthdays, observers learn nothing specific about you.

This is the key insight: we need multiple valid encodings for each input.

\subsection{Bernoulli Encoding}

\begin{definition}[Bernoulli Encoding]
A Bernoulli encoding of value $v$ is a probabilistic function $\tilde{h}: V \to 2^H$ that:
\begin{enumerate}
\item Maps each value to a set of valid hash outputs
\item Randomly selects from this set for each encoding
\item Maintains uniform distribution across all outputs
\item Preserves necessary properties (e.g., membership testing)
\end{enumerate}
\end{definition}

Formally:
\[
\tilde{h}(v) = \text{RandomChoice}(\text{ValidEncodings}(v))
\]

\section{Building Uniform Representations}

The goal is to make encoded values indistinguishable from random noise:

\begin{theorem}[Uniformity Through Encoding]
If ValidEncodings$(v)$ is constructed such that:
\[
\Pr[\tilde{h}(v_1) = h] = \Pr[\tilde{h}(v_2) = h] = \frac{1}{|H|}
\]
for all values $v_1, v_2$ and hash outputs $h \in H$, then the encoding is perfectly uniform and reveals no information about the input value.
\end{theorem}

\begin{proof}[Proof Sketch]
By the definition of entropy, if all outputs are equally likely regardless of input, the mutual information $I(V; H) = 0$, meaning the output reveals nothing about the input.
\end{proof}

\section{Implementation: From Theory to Practice}

Let's implement Bernoulli encoding in Python:

\begin{lstlisting}[language=Python, caption={Bernoulli encoding implementation}]
import hashlib
import random
from typing import Set, List

class BernoulliEncoder:
    """Encodes values using probabilistic hashing for uniformity"""
    
    def __init__(self, output_size: int = 256, redundancy: int = 10):
        """
        Initialize encoder
        
        Args:
            output_size: Size of output space (bits)
            redundancy: Number of valid encodings per value
        """
        self.output_size = output_size
        self.redundancy = redundancy
        # Secret salt for this instance (never revealed)
        self.salt = random.randbytes(32)
    
    def _base_hash(self, value: str, nonce: int) -> bytes:
        """
        Generate base hash with nonce
        
        Note: Uses SHA-256 for security. In production, consider
        BLAKE2 for better performance with same security.
        """
        data = f"{value}:{nonce}:{self.salt.hex()}".encode()
        return hashlib.sha256(data).digest()
    
    def valid_encodings(self, value: str) -> Set[bytes]:
        """Generate set of valid encodings for a value"""
        encodings = set()
        for nonce in range(self.redundancy):
            h = self._base_hash(value, nonce)
            encodings.add(h[:self.output_size // 8])
        return encodings
    
    def encode(self, value: str) -> bytes:
        """
        Probabilistically encode a value
        
        Returns different encoding each time (uniformly random)
        """
        valid = self.valid_encodings(value)
        return random.choice(list(valid))
    
    def verify(self, value: str, encoding: bytes) -> bool:
        """Check if encoding is valid for value"""
        return encoding in self.valid_encodings(value)
    
    def encode_uniform(self, value: str, frequency: float) -> bytes:
        """
        Encode with frequency-aware uniformity
        
        More encodings for rare values (1/p principle)
        """
        # Adjust redundancy based on frequency
        adjusted_redundancy = max(1, int(self.redundancy / frequency))
        
        # Generate more encodings for rare values
        encodings = set()
        for nonce in range(adjusted_redundancy):
            h = self._base_hash(value, nonce)
            encodings.add(h[:self.output_size // 8])
        
        return random.choice(list(encodings))
\end{lstlisting}

\section{Security Analysis}

\subsection{Threat Model}

We consider an adversary who:
\begin{itemize}
\item Observes all encoded values
\item Knows the encoding algorithm (but not the salt)
\item Has unlimited computational power for analysis
\item Cannot perform adaptive chosen-plaintext attacks
\end{itemize}

\subsection{Security Properties}

Under this threat model, Bernoulli encoding provides:

\begin{theorem}[Observational Indistinguishability]
Given encodings $E_1 = \{\tilde{h}(v_1)^{(i)}\}$ and $E_2 = \{\tilde{h}(v_2)^{(i)}\}$ for different values $v_1, v_2$, an adversary cannot distinguish which set corresponds to which value with probability better than random guessing.
\end{theorem}

However, it does NOT provide:
\begin{itemize}
\item Protection against frequency analysis (without additional measures)
\item Resistance to correlation attacks (without tuple encoding)
\item Security against active adversaries who can inject chosen values
\end{itemize}

\section{The Path to Privacy}

Bernoulli encoding transforms deterministic operations into probabilistic ones, but alone it's not enough. We need additional techniques:

\begin{enumerate}
\item \textbf{Frequency normalization}: Equal encoding rates regardless of value frequency
\item \textbf{Tuple encoding}: Hide correlations between values
\item \textbf{Temporal shuffling}: Randomize query timing
\item \textbf{Noise injection}: Add fake queries
\end{enumerate}

These build on the foundation of Bernoulli encoding to achieve true privacy.

\section{Exercises}

\begin{enumerate}
\item \textbf{Avalanche Testing}: Implement a function to measure the avalanche effect of different hash functions. Test MD5, SHA-256, and MurmurHash3.

\item \textbf{Collision Finding}: For a hash function with $n$-bit output, how many random inputs do you need (on average) before finding a collision? Verify experimentally with small $n$.

\item \textbf{Encoding Distribution}: Prove that if each value has $k$ valid encodings selected uniformly, the output distribution approaches uniform as $k$ increases.

\item \textbf{Security Analysis}: Under what conditions can an adversary distinguish between Bernoulli encodings of two values? Consider frequency and correlation.

\item \textbf{Implementation Challenge}: Extend the BernoulliEncoder to support efficient batch encoding while maintaining uniformity.
\end{enumerate}

\section{Chapter Summary}

Hash functions are more than tools for efficiency—they're the foundation of privacy. By transforming deterministic hashing into probabilistic Bernoulli encoding, we achieve uniform representations indistinguishable from random noise.

Key insights:
\begin{itemize}
\item Information loss through hashing enables privacy
\item The avalanche effect amplifies small changes, hiding patterns
\item Probabilistic encoding breaks the deterministic link between input and output
\item Uniformity is achievable but requires careful construction
\end{itemize}

Next, we'll see what adversaries can observe and why traditional approaches fail to protect privacy, setting the stage for our complete oblivious solution.

\section{Further Reading}

\begin{itemize}
\item Menezes, A. et al. (1996). ``Handbook of Applied Cryptography'' (Chapter 9: Hash Functions)
\item Rogaway, P. \& Shrimpton, T. (2004). ``Cryptographic hash-function basics''
\item Aumasson, J.P. (2019). ``Serious Cryptography'' (Modern hash function design)
\item Bellare, M. et al. (2020). ``Mass-surveillance without the State: Strongly undetectable algorithm-substitution attacks''
\end{itemize}