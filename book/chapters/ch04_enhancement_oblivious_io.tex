\section{Enhancement: Fully Oblivious Input-Output}
\label{sec:oblivious-io}

\subsection{The Observable Output Problem}

Our current implementation has a critical weakness:

\begin{lstlisting}[language=Python, caption={Current approach - outputs are observable}]
def search(self, query: str) -> bool:  # PROBLEM: Boolean output is observable!
    encoded_query = self.encode(query)
    bloom_result = self.index[encoded_query]
    return bloom_result.contains(doc_id)  # Returns True/False - OBSERVABLE!
\end{lstlisting}

Even with oblivious queries, returning plaintext booleans leaks information:
\begin{itemize}
\item Result patterns reveal query selectivity
\item True/False sequences can be correlated
\item Timing of True results reveals investigation progress
\end{itemize}

\subsection{Bernoulli Map Construction for Oblivious Outputs}

Instead of Bloom filters (which output observable booleans), we use Bernoulli maps that output encoded values:

\begin{lstlisting}[language=Python, caption={Bernoulli map with encoded outputs}]
class BernoulliMap:
    """Maps encoded inputs to encoded outputs obliviously"""
    
    def __init__(self, secret_key: bytes):
        self.secret_key = secret_key
        # Map from input encodings to output encodings
        self.encoding_map = {}
        
    def insert(self, key: str, value: Any):
        """Store key-value pair with encoded representations"""
        # Generate multiple input encodings
        input_encodings = self.generate_encodings(key)
        
        # Generate encoded output (not plaintext!)
        output_encoding = self.encode_output(value)
        
        # Store all input encodings -> same output encoding
        for enc_in in input_encodings:
            self.encoding_map[enc_in] = output_encoding
    
    def lookup(self, encoded_key: bytes) -> bytes:
        """Returns ENCODED output, not plaintext"""
        # Return encoded value or encoded "not found"
        return self.encoding_map.get(
            encoded_key,
            self.encode_output(None)  # Encoded null
        )
    
    def encode_output(self, value: Any) -> bytes:
        """Encode output values uniformly"""
        if value is None:
            # Special encoding for "not found"
            data = b"NULL" + self.secret_key
        else:
            # Encode actual value
            data = str(value).encode() + self.secret_key
        
        # Generate uniform encoding
        return hashlib.sha256(data).digest()
    
    def decode_output(self, encoded: bytes, expected_values: List[Any]) -> Any:
        """Client-side decoding with expected value set"""
        for value in expected_values:
            if self.encode_output(value) == encoded:
                return value
        return None  # Not in expected set
\end{lstlisting}

\subsection{End-to-End Oblivious Computation}

Now both inputs and outputs remain encoded throughout:

\begin{lstlisting}[language=Python, caption={Fully oblivious search system}]
class FullyObliviousSearch:
    """Complete oblivious computation - encoded inputs AND outputs"""
    
    def __init__(self, secret_key: bytes):
        self.secret_key = secret_key
        self.bernoulli_map = BernoulliMap(secret_key)
        
    def oblivious_search(self, query: str) -> bytes:
        """
        Fully oblivious: 
        - Input: plaintext query (on client)
        - Processing: encoded query (on server) 
        - Output: encoded result (from server)
        - Decoding: back to plaintext (on client)
        """
        # Client side: encode query
        encoded_query = self.encode_query(query)
        
        # Server side: process encoded, return encoded
        encoded_result = self.server_process(encoded_query)
        
        # Client side: decode result
        # (Would happen on trusted client machine)
        return encoded_result  # Still encoded!
    
    def server_process(self, encoded_query: bytes) -> bytes:
        """
        Server NEVER sees plaintext
        Input: encoded query
        Output: encoded result
        """
        # Server just does lookup in Bernoulli map
        # Doesn't know what query means or what result means
        return self.bernoulli_map.lookup(encoded_query)
    
    def client_decode(self, encoded_result: bytes) -> bool:
        """Decode on trusted client only"""
        # Client knows expected values
        expected = [True, False]  # For boolean results
        
        # Try each possibility
        for value in expected:
            test_encoding = self.encode_output(value)
            if test_encoding == encoded_result:
                return value
        
        raise ValueError("Unexpected encoded result")
\end{lstlisting}

\subsection{Bernoulli Tuples for Correlation Hiding}

Beyond pairs, we can encode arbitrary tuples to hide complex correlations:

\begin{lstlisting}[language=Python, caption={Generalized tuple encoding}]
class BernoulliTupleEncoder:
    """Hide correlations through tuple encoding"""
    
    def __init__(self, correlation_sets):
        """
        correlation_sets: List of commonly correlated term sets
        Example: [
            ("covid", "vaccine", "trial"),
            ("password", "leak", "breach"),
            ("financial", "fraud", "investigation")
        ]
        """
        self.correlation_sets = correlation_sets
        self.tuple_encodings = self._precompute_tuples()
    
    def encode_query(self, terms: List[str]) -> bytes:
        """Encode query, using tuples when possible"""
        # Sort for canonical ordering
        terms_sorted = tuple(sorted(terms))
        
        # Check if this is a known correlation set
        if terms_sorted in self.correlation_sets:
            # Use single tuple encoding
            return self._encode_tuple(terms_sorted)
        
        # Check for subsets that form tuples
        for size in range(len(terms), 1, -1):
            for subset in combinations(terms, size):
                subset_sorted = tuple(sorted(subset))
                if subset_sorted in self.correlation_sets:
                    # Encode subset as tuple, rest individually
                    tuple_enc = self._encode_tuple(subset_sorted)
                    remaining = [t for t in terms if t not in subset]
                    other_encs = [self._encode_single(t) for t in remaining]
                    
                    # Combine encodings
                    return self._combine_encodings([tuple_enc] + other_encs)
        
        # No correlations found, encode individually
        return self._combine_encodings([self._encode_single(t) for t in terms])
    
    def _encode_tuple(self, term_tuple: Tuple[str]) -> bytes:
        """Create single encoding for entire tuple"""
        # Single encoding hides correlation!
        data = ":".join(term_tuple).encode()
        return hashlib.sha256(data).digest()
    
    def _combine_encodings(self, encodings: List[bytes]) -> bytes:
        """Combine multiple encodings obliviously"""
        # XOR for simplicity (could use more sophisticated combination)
        result = encodings[0]
        for enc in encodings[1:]:
            result = bytes(a ^ b for a, b in zip(result, enc))
        return result
\end{lstlisting}

\subsection{Two-Level Hash Schemes}

For more sophisticated constructions, we can use two-level perfect hashing with seed search:

\begin{lstlisting}[language=Python, caption={Two-level perfect hashing for Bernoulli maps}]
class TwoLevelBernoulliMap:
    """
    Two-level scheme:
    - Level 1: Bins with minimal collisions
    - Level 2: Per-bin seed search for perfect mapping
    """
    
    def __init__(self, items: List[Tuple[str, Any]], max_attempts=1000):
        self.level1_bins = defaultdict(list)
        self.level2_seeds = {}
        self.level2_maps = {}
        
        # Level 1: Distribute into bins
        for key, value in items:
            bin_id = self._hash_to_bin(key)
            self.level1_bins[bin_id].append((key, value))
        
        # Level 2: Find perfect hash seed for each bin
        for bin_id, bin_items in self.level1_bins.items():
            seed = self._find_perfect_seed(bin_items, max_attempts)
            if seed is None:
                # Fall back to standard encoding
                seed = 0
            
            self.level2_seeds[bin_id] = seed
            self.level2_maps[bin_id] = self._build_perfect_map(bin_items, seed)
    
    def _find_perfect_seed(self, items: List[Tuple[str, Any]], 
                          max_attempts: int) -> Optional[int]:
        """
        Search for seed that gives perfect (or near-perfect) mapping
        Most input-output pairs should map correctly
        """
        n = len(items)
        if n == 0:
            return 0
        
        # Target: at least 90% correct mappings
        target_correct = int(0.9 * n)
        
        for seed in range(max_attempts):
            positions = {}
            collisions = 0
            
            for key, value in items:
                pos = self._hash_with_seed(key, seed) % (2 * n)
                if pos in positions:
                    collisions += 1
                else:
                    positions[pos] = (key, value)
            
            # Check if this seed is good enough
            correct_mappings = n - collisions
            if correct_mappings >= target_correct:
                return seed
        
        return None  # No good seed found
    
    def _build_perfect_map(self, items: List[Tuple[str, Any]], 
                          seed: int) -> Dict[int, bytes]:
        """Build encoded map with the chosen seed"""
        n = len(items)
        encoded_map = {}
        
        for key, value in items:
            pos = self._hash_with_seed(key, seed) % (2 * n)
            # Encode the output value
            encoded_value = self._encode_value(value)
            encoded_map[pos] = encoded_value
        
        return encoded_map
    
    def lookup(self, key: str) -> bytes:
        """Two-level lookup returning encoded result"""
        # Level 1: Find bin
        bin_id = self._hash_to_bin(key)
        
        # Level 2: Use bin's seed for lookup
        seed = self.level2_seeds.get(bin_id, 0)
        bin_map = self.level2_maps.get(bin_id, {})
        
        n = len(self.level1_bins[bin_id])
        if n == 0:
            return self._encode_value(None)
        
        pos = self._hash_with_seed(key, seed) % (2 * n)
        
        # Return encoded value (or encoded null)
        return bin_map.get(pos, self._encode_value(None))
\end{lstlisting}

\subsection{Extending to Arbitrary Computations}

The pattern generalizes to any computation:

\begin{lstlisting}[language=Python, caption={Oblivious computation framework}]
class ObliviousComputation:
    """Framework for arbitrary oblivious computations"""
    
    def __init__(self, computation_type: str):
        self.computation_type = computation_type
        self.input_encoder = BernoulliEncoder()
        self.output_encoder = BernoulliEncoder()
        
    def oblivious_compute(self, 
                         encoded_inputs: List[bytes], 
                         operation: str) -> bytes:
        """
        Perform computation on encoded inputs,
        return encoded output
        """
        if self.computation_type == "boolean":
            return self._boolean_op(encoded_inputs, operation)
        elif self.computation_type == "arithmetic":
            return self._arithmetic_op(encoded_inputs, operation)
        elif self.computation_type == "comparison":
            return self._comparison_op(encoded_inputs, operation)
        else:
            raise ValueError(f"Unknown computation: {self.computation_type}")
    
    def _boolean_op(self, inputs: List[bytes], op: str) -> bytes:
        """Boolean operations on encoded values"""
        # Server doesn't decode - works directly on encodings
        if op == "AND":
            # Intersection of Bernoulli sets
            result = self._intersect_encodings(inputs)
        elif op == "OR":
            # Union of Bernoulli sets
            result = self._union_encodings(inputs)
        elif op == "NOT":
            # Complement of Bernoulli set
            result = self._complement_encoding(inputs[0])
        
        return result  # Still encoded!
    
    def _intersect_encodings(self, encodings: List[bytes]) -> bytes:
        """Intersection without decoding"""
        # Bit-wise AND for bloom-filter-like encodings
        result = encodings[0]
        for enc in encodings[1:]:
            result = bytes(a & b for a, b in zip(result, enc))
        return result
    
    def _union_encodings(self, encodings: List[bytes]) -> bytes:
        """Union without decoding"""
        # Bit-wise OR for bloom-filter-like encodings
        result = encodings[0]
        for enc in encodings[1:]:
            result = bytes(a | b for a, b in zip(result, enc))
        return result
\end{lstlisting}

\subsection{Key Insights}

This enhanced approach provides:

1. **End-to-end obliviousness**: Both inputs and outputs remain encoded
2. **Correlation hiding**: Through generalized tuple encoding
3. **Efficient constructions**: Two-level hashing with seed search
4. **Computational completeness**: Extends to arbitrary operations
5. **No observable patterns**: Server never sees plaintext values or results

The server becomes a truly oblivious processor - it performs computations on encodings without ever learning what they represent.