\section{Deep Dive: Advanced Oblivious Computing Concepts}
\label{sec:deep-dive-advanced}

Our exploration of the additional repositories reveals a comprehensive framework for oblivious computing that extends far beyond simple encrypted search. These projects demonstrate three fundamental pillars of complete oblivious systems:

\subsection{Oblivious Maps: Universal Function Approximators}

The \texttt{cipher\_maps} project introduces the concept of \textit{oblivious maps} as universal Bernoulli approximators:

\begin{definition}[Oblivious Map]
An oblivious map $f^* = (f, \mathcal{C})$ where $\mathcal{C}$ is the computational basis subset, satisfies:
\begin{itemize}
\item The function $f^* : X \mapsto Y$ is a Bernoulli approximation of $f$
\item If $x \in X$ is not in the domain of definition, $f^*(x)$ returns a random oracle output over $Y$
\item A particular mapping $y = f^*(x)$ may only be learned by applying $f^*$ to $x$
\end{itemize}
\end{definition}

\begin{lstlisting}[language=Python, caption={Oblivious map with random oracle behavior}]
class ObliviousMap:
    """
    Implements f* : X* -> Y* where both domain and codomain are oblivious
    Key properties:
    - Undefined elements return deterministic random oracle outputs
    - Mappings are learned lazily upon request
    - Error rates are tracked: error_rate(f_hat, x)
    """
    
    def __init__(self, secret_key: bytes):
        self.secret = secret_key
        self.defined_mappings = {}
        self.random_oracle_seed = hashlib.sha256(secret_key).digest()
    
    def __getitem__(self, encoded_x: bytes) -> bytes:
        """
        Returns encoded output for encoded input
        Undefined inputs get random oracle outputs
        """
        if encoded_x in self.defined_mappings:
            return self.defined_mappings[encoded_x]
        else:
            # Random oracle: deterministic but unpredictable
            return hashlib.sha256(
                self.random_oracle_seed + encoded_x
            ).digest()
    
    def error_rate(self, x: bytes) -> float:
        """
        Returns Pr{f_hat(x) != f(x)} for defined f
        For undefined x, returns 1.0 (always error)
        """
        return 0.0 if x in self.defined_mappings else 1.0
\end{lstlisting}

The theoretical paper proves that the \textit{singular hash map} (SHM) achieves the space-complexity lower bound for oblivious maps with arbitrary-length inputs and outputs, with false positive rate $\varepsilon$ and false negative rate $\eta$.

\subsection{Cipher Trapdoor Sets: Homomorphic Boolean Algebras}

The \texttt{cipher\_trapdoor\_sets} project develops a complete Boolean algebra over cipher sets that preserves operations while obscuring identities:

\begin{theorem}[Cipher Boolean Algebra Homomorphism]
Given the Boolean algebra $A = (\mathcal{P}(\{0,1\}^*), \cup, \cap, \complement, \emptyset, \{0,1\}^*)$ and $B = (\{0,1\}^m, \lor, \land, \neg, 0^m, 1^m)$, the homomorphism $F: A \mapsto B$ defined as:
\begin{equation}
F(\beta) = \begin{cases}
    h(\beta) & \beta \in \{0,1\}^* \\
    \lor & \beta = \cup \\
    \land & \beta = \cap \\
    \neg & \beta = \complement \\
    0^m & \beta = \emptyset \\
    1^m & \beta = \{0,1\}^*
\end{cases}
\end{equation}
preserves Boolean operations while providing one-way trapdoor properties through cryptographic hash function $h$.
\end{theorem}

The paper derives the false positive rate for these constructions:

\begin{theorem}[False Positive Rate]
The false positive rate is:
\begin{equation}
\text{FPR}(m,n) = \alpha^m(n) \text{ where } \alpha(n) = \left(1 - 2^{-(n+1)}\right)
\end{equation}
yielding a bit-rate per element:
\begin{equation}
b(n,\text{FPR}) = \frac{\log_2 \text{FPR}}{n \cdot \alpha(n)}
\end{equation}
\end{theorem}

\begin{lstlisting}[language=C++, caption={k-Disjoint Hash Set implementation}]
template<size_t k>
class DisjointHashSet {
    // k-DHS: bins m elements into m/k bins with k elements per bin
    // Special cases:
    //   k=1: Perfect Hash Filter
    //   k=m: Singular Hash Set (optimal space)
    
public:
    DisjointHashSet(size_t m, double target_fpr) 
        : num_elements(m), 
          bins(m/k), 
          elements_per_bin(k) {
        // Compute optimal parameters
        compute_hash_seeds(target_fpr);
    }
    
    bool contains(const bytes& encoded_element) {
        size_t bin_id = hash_to_bin(encoded_element);
        bytes bin_hash = compute_bin_hash(bin_id, encoded_element);
        
        // Check if hash indicates membership
        return check_membership_encoding(bin_hash);
    }
    
private:
    // Achieves information-theoretic lower bound
    // for space complexity when k=m (Singular Hash Set)
    size_t compute_optimal_space() {
        return -num_elements * log2(target_fpr) / log2(e);
    }
};
\end{lstlisting}

\subsection{Algebraic Cipher Types: Lifting Monoids to Cipher Monoids}

The \texttt{algebraic\_cipher\_types} project provides the mathematical foundation for lifting algebraic structures into the cipher domain:

\begin{definition}[Cipher Functor]
The cipher functor lifts a monoid $(S,*,e)$ to $c_A(S,*,e)$ where:
\begin{itemize}
\item $A \subseteq S$ is the subset of observable elements
\item $s: S \times \mathbb{N} \mapsto c_A S$ maps elements to their $k$-th cipher representation
\item $s': c_A S \mapsto S$ satisfies $s'(s(a,k)) = a$ for all $a \in S, k \in \mathbb{N}$
\item The operation $(c_A *): (c_A S, c_A S) \mapsto c_A S$ preserves associativity
\end{itemize}
\end{definition}

\begin{lstlisting}[language=C++, caption={Type-erased cipher values with monadic noise}]
template <typename T>
class cipher {
public:
    using plain_value_type = T;
    
    // Track error rates through operations
    double fpr() const { return impl_->fpr(); }
    double fnr() const { return impl_->fnr(); }
    
    // Decode only with secret key
    optional<T> try_decode(string_view secret) const {
        return impl_->try_decode(secret);
    }
    
    // Type erasure for polymorphic cipher types
private:
    struct concept {
        virtual double fpr() const = 0;
        virtual double fnr() const = 0;
        virtual optional<T> try_decode(string_view) const = 0;
    };
    
    template <typename CipherImpl>
    struct model : concept {
        // Concrete cipher implementation
        CipherImpl cipher_;
        
        double fpr() const override { 
            return compute_fpr(cipher_); 
        }
        double fnr() const override { 
            return compute_fnr(cipher_); 
        }
    };
    
    shared_ptr<const concept> impl_;
};

// Noisy cipher with level tracking for composition
template <typename T, unsigned int H, unsigned int L>
struct noisy_cipher {
    static constexpr auto secret_hash = H;
    static constexpr auto level = L;
    
    string code;  // Cipher encoding
    double fpr, fnr;  // Error rates
};

// Logical operations preserve level structure
template <unsigned int H, unsigned int L>
noisy_cipher<bool,H,L+1> logical_or(
    const noisy_cipher<bool,H,L>& x,
    const noisy_cipher<bool,H,L>& y
) {
    // Compose at next level with error propagation
    return compose_with_error_tracking(x, y, or_operation);
}
\end{lstlisting}

\subsection{Entropy Maximization for Encrypted Search}

The \texttt{encrypted\_search\_stream\_entropy\_maximization} project provides information-theoretic analysis of confidentiality:

\begin{theorem}[Maximum Entropy Distribution]
For an encrypted search system with $k$ search agents, the maximum entropy distribution has:
\begin{itemize}
\item Agent identities: $A_j \sim \text{Uniform}(k)$
\item Inter-arrival times: $T_j \sim \text{Geometric}(\lambda = 1/2)$
\item Trapdoors per query: $N_j \sim \text{Geometric}(\mu = 2)$
\item Trapdoor values: $Y_j \sim \text{Uniform}(0, 2^m - 1)$
\end{itemize}
This achieves expected compressed bit length:
\begin{equation}
\ell = \frac{1}{\lambda} + p + \mu(1 + m)
\end{equation}
where $p$ is the bits for agent encoding and $m$ is bits per trapdoor.
\end{theorem}

\subsection{Practical Implementation: Oblivious Key-Value Store}

The implementations demonstrate a complete oblivious KVS system:

\begin{lstlisting}[language=C++, caption={Complete oblivious KVS with cipher operations}]
class ObliviousKVS {
    // Store maps cipher keys to cipher values
    map<cipher<string>, cipher<string>> store;
    
    // Oblivious operations
    cipher<string> get(cipher<string> key) {
        auto it = store.find(key);
        if (it != store.end()) {
            return it->second;
        }
        // Return noise for undefined keys
        return cipher<string>::noise();
    }
    
    // Boolean operations on cipher values
    noisy_cipher<bool> contains_all(
        vector<cipher<string>> keys
    ) {
        noisy_cipher<bool> result = 
            noisy_cipher<bool>::true_value();
        
        for (const auto& key : keys) {
            auto exists = contains(key);
            result = logical_and(result, exists);
        }
        
        return result;  // Still encrypted!
    }
    
    // Oblivious function composition
    template<typename F>
    auto apply_oblivious(
        F encoded_function,
        cipher<string> encoded_input
    ) -> decltype(auto) {
        // Server applies function without knowing:
        // - What function it is
        // - What the input represents
        // - What the output means
        return encoded_function(encoded_input);
    }
};
\end{lstlisting}

\subsection{Key Mathematical Results}

\subsubsection{Space-Optimal Constructions}

The singular hash map achieves the information-theoretic lower bound:
\begin{equation}
\text{Space}(n, \varepsilon) = -n \cdot \frac{\log_2 \varepsilon}{1.44} \text{ bits}
\end{equation}

\subsubsection{Error Composition Through Operations}

For composed operations, error rates propagate as:
\begin{equation}
\text{FPR}(f \circ g) \leq \text{FPR}(f) + \text{FPR}(g) - \text{FPR}(f) \cdot \text{FPR}(g)
\end{equation}

\subsubsection{Entropy Bounds}

The entropy of an oblivious system is bounded by:
\begin{equation}
H(\text{Cipher}) \geq H(\text{Plain}) - I(\text{Plain}; \text{Observable})
\end{equation}
where $I$ is the mutual information between plaintext and observable ciphertext.

\subsection{Unified Framework Insights}

These projects together reveal a unified framework where:

\begin{enumerate}
\item \textbf{Universal Approximation}: Oblivious maps can approximate any function with controllable error rates
\item \textbf{Algebraic Preservation}: Boolean and monoid operations are preserved in the cipher domain
\item \textbf{Optimal Space}: Constructions achieve information-theoretic lower bounds
\item \textbf{Composability}: Operations compose with predictable error propagation
\item \textbf{Complete Obliviousness}: Server operates as pure $\text{bytes} \rightarrow \text{bytes}$ transformer
\end{enumerate}

The complete system provides:
\begin{itemize}
\item \textbf{Theoretical guarantees}: Proven optimal space and entropy bounds
\item \textbf{Practical implementations}: Working C++ and Python code
\item \textbf{Flexible abstractions}: Type erasure allows polymorphic cipher types
\item \textbf{Production readiness}: Error tracking, key management, and performance optimization
\end{itemize}

This represents a complete realization of the vision: computation on encrypted data where the server learns nothing beyond input/output sizes.