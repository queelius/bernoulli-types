\chapter{Building Your First Oblivious System}
\label{ch:first-system}

\begin{quote}
\textit{``In theory, theory and practice are the same. In practice, they are not.''} — Yogi Berra

\textit{``Let's build something that actually works.''} — This chapter
\end{quote}

\section*{Learning Objectives}
By the end of this chapter, you will:
\begin{itemize}
\item Build a complete privacy-preserving document search system
\item Understand the progression from naive to oblivious implementation
\item Learn to handle false positives and error propagation
\item Implement frequency hiding and correlation breaking
\item Measure and validate privacy guarantees
\end{itemize}

\section{The Challenge: Private Document Search}

We'll build a system where users can search documents without revealing what they're searching for. Our requirements:

\begin{itemize}
\item Store 1 million documents
\item Support keyword and Boolean queries
\item Sub-millisecond query time
\item No information leakage to server
\item 99.9\% accuracy (0.1\% false positive rate)
\end{itemize}

\section{Stage 1: The Naive Approach}

Let's start with what doesn't work—a traditional inverted index:

\begin{lstlisting}[language=Python, caption={Naive approach - NO PRIVACY}]
class NaiveDocumentSearch:
    """Traditional inverted index - fast but zero privacy"""
    
    def __init__(self):
        # Maps keywords to document IDs
        self.index = defaultdict(set)
    
    def index_document(self, doc_id: str, content: str):
        """Add document to index"""
        keywords = self.extract_keywords(content)
        for keyword in keywords:
            self.index[keyword].add(doc_id)
    
    def search(self, keyword: str) -> Set[str]:
        """Search for documents containing keyword"""
        # PRIVACY LEAK: Server sees exact keyword!
        return self.index.get(keyword, set())
    
    def extract_keywords(self, content: str) -> Set[str]:
        """Simple keyword extraction"""
        # In practice: use NLP, stemming, etc.
        words = content.lower().split()
        stop_words = {'the', 'a', 'an', 'and', 'or', 'but'}
        return {w for w in words if w not in stop_words and len(w) > 2}
\end{lstlisting}

Problems:
\begin{itemize}
\item Server sees exact search terms
\item Query patterns reveal user interests
\item Frequency analysis trivial
\item No plausible deniability
\end{itemize}

\section{Stage 2: Adding Cryptographic Hashing}

First improvement: hash the keywords:

\begin{lstlisting}[language=Python, caption={Hashed keywords - still not private}]
import hashlib
import hmac

class HashedDocumentSearch:
    """Hashed inverted index - hides keywords but not patterns"""
    
    def __init__(self, secret_key: bytes):
        self.secret_key = secret_key
        self.index = defaultdict(set)
    
    def hash_keyword(self, keyword: str) -> str:
        """Keyed hash for keywords (PRF)"""
        return hmac.new(
            self.secret_key,
            keyword.encode(),
            hashlib.sha256
        ).hexdigest()
    
    def index_document(self, doc_id: str, content: str):
        keywords = self.extract_keywords(content)
        for keyword in keywords:
            # Store hashed keyword
            hashed = self.hash_keyword(keyword)
            self.index[hashed].add(doc_id)
    
    def search(self, keyword: str) -> Set[str]:
        # PRIVACY LEAK: Patterns still visible!
        hashed = self.hash_keyword(keyword)
        return self.index.get(hashed, set())
\end{lstlisting}

Better, but still leaks:
\begin{itemize}
\item Search frequency (common terms searched more)
\item Correlations (terms searched together)
\item Result sizes (popular terms return more documents)
\end{itemize}

\section{Stage 3: Bloom Filter Transformation}

Now we use Bloom filters to store document sets:

\begin{lstlisting}[language=Python, caption={Bloom filter index - getting closer}]
class BloomFilterSearch:
    """Uses Bloom filters for space efficiency and some privacy"""
    
    def __init__(self, secret_key: bytes, docs_per_keyword: int = 10000):
        self.secret_key = secret_key
        self.docs_per_keyword = docs_per_keyword
        # Maps hashed keywords to Bloom filters
        self.index = {}
    
    def index_document(self, doc_id: str, content: str):
        keywords = self.extract_keywords(content)
        for keyword in keywords:
            hashed = self.hash_keyword(keyword)
            
            # Create Bloom filter if needed
            if hashed not in self.index:
                self.index[hashed] = BloomFilter(
                    expected_items=self.docs_per_keyword,
                    fp_rate=0.001
                )
            
            # Add document to Bloom filter
            self.index[hashed].add(doc_id)
    
    def search(self, keyword: str) -> BloomFilter:
        """Returns Bloom filter (may have false positives)"""
        hashed = self.hash_keyword(keyword)
        # Return empty filter if keyword not found
        return self.index.get(hashed, BloomFilter(1, 1.0))
    
    def search_boolean(self, query: str) -> BloomFilter:
        """Support AND/OR queries"""
        if " AND " in query:
            terms = query.split(" AND ")
            results = [self.search(t.strip()) for t in terms]
            # Intersection of Bloom filters
            return reduce(lambda a, b: a.intersect(b), results)
        elif " OR " in query:
            terms = query.split(" OR ")
            results = [self.search(t.strip()) for t in terms]
            # Union of Bloom filters
            return reduce(lambda a, b: a.union(b), results)
        else:
            return self.search(query)
\end{lstlisting}

Improvements:
\begin{itemize}
\item Space efficient (95\% reduction)
\item Can't extract original documents
\item Supports Boolean operations
\item Some uncertainty from false positives
\end{itemize}

Still visible: patterns, frequencies, correlations

\section{Stage 4: Frequency Hiding}

The key insight: make rare and common terms indistinguishable:

\begin{lstlisting}[language=Python, caption={Frequency hiding through multiple encodings}]
class FrequencyHidingSearch:
    """Hide term frequencies using the 1/p(x) principle"""
    
    def __init__(self, secret_key: bytes):
        self.secret_key = secret_key
        self.index = {}
        self.term_frequencies = {}  # Estimated from corpus
        
    def compute_encodings(self, keyword: str) -> List[str]:
        """
        Generate multiple encodings based on frequency
        Rare terms get more encodings (1/p principle)
        """
        base_hash = self.hash_keyword(keyword)
        freq = self.term_frequencies.get(keyword, 0.001)
        
        # More encodings for rare terms
        num_encodings = max(1, int(1.0 / freq))
        num_encodings = min(num_encodings, 100)  # Cap at 100
        
        encodings = []
        for i in range(num_encodings):
            # Generate variant encoding
            variant = hmac.new(
                self.secret_key,
                f"{base_hash}:{i}".encode(),
                hashlib.sha256
            ).hexdigest()
            encodings.append(variant)
        
        return encodings
    
    def index_document(self, doc_id: str, content: str):
        """Index with all possible encodings"""
        keywords = self.extract_keywords(content)
        for keyword in keywords:
            encodings = self.compute_encodings(keyword)
            for encoding in encodings:
                if encoding not in self.index:
                    self.index[encoding] = BloomFilter(10000, 0.001)
                self.index[encoding].add(doc_id)
    
    def search_uniform(self, keyword: str) -> BloomFilter:
        """Search with random encoding selection"""
        encodings = self.compute_encodings(keyword)
        # Randomly select one encoding
        chosen = random.choice(encodings)
        return self.index.get(chosen, BloomFilter(1, 1.0))
\end{lstlisting}

Now all queries look equally likely regardless of actual frequency!

\section{Stage 5: Correlation Breaking}

Hide correlations using tuple encoding:

\begin{lstlisting}[language=Python, caption={Tuple encoding for correlation hiding}]
class CorrelationHidingSearch(FrequencyHidingSearch):
    """Hide correlations between commonly co-searched terms"""
    
    def __init__(self, secret_key: bytes):
        super().__init__(secret_key)
        # Identify common pairs from query logs or corpus analysis
        self.common_pairs = {
            ("covid", "vaccine"),
            ("password", "leak"),
            ("encryption", "backdoor"),
            # ... more pairs ...
        }
    
    def index_tuple(self, doc_id: str, keywords: Set[str]):
        """Index common pairs as single tuples"""
        for kw1, kw2 in combinations(sorted(keywords), 2):
            pair = (kw1, kw2) if kw1 < kw2 else (kw2, kw1)
            if pair in self.common_pairs:
                # Encode pair as single unit
                tuple_encoding = hmac.new(
                    self.secret_key,
                    f"TUPLE:{pair[0]}:{pair[1]}".encode(),
                    hashlib.sha256
                ).hexdigest()
                
                if tuple_encoding not in self.index:
                    self.index[tuple_encoding] = BloomFilter(10000, 0.001)
                self.index[tuple_encoding].add(doc_id)
    
    def search_and(self, term1: str, term2: str) -> BloomFilter:
        """Search for documents with both terms"""
        # Check if this is a common pair
        pair = (term1, term2) if term1 < term2 else (term2, term1)
        if pair in self.common_pairs:
            # Use tuple encoding - correlation hidden!
            tuple_encoding = hmac.new(
                self.secret_key,
                f"TUPLE:{pair[0]}:{pair[1]}".encode(),
                hashlib.sha256
            ).hexdigest()
            return self.index.get(tuple_encoding, BloomFilter(1, 1.0))
        else:
            # Fall back to intersection
            result1 = self.search_uniform(term1)
            result2 = self.search_uniform(term2)
            return result1.intersect(result2)
\end{lstlisting}

\section{Stage 6: Complete Oblivious System}

Putting it all together with noise injection:

\begin{lstlisting}[language=Python, caption={Complete oblivious document search}]
class ObliviousDocumentSearch:
    """
    Complete privacy-preserving document search
    Combines all techniques for true oblivious operation
    """
    
    def __init__(self, secret_key: bytes):
        # Initialize with security parameters
        self.secret_key = secret_key
        self.search_engine = CorrelationHidingSearch(secret_key)
        
        # Estimate term frequencies from corpus
        self.estimate_frequencies()
        
        # Query cache for performance
        self.query_cache = {}
        
        # Noise generation parameters
        self.noise_rate = 0.2  # 20% fake queries
        
    def index_corpus(self, documents: List[Tuple[str, str]]):
        """Index all documents"""
        for doc_id, content in documents:
            # Extract keywords
            keywords = self.extract_keywords(content)
            
            # Index single terms with frequency hiding
            self.search_engine.index_document(doc_id, content)
            
            # Index common pairs as tuples
            self.search_engine.index_tuple(doc_id, keywords)
    
    def oblivious_search(self, query: str) -> List[str]:
        """
        Execute truly oblivious search
        Returns document IDs with bounded false positive rate
        """
        # Add noise queries
        queries = self.add_noise_queries([query])
        
        # Execute all queries (real + noise)
        all_results = []
        real_result = None
        
        for q in queries:
            if " AND " in q:
                terms = q.split(" AND ")
                result = self.search_engine.search_and(
                    terms[0].strip(), 
                    terms[1].strip()
                )
            else:
                result = self.search_engine.search_uniform(q)
            
            if q == query:
                real_result = result
            all_results.append(result)
        
        # Extract matching document IDs
        matching_docs = []
        for doc_id in self.all_doc_ids:
            if real_result.contains(doc_id):
                matching_docs.append(doc_id)
        
        return matching_docs
    
    def add_noise_queries(self, real_queries: List[str]) -> List[str]:
        """Add fake queries to hide patterns"""
        num_noise = int(len(real_queries) * self.noise_rate)
        
        # Select noise terms from vocabulary
        noise_terms = random.sample(
            list(self.search_engine.term_frequencies.keys()),
            min(num_noise, len(self.search_engine.term_frequencies))
        )
        
        # Mix real and noise queries
        all_queries = real_queries + noise_terms
        random.shuffle(all_queries)
        
        return all_queries
    
    def estimate_frequencies(self):
        """Estimate term frequencies from corpus"""
        # In practice: compute from actual corpus
        self.search_engine.term_frequencies = {
            "the": 0.07, "of": 0.04, "and": 0.03,  # Common
            "covid": 0.002, "vaccine": 0.001,       # Medium
            "whistleblower": 0.0001,                # Rare
            # ... more terms ...
        }
    
    def verify_privacy(self) -> Dict[str, float]:
        """Measure privacy properties"""
        # Statistical tests for uniformity
        return {
            "entropy": self.measure_entropy(),
            "distinguishability": self.measure_distinguishability(),
            "pattern_leakage": self.measure_patterns()
        }
\end{lstlisting}

\section{Security Analysis}

\subsection{What the Adversary Sees}

For each query, the server observes:
\begin{itemize}
\item A uniformly random encoding (indistinguishable from noise)
\item Result size (but padded/noised)
\item Timing (but includes fake queries)
\end{itemize}

\subsection{What the Adversary Cannot Learn}

With our oblivious system:
\begin{itemize}
\item \textbf{Query content}: All encodings equally likely
\item \textbf{Query frequency}: Uniform distribution achieved
\item \textbf{Correlations}: Tuple encoding hides relationships
\item \textbf{User interest}: Noise queries obscure patterns
\end{itemize}

\subsection{Formal Privacy Guarantee}

\begin{theorem}[Oblivious Search Privacy]
Given polynomial-time adversary $\mathcal{A}$ observing queries $Q = \{q_1, \ldots, q_n\}$:
$$\Pr[\mathcal{A} \text{ distinguishes } Q \text{ from random}] \leq \frac{1}{2} + \text{negl}(\lambda)$$
where $\lambda$ is the security parameter.
\end{theorem}

\begin{proof}[Proof Sketch]
By construction:
\begin{enumerate}
\item Each encoding is uniformly distributed
\item Encodings are independent across queries
\item Noise queries make timing analysis infeasible
\item Result sizes are bounded and noised
\end{enumerate}
Therefore, observations are computationally indistinguishable from random.
\end{proof}

\section{Performance Analysis}

Our oblivious system achieves:

\begin{center}
\begin{tabular}{|l|c|c|c|}
\hline
\textbf{Metric} & \textbf{Naive} & \textbf{Oblivious} & \textbf{Improvement} \\
\hline
Storage & 32 GB & 1.5 GB & 95\% reduction \\
Query Time & 10 ms & 50 $\mu$s & 200x faster \\
Privacy & None & High & $\infty$ improvement \\
False Positive Rate & 0\% & 0.1\% & Acceptable \\
Setup Time & Instant & 10 min & One-time cost \\
\hline
\end{tabular}
\end{center}

\section{Handling False Positives}

False positives are inherent but manageable:

\begin{lstlisting}[language=Python, caption={Client-side false positive filtering}]
class ClientSideFilter:
    """Filter false positives on client (trusted) side"""
    
    def filter_results(self, 
                      query: str, 
                      candidate_docs: List[str], 
                      fetch_func) -> List[str]:
        """
        Filter false positives by fetching and checking
        
        Args:
            query: Original search query
            candidate_docs: Document IDs from oblivious search
            fetch_func: Function to retrieve document content
        """
        true_matches = []
        
        for doc_id in candidate_docs:
            # Fetch document (encrypted channel)
            content = fetch_func(doc_id)
            
            # Check if actually contains search terms
            if self.verify_match(query, content):
                true_matches.append(doc_id)
        
        return true_matches
    
    def verify_match(self, query: str, content: str) -> bool:
        """Verify document actually matches query"""
        keywords = query.lower().split()
        content_lower = content.lower()
        
        if " AND " in query:
            # All terms must be present
            terms = query.split(" AND ")
            return all(term.strip().lower() in content_lower 
                      for term in terms)
        elif " OR " in query:
            # At least one term must be present
            terms = query.split(" OR ")
            return any(term.strip().lower() in content_lower 
                      for term in terms)
        else:
            # Single term
            return query.lower() in content_lower
\end{lstlisting}

\section{Testing and Validation}

\begin{lstlisting}[language=Python, caption={Testing privacy properties}]
import numpy as np
from scipy import stats

def test_uniformity(search_system, queries: List[str], trials: int = 1000):
    """Test if encodings are uniformly distributed"""
    encoding_counts = defaultdict(int)
    
    for _ in range(trials):
        for query in queries:
            # Get encoding (different each time)
            encoding = search_system.get_encoding(query)
            encoding_counts[encoding] += 1
    
    # Chi-square test for uniformity
    observed = list(encoding_counts.values())
    expected = [trials * len(queries) / len(encoding_counts)] * len(observed)
    
    chi2, p_value = stats.chisquare(observed, expected)
    
    # p > 0.05 suggests uniform distribution
    return p_value > 0.05

def test_independence(search_system, query_pairs: List[Tuple[str, str]]):
    """Test if query encodings are independent"""
    correlations = []
    
    for q1, q2 in query_pairs:
        # Get multiple encodings
        enc1 = [search_system.get_encoding(q1) for _ in range(100)]
        enc2 = [search_system.get_encoding(q2) for _ in range(100)]
        
        # Compute correlation
        corr = np.corrcoef(
            [hash(e) % 1000 for e in enc1],
            [hash(e) % 1000 for e in enc2]
        )[0, 1]
        
        correlations.append(abs(corr))
    
    # Average correlation should be near 0
    return np.mean(correlations) < 0.05
\end{lstlisting}

\section{Common Pitfalls and Solutions}

\begin{enumerate}
\item \textbf{Pitfall}: Using same encoding repeatedly
   \textbf{Solution}: Always generate fresh random encoding

\item \textbf{Pitfall}: Predictable noise queries
   \textbf{Solution}: Use cryptographically secure randomness

\item \textbf{Pitfall}: Leaking through timing
   \textbf{Solution}: Constant-time operations, batch processing

\item \textbf{Pitfall}: Result size leakage
   \textbf{Solution}: Pad results or return fixed-size batches

\item \textbf{Pitfall}: Not updating encodings
   \textbf{Solution}: Periodic re-encoding with fresh keys
\end{enumerate}

\section{Exercises}

\begin{enumerate}
\item \textbf{Implementation}: Build the complete ObliviousDocumentSearch system and test with 10,000 documents.

\item \textbf{Analysis}: Measure the actual false positive rate of your implementation. How does it compare to the theoretical prediction?

\item \textbf{Optimization}: Improve query performance using caching while maintaining privacy. Hint: Cache encodings, not queries.

\item \textbf{Extension}: Add support for phrase queries ("new york") while maintaining obliviousness.

\item \textbf{Security Challenge}: Design an attack that tries to distinguish between medical and financial queries. How does the system defend against it?
\end{enumerate}

\section{Advanced Enhancements: Achieving Complete Obliviousness}

The system we've built provides significant privacy improvements, but careful analysis reveals remaining information leaks. In this section, we'll address three critical enhancements that achieve \textit{complete} obliviousness.

\subsection{Fully Oblivious Input-Output}

\section{Enhancement: Fully Oblivious Input-Output}
\label{sec:oblivious-io}

\subsection{The Observable Output Problem}

Our current implementation has a critical weakness:

\begin{lstlisting}[language=Python, caption={Current approach - outputs are observable}]
def search(self, query: str) -> bool:  # PROBLEM: Boolean output is observable!
    encoded_query = self.encode(query)
    bloom_result = self.index[encoded_query]
    return bloom_result.contains(doc_id)  # Returns True/False - OBSERVABLE!
\end{lstlisting}

Even with oblivious queries, returning plaintext booleans leaks information:
\begin{itemize}
\item Result patterns reveal query selectivity
\item True/False sequences can be correlated
\item Timing of True results reveals investigation progress
\end{itemize}

\subsection{Bernoulli Map Construction for Oblivious Outputs}

Instead of Bloom filters (which output observable booleans), we use Bernoulli maps that output encoded values:

\begin{lstlisting}[language=Python, caption={Bernoulli map with encoded outputs}]
class BernoulliMap:
    """Maps encoded inputs to encoded outputs obliviously"""
    
    def __init__(self, secret_key: bytes):
        self.secret_key = secret_key
        # Map from input encodings to output encodings
        self.encoding_map = {}
        
    def insert(self, key: str, value: Any):
        """Store key-value pair with encoded representations"""
        # Generate multiple input encodings
        input_encodings = self.generate_encodings(key)
        
        # Generate encoded output (not plaintext!)
        output_encoding = self.encode_output(value)
        
        # Store all input encodings -> same output encoding
        for enc_in in input_encodings:
            self.encoding_map[enc_in] = output_encoding
    
    def lookup(self, encoded_key: bytes) -> bytes:
        """Returns ENCODED output, not plaintext"""
        # Return encoded value or encoded "not found"
        return self.encoding_map.get(
            encoded_key,
            self.encode_output(None)  # Encoded null
        )
    
    def encode_output(self, value: Any) -> bytes:
        """Encode output values uniformly"""
        if value is None:
            # Special encoding for "not found"
            data = b"NULL" + self.secret_key
        else:
            # Encode actual value
            data = str(value).encode() + self.secret_key
        
        # Generate uniform encoding
        return hashlib.sha256(data).digest()
    
    def decode_output(self, encoded: bytes, expected_values: List[Any]) -> Any:
        """Client-side decoding with expected value set"""
        for value in expected_values:
            if self.encode_output(value) == encoded:
                return value
        return None  # Not in expected set
\end{lstlisting}

\subsection{End-to-End Oblivious Computation}

Now both inputs and outputs remain encoded throughout:

\begin{lstlisting}[language=Python, caption={Fully oblivious search system}]
class FullyObliviousSearch:
    """Complete oblivious computation - encoded inputs AND outputs"""
    
    def __init__(self, secret_key: bytes):
        self.secret_key = secret_key
        self.bernoulli_map = BernoulliMap(secret_key)
        
    def oblivious_search(self, query: str) -> bytes:
        """
        Fully oblivious: 
        - Input: plaintext query (on client)
        - Processing: encoded query (on server) 
        - Output: encoded result (from server)
        - Decoding: back to plaintext (on client)
        """
        # Client side: encode query
        encoded_query = self.encode_query(query)
        
        # Server side: process encoded, return encoded
        encoded_result = self.server_process(encoded_query)
        
        # Client side: decode result
        # (Would happen on trusted client machine)
        return encoded_result  # Still encoded!
    
    def server_process(self, encoded_query: bytes) -> bytes:
        """
        Server NEVER sees plaintext
        Input: encoded query
        Output: encoded result
        """
        # Server just does lookup in Bernoulli map
        # Doesn't know what query means or what result means
        return self.bernoulli_map.lookup(encoded_query)
    
    def client_decode(self, encoded_result: bytes) -> bool:
        """Decode on trusted client only"""
        # Client knows expected values
        expected = [True, False]  # For boolean results
        
        # Try each possibility
        for value in expected:
            test_encoding = self.encode_output(value)
            if test_encoding == encoded_result:
                return value
        
        raise ValueError("Unexpected encoded result")
\end{lstlisting}

\subsection{Bernoulli Tuples for Correlation Hiding}

Beyond pairs, we can encode arbitrary tuples to hide complex correlations:

\begin{lstlisting}[language=Python, caption={Generalized tuple encoding}]
class BernoulliTupleEncoder:
    """Hide correlations through tuple encoding"""
    
    def __init__(self, correlation_sets):
        """
        correlation_sets: List of commonly correlated term sets
        Example: [
            ("covid", "vaccine", "trial"),
            ("password", "leak", "breach"),
            ("financial", "fraud", "investigation")
        ]
        """
        self.correlation_sets = correlation_sets
        self.tuple_encodings = self._precompute_tuples()
    
    def encode_query(self, terms: List[str]) -> bytes:
        """Encode query, using tuples when possible"""
        # Sort for canonical ordering
        terms_sorted = tuple(sorted(terms))
        
        # Check if this is a known correlation set
        if terms_sorted in self.correlation_sets:
            # Use single tuple encoding
            return self._encode_tuple(terms_sorted)
        
        # Check for subsets that form tuples
        for size in range(len(terms), 1, -1):
            for subset in combinations(terms, size):
                subset_sorted = tuple(sorted(subset))
                if subset_sorted in self.correlation_sets:
                    # Encode subset as tuple, rest individually
                    tuple_enc = self._encode_tuple(subset_sorted)
                    remaining = [t for t in terms if t not in subset]
                    other_encs = [self._encode_single(t) for t in remaining]
                    
                    # Combine encodings
                    return self._combine_encodings([tuple_enc] + other_encs)
        
        # No correlations found, encode individually
        return self._combine_encodings([self._encode_single(t) for t in terms])
    
    def _encode_tuple(self, term_tuple: Tuple[str]) -> bytes:
        """Create single encoding for entire tuple"""
        # Single encoding hides correlation!
        data = ":".join(term_tuple).encode()
        return hashlib.sha256(data).digest()
    
    def _combine_encodings(self, encodings: List[bytes]) -> bytes:
        """Combine multiple encodings obliviously"""
        # XOR for simplicity (could use more sophisticated combination)
        result = encodings[0]
        for enc in encodings[1:]:
            result = bytes(a ^ b for a, b in zip(result, enc))
        return result
\end{lstlisting}

\subsection{Two-Level Hash Schemes}

For more sophisticated constructions, we can use two-level perfect hashing with seed search:

\begin{lstlisting}[language=Python, caption={Two-level perfect hashing for Bernoulli maps}]
class TwoLevelBernoulliMap:
    """
    Two-level scheme:
    - Level 1: Bins with minimal collisions
    - Level 2: Per-bin seed search for perfect mapping
    """
    
    def __init__(self, items: List[Tuple[str, Any]], max_attempts=1000):
        self.level1_bins = defaultdict(list)
        self.level2_seeds = {}
        self.level2_maps = {}
        
        # Level 1: Distribute into bins
        for key, value in items:
            bin_id = self._hash_to_bin(key)
            self.level1_bins[bin_id].append((key, value))
        
        # Level 2: Find perfect hash seed for each bin
        for bin_id, bin_items in self.level1_bins.items():
            seed = self._find_perfect_seed(bin_items, max_attempts)
            if seed is None:
                # Fall back to standard encoding
                seed = 0
            
            self.level2_seeds[bin_id] = seed
            self.level2_maps[bin_id] = self._build_perfect_map(bin_items, seed)
    
    def _find_perfect_seed(self, items: List[Tuple[str, Any]], 
                          max_attempts: int) -> Optional[int]:
        """
        Search for seed that gives perfect (or near-perfect) mapping
        Most input-output pairs should map correctly
        """
        n = len(items)
        if n == 0:
            return 0
        
        # Target: at least 90% correct mappings
        target_correct = int(0.9 * n)
        
        for seed in range(max_attempts):
            positions = {}
            collisions = 0
            
            for key, value in items:
                pos = self._hash_with_seed(key, seed) % (2 * n)
                if pos in positions:
                    collisions += 1
                else:
                    positions[pos] = (key, value)
            
            # Check if this seed is good enough
            correct_mappings = n - collisions
            if correct_mappings >= target_correct:
                return seed
        
        return None  # No good seed found
    
    def _build_perfect_map(self, items: List[Tuple[str, Any]], 
                          seed: int) -> Dict[int, bytes]:
        """Build encoded map with the chosen seed"""
        n = len(items)
        encoded_map = {}
        
        for key, value in items:
            pos = self._hash_with_seed(key, seed) % (2 * n)
            # Encode the output value
            encoded_value = self._encode_value(value)
            encoded_map[pos] = encoded_value
        
        return encoded_map
    
    def lookup(self, key: str) -> bytes:
        """Two-level lookup returning encoded result"""
        # Level 1: Find bin
        bin_id = self._hash_to_bin(key)
        
        # Level 2: Use bin's seed for lookup
        seed = self.level2_seeds.get(bin_id, 0)
        bin_map = self.level2_maps.get(bin_id, {})
        
        n = len(self.level1_bins[bin_id])
        if n == 0:
            return self._encode_value(None)
        
        pos = self._hash_with_seed(key, seed) % (2 * n)
        
        # Return encoded value (or encoded null)
        return bin_map.get(pos, self._encode_value(None))
\end{lstlisting}

\subsection{Extending to Arbitrary Computations}

The pattern generalizes to any computation:

\begin{lstlisting}[language=Python, caption={Oblivious computation framework}]
class ObliviousComputation:
    """Framework for arbitrary oblivious computations"""
    
    def __init__(self, computation_type: str):
        self.computation_type = computation_type
        self.input_encoder = BernoulliEncoder()
        self.output_encoder = BernoulliEncoder()
        
    def oblivious_compute(self, 
                         encoded_inputs: List[bytes], 
                         operation: str) -> bytes:
        """
        Perform computation on encoded inputs,
        return encoded output
        """
        if self.computation_type == "boolean":
            return self._boolean_op(encoded_inputs, operation)
        elif self.computation_type == "arithmetic":
            return self._arithmetic_op(encoded_inputs, operation)
        elif self.computation_type == "comparison":
            return self._comparison_op(encoded_inputs, operation)
        else:
            raise ValueError(f"Unknown computation: {self.computation_type}")
    
    def _boolean_op(self, inputs: List[bytes], op: str) -> bytes:
        """Boolean operations on encoded values"""
        # Server doesn't decode - works directly on encodings
        if op == "AND":
            # Intersection of Bernoulli sets
            result = self._intersect_encodings(inputs)
        elif op == "OR":
            # Union of Bernoulli sets
            result = self._union_encodings(inputs)
        elif op == "NOT":
            # Complement of Bernoulli set
            result = self._complement_encoding(inputs[0])
        
        return result  # Still encoded!
    
    def _intersect_encodings(self, encodings: List[bytes]) -> bytes:
        """Intersection without decoding"""
        # Bit-wise AND for bloom-filter-like encodings
        result = encodings[0]
        for enc in encodings[1:]:
            result = bytes(a & b for a, b in zip(result, enc))
        return result
    
    def _union_encodings(self, encodings: List[bytes]) -> bytes:
        """Union without decoding"""
        # Bit-wise OR for bloom-filter-like encodings
        result = encodings[0]
        for enc in encodings[1:]:
            result = bytes(a | b for a, b in zip(result, enc))
        return result
\end{lstlisting}

\subsection{Key Insights}

This enhanced approach provides:

1. **End-to-end obliviousness**: Both inputs and outputs remain encoded
2. **Correlation hiding**: Through generalized tuple encoding
3. **Efficient constructions**: Two-level hashing with seed search
4. **Computational completeness**: Extends to arbitrary operations
5. **No observable patterns**: Server never sees plaintext values or results

The server becomes a truly oblivious processor - it performs computations on encodings without ever learning what they represent.

\subsection{Oblivious Operations} 

\section{Fully Oblivious Operations}
\label{sec:fully-oblivious-ops}

\subsection{The Operation Leakage Problem}

Even with encoded inputs and outputs, if the server knows which operation is being performed, it learns information:

\begin{lstlisting}[language=Python, caption={Problem: Operations are observable}]
def server_process(encoded_a, encoded_b, operation="AND"):  # LEAK: operation visible!
    if operation == "AND":
        return self.and_operation(encoded_a, encoded_b)
    elif operation == "OR":
        return self.or_operation(encoded_a, encoded_b)
    # Server knows logical structure of computation!
\end{lstlisting}

\subsection{Oblivious Operations}

The solution: encode the operations themselves!

\begin{lstlisting}[language=Python, caption={Oblivious boolean operations}]
class ObliviousBoolean:
    """Oblivious representation of boolean values and operations"""
    
    def __init__(self, secret_key: bytes):
        self.secret_key = secret_key
        # Pre-compute operation encodings
        self.operation_encodings = self._generate_op_encodings()
        
    def encode_boolean(self, value: bool) -> bytes:
        """Encode boolean uniformly"""
        # Multiple encodings per value for uniformity
        nonce = random.randint(0, 100)
        data = f"{value}:{nonce}:{self.secret_key.hex()}".encode()
        return hashlib.sha256(data).digest()
    
    def encode_operation(self, op: str) -> bytes:
        """Operations are also encoded!"""
        # Server doesn't know if it's AND, OR, XOR, etc.
        nonce = random.randint(0, 100)
        data = f"{op}:{nonce}:{self.secret_key.hex()}".encode()
        return hashlib.sha256(data).digest()
    
    def oblivious_binary_op(self, 
                           encoded_op: bytes,
                           encoded_a: bytes, 
                           encoded_b: bytes) -> bytes:
        """
        Perform operation without knowing what it is!
        All inputs and outputs are encoded.
        """
        # Server executes without knowing operation type
        # Uses homomorphic-like property of encodings
        
        # Combine all encodings into single computation
        combined = self._combine_for_computation(
            encoded_op, encoded_a, encoded_b
        )
        
        # Result is still encoded
        return self._execute_oblivious(combined)
    
    def _execute_oblivious(self, combined: bytes) -> bytes:
        """
        Execute without revealing operation or values
        Uses pre-computed lookup table
        """
        # Hash to index in pre-computed results table
        index = int.from_bytes(combined[:8], 'big') % len(self.result_table)
        return self.result_table[index]
\end{lstlisting}

\subsection{Oblivious Pairs vs Pairs of Oblivious Values}

Critical distinction for correlation hiding:

\begin{lstlisting}[language=Python, caption={Two approaches to encoding pairs}]
class ObliviousPairEncoding:
    """Different encodings reveal different information"""
    
    def pair_of_oblivious(self, a: bool, b: bool) -> Tuple[bytes, bytes]:
        """
        Encode each value separately
        PROBLEM: Correlations visible through patterns
        """
        encoded_a = self.encode_boolean(a)
        encoded_b = self.encode_boolean(b)
        return (encoded_a, encoded_b)  # Two separate encodings
    
    def oblivious_pair(self, a: bool, b: bool) -> bytes:
        """
        Encode the pair as single unit
        BENEFIT: Correlation hidden!
        """
        # Single encoding for the pair
        pair_value = (a, b)
        data = f"{pair_value}:{self.secret_key.hex()}".encode()
        return hashlib.sha256(data).digest()  # One encoding for both!
    
    def demonstrate_difference(self):
        """Show why oblivious pairs hide correlations"""
        
        # Scenario: Always query (term1=True, term2=True) together
        correleted_queries = [(True, True)] * 100
        
        # Method 1: Pair of oblivious values
        separate_encodings = [
            self.pair_of_oblivious(a, b) 
            for a, b in correleted_queries
        ]
        # PROBLEM: Statistical analysis reveals both values usually True
        
        # Method 2: Oblivious pair
        pair_encodings = [
            self.oblivious_pair(a, b)
            for a, b in correleted_queries
        ]
        # BENEFIT: Just see random noise, no correlation visible!
\end{lstlisting}

\subsection{Universal Oblivious Computation}

The complete framework where everything is oblivious:

\begin{lstlisting}[language=Python, caption={Universal oblivious computation}]
class UniversalObliviousComputer:
    """
    Everything is oblivious:
    - Input values
    - Operations  
    - Intermediate results
    - Output values
    """
    
    def __init__(self, computation_space_size: int = 2**20):
        """
        Pre-compute results for all possible computations
        in the defined space
        """
        self.computation_table = self._build_computation_table(
            computation_space_size
        )
        
    def _build_computation_table(self, size: int) -> Dict[bytes, bytes]:
        """
        Pre-compute all possible results
        Maps: (encoded_op, encoded_inputs) -> encoded_output
        """
        table = {}
        
        # For each possible operation encoding
        for op in ["AND", "OR", "XOR", "NAND", "NOR"]:
            op_encodings = self._generate_encodings(op, count=100)
            
            # For each possible input combination
            for a in [True, False]:
                for b in [True, False]:
                    # Generate input encodings
                    a_encodings = self._generate_encodings(a, count=100)
                    b_encodings = self._generate_encodings(b, count=100)
                    
                    # Compute actual result
                    result = self._compute(op, a, b)
                    result_encoding = self._generate_encoding(result)
                    
                    # Store all encoding combinations
                    for op_enc in op_encodings:
                        for a_enc in a_encodings:
                            for b_enc in b_encodings:
                                key = self._combine_encodings(
                                    op_enc, a_enc, b_enc
                                )
                                table[key] = result_encoding
        
        return table
    
    def oblivious_compute(self,
                         encoded_operation: bytes,
                         encoded_inputs: bytes) -> bytes:
        """
        Server executes this:
        - Doesn't know the operation
        - Doesn't know the inputs
        - Doesn't know the output
        Just a lookup in pre-computed table!
        """
        key = self._combine_encodings(encoded_operation, encoded_inputs)
        
        # Simple table lookup - reveals nothing!
        return self.computation_table.get(
            key,
            self._encode_error()  # Encoded error value
        )
\end{lstlisting}

\subsection{Oblivious Function Composition}

Operations can be composed while maintaining obliviousness:

\begin{lstlisting}[language=Python, caption={Composing oblivious operations}]
class ObliviousFunctionComposition:
    """Compose functions without revealing structure"""
    
    def encode_function(self, func_description: str) -> bytes:
        """Encode entire function as single unit"""
        # E.g., "AND(OR(a,b), NOT(c))" encoded as one blob
        return hashlib.sha256(func_description.encode()).digest()
    
    def oblivious_compose(self, 
                         encoded_funcs: List[bytes],
                         encoded_inputs: bytes) -> bytes:
        """
        Compose multiple functions obliviously
        Server doesn't know composition structure
        """
        # Method 1: Encode composition pattern
        composition_encoding = self._encode_composition_pattern(encoded_funcs)
        
        # Method 2: Use garbled circuits approach
        garbled_circuit = self._create_garbled_circuit(
            composition_encoding,
            encoded_inputs
        )
        
        # Execute without knowing structure
        return self._evaluate_garbled(garbled_circuit)
    
    def oblivious_recursive(self,
                          encoded_func: bytes,
                          encoded_base: bytes,
                          encoded_recursive_case: bytes,
                          depth: int) -> bytes:
        """Even recursion can be oblivious!"""
        
        result = encoded_base
        for _ in range(depth):
            # Apply function obliviously
            result = self.oblivious_compute(
                encoded_func,
                self._combine_encodings(result, encoded_recursive_case)
            )
        
        return result  # Still encoded!
\end{lstlisting}

\subsection{Practical Implementation with Garbled Circuits}

For practical oblivious operations, we can use garbled circuit techniques:

\begin{lstlisting}[language=Python, caption={Garbled circuit approach}]
class GarbledObliviousComputation:
    """Use garbled circuits for oblivious operations"""
    
    def __init__(self):
        self.wire_labels = {}  # Maps wire -> (label_0, label_1)
        self.garbled_gates = {}  # Encrypted truth tables
        
    def garble_operation(self, operation: str) -> Dict[str, bytes]:
        """
        Create garbled circuit for operation
        Server can evaluate without knowing operation type
        """
        # Generate random labels for wires
        input_a_labels = (os.urandom(16), os.urandom(16))  # For 0, 1
        input_b_labels = (os.urandom(16), os.urandom(16))
        output_labels = (os.urandom(16), os.urandom(16))
        
        # Create garbled truth table
        garbled_table = []
        for a_val in [0, 1]:
            for b_val in [0, 1]:
                # Compute actual result
                if operation == "AND":
                    result = a_val & b_val
                elif operation == "OR":
                    result = a_val | b_val
                elif operation == "XOR":
                    result = a_val ^ b_val
                
                # Encrypt output label with input labels
                a_label = input_a_labels[a_val]
                b_label = input_b_labels[b_val]
                out_label = output_labels[result]
                
                # Double encryption
                encrypted = self._encrypt(
                    self._encrypt(out_label, a_label),
                    b_label
                )
                garbled_table.append(encrypted)
        
        # Shuffle table to hide operation
        random.shuffle(garbled_table)
        
        return {
            'table': garbled_table,
            'input_labels': (input_a_labels, input_b_labels),
            'output_labels': output_labels
        }
    
    def evaluate_garbled(self,
                        garbled_circuit: Dict,
                        input_a_label: bytes,
                        input_b_label: bytes) -> bytes:
        """
        Evaluate garbled circuit
        Server doesn't know:
        - What operation this implements
        - What values the labels represent
        - What the output means
        """
        # Try decrypting each table entry
        for encrypted_entry in garbled_circuit['table']:
            try:
                # Attempt double decryption
                output_label = self._decrypt(
                    self._decrypt(encrypted_entry, input_a_label),
                    input_b_label
                )
                return output_label  # Found valid output
            except:
                continue  # Wrong input labels for this entry
        
        raise ValueError("Invalid input labels")
\end{lstlisting}

\subsection{The Complete Oblivious System}

Putting it all together:

\begin{lstlisting}[language=Python, caption={Complete oblivious system}]
class CompletelyObliviousSystem:
    """
    Nothing is visible to the server:
    - Not the data
    - Not the operations
    - Not the structure
    - Not the results
    """
    
    def client_side_prepare(self, 
                          query: str,
                          operation: str,
                          correlation_aware: bool = True) -> bytes:
        """Client prepares oblivious computation"""
        
        if correlation_aware:
            # Encode correlated terms together
            encoded_input = self.encode_oblivious_tuple(
                self.extract_correlated_terms(query)
            )
        else:
            # Encode separately
            encoded_input = self.encode_separate_terms(query)
        
        # Encode the operation
        encoded_op = self.encode_operation(operation)
        
        # Combine into single oblivious computation request
        return self.create_oblivious_request(encoded_op, encoded_input)
    
    def server_side_process(self, oblivious_request: bytes) -> bytes:
        """
        Server processes without knowing anything
        Just a deterministic function: bytes -> bytes
        """
        # Could be:
        # - Table lookup
        # - Garbled circuit evaluation  
        # - Homomorphic computation
        # - ORAM access
        # Server doesn't know which!
        
        return self.oblivious_computation_engine.process(oblivious_request)
    
    def client_side_decode(self, 
                         oblivious_result: bytes,
                         expected_type: str) -> Any:
        """Client decodes result"""
        
        if expected_type == "boolean":
            return self.decode_oblivious_boolean(oblivious_result)
        elif expected_type == "tuple":
            return self.decode_oblivious_tuple(oblivious_result)
        elif expected_type == "list":
            return self.decode_oblivious_list(oblivious_result)
        else:
            return self.decode_generic(oblivious_result)
\end{lstlisting}

\subsection{Key Insight: Complete Obliviousness}

The crucial realization: **everything** can and should be oblivious:

1. **Values**: Encoded uniformly (Bernoulli types)
2. **Operations**: Encoded so server doesn't know AND vs OR vs XOR
3. **Composition**: Structure of computation hidden
4. **Correlations**: Tuple encoding hides relationships
5. **Control flow**: Even branches can be oblivious (compute both, select one)

The server becomes a pure function evaluator: `bytes → bytes` with no semantic understanding of what it's computing. This achieves true oblivious computing where the server learns nothing beyond the size of inputs and outputs.

\subsection{Adaptive Frequency Normalization}

\section{Adaptive Frequency Normalization}
\label{sec:adaptive-frequency}

\subsection{The Distribution Tracking Problem}

Even with oblivious operations, frequency analysis remains a threat. The key insight: track distributions at every stage and adaptively adjust encodings to maintain uniformity.

\begin{lstlisting}[language=Python, caption={Distribution tracking through computation}]
class DistributionTracker:
    """Track distributions through oblivious computation pipeline"""
    
    def __init__(self, window_size: int = 10000):
        self.window_size = window_size
        # Track distributions at each stage
        self.input_distribution = defaultdict(int)
        self.encoding_distribution = defaultdict(int)
        self.intermediate_distributions = defaultdict(lambda: defaultdict(int))
        self.output_distribution = defaultdict(int)
        
    def observe_input(self, value: Any) -> None:
        """Track raw input distribution"""
        self.input_distribution[value] += 1
        self._update_encoding_strategy()
    
    def observe_encoding(self, value: Any, encoding: bytes) -> None:
        """Track how values map to encodings"""
        self.encoding_distribution[encoding] += 1
        
    def observe_propagation(self, stage: str, encoding: bytes) -> None:
        """Track distribution at each computation stage"""
        self.intermediate_distributions[stage][encoding] += 1
        
    def _update_encoding_strategy(self) -> None:
        """Dynamically adjust encoding counts based on observed frequencies"""
        if len(self.input_distribution) < 10:
            return  # Not enough data
            
        # Calculate frequency for each input
        total = sum(self.input_distribution.values())
        frequencies = {
            value: count / total 
            for value, count in self.input_distribution.items()
        }
        
        # Adjust encoding counts inversely to frequency
        self.encoding_counts = {}
        max_encodings = 1000  # Cap for practical reasons
        
        for value, freq in frequencies.items():
            # More encodings for rare values (1/p principle)
            if freq > 0:
                count = min(max_encodings, int(1.0 / freq))
                self.encoding_counts[value] = max(1, count)
\end{lstlisting}

\subsection{Adaptive Encoding Strategy}

Dynamically adjust the number of encodings per value based on observed distributions:

\begin{lstlisting}[language=Python, caption={Adaptive encoding based on frequency}]
class AdaptiveFrequencyNormalizer:
    """
    Adaptively normalize frequencies to achieve uniform distribution
    across all computation stages
    """
    
    def __init__(self, target_uniformity: float = 0.95):
        self.target_uniformity = target_uniformity
        self.tracker = DistributionTracker()
        self.encoding_pool = {}  # Pre-computed encodings
        self.rebalance_threshold = 0.1  # Rebalance when skew > 10%
        
    def adaptive_encode(self, value: Any) -> bytes:
        """Encode with adaptive frequency normalization"""
        # Track input
        self.tracker.observe_input(value)
        
        # Get current encoding count for this value
        encoding_count = self.get_adaptive_encoding_count(value)
        
        # Generate or retrieve encodings
        if value not in self.encoding_pool:
            self.encoding_pool[value] = self.generate_encoding_pool(
                value, encoding_count
            )
        
        # Check if we need more encodings due to frequency changes
        if len(self.encoding_pool[value]) < encoding_count:
            # Generate additional encodings
            additional = encoding_count - len(self.encoding_pool[value])
            new_encodings = self.generate_encoding_pool(value, additional)
            self.encoding_pool[value].extend(new_encodings)
        
        # Select encoding uniformly from pool
        encoding = random.choice(self.encoding_pool[value])
        
        # Track the encoding
        self.tracker.observe_encoding(value, encoding)
        
        # Check if distribution is becoming skewed
        if self.is_distribution_skewed():
            self.rebalance_encodings()
        
        return encoding
    
    def get_adaptive_encoding_count(self, value: Any) -> int:
        """
        Calculate encoding count based on observed frequency
        and current distribution skew
        """
        # Get base count from frequency
        freq = self.get_observed_frequency(value)
        if freq == 0:
            base_count = 100  # Default for unseen values
        else:
            base_count = int(1.0 / freq)
        
        # Adjust based on current distribution uniformity
        uniformity = self.measure_uniformity()
        if uniformity < self.target_uniformity:
            # Need more encodings to improve uniformity
            adjustment_factor = self.target_uniformity / max(uniformity, 0.1)
            base_count = int(base_count * adjustment_factor)
        
        return min(1000, max(1, base_count))  # Practical bounds
    
    def measure_uniformity(self) -> float:
        """
        Measure how uniform the encoding distribution is
        1.0 = perfectly uniform, 0.0 = completely skewed
        """
        if not self.tracker.encoding_distribution:
            return 1.0
        
        # Calculate entropy of encoding distribution
        total = sum(self.tracker.encoding_distribution.values())
        entropy = 0
        max_entropy = 0
        
        for encoding, count in self.tracker.encoding_distribution.items():
            p = count / total
            if p > 0:
                entropy -= p * math.log2(p)
        
        # Maximum entropy for this number of encodings
        n = len(self.tracker.encoding_distribution)
        if n > 1:
            max_entropy = math.log2(n)
            return entropy / max_entropy if max_entropy > 0 else 0
        
        return 1.0
    
    def rebalance_encodings(self) -> None:
        """
        Rebalance encoding pools when distribution becomes skewed
        """
        print(f"Rebalancing: uniformity = {self.measure_uniformity():.3f}")
        
        # Identify over-represented encodings
        total = sum(self.tracker.encoding_distribution.values())
        mean_count = total / len(self.tracker.encoding_distribution)
        
        for value in self.encoding_pool:
            current_encodings = self.encoding_pool[value]
            
            # Check how often each encoding is used
            encoding_usage = defaultdict(int)
            for enc in current_encodings:
                encoding_usage[enc] = self.tracker.encoding_distribution.get(enc, 0)
            
            # Remove over-used encodings
            overused = [
                enc for enc, count in encoding_usage.items()
                if count > mean_count * 1.5  # 50% above mean
            ]
            
            if overused:
                # Remove overused encodings
                self.encoding_pool[value] = [
                    enc for enc in current_encodings
                    if enc not in overused
                ]
                
                # Generate fresh encodings to replace
                new_count = len(overused)
                new_encodings = self.generate_encoding_pool(value, new_count)
                self.encoding_pool[value].extend(new_encodings)
\end{lstlisting}

\subsection{Propagation-Aware Encoding}

Track how encodings propagate through computation stages:

\begin{lstlisting}[language=Python, caption={Tracking distribution through computation}]
class PropagationAwareEncoder:
    """
    Track how encodings flow through computation
    and adjust to maintain uniformity at each stage
    """
    
    def __init__(self):
        self.stage_distributions = {}
        self.propagation_graph = defaultdict(lambda: defaultdict(list))
        
    def track_computation(self, 
                         input_encoding: bytes,
                         operation: str,
                         output_encoding: bytes,
                         stage: str) -> None:
        """Track how encodings propagate through operations"""
        # Record the transformation
        self.propagation_graph[stage][operation].append({
            'input': input_encoding,
            'output': output_encoding,
            'timestamp': time.time()
        })
        
        # Update stage distribution
        if stage not in self.stage_distributions:
            self.stage_distributions[stage] = defaultdict(int)
        self.stage_distributions[stage][output_encoding] += 1
        
        # Check for distribution anomalies
        self.detect_propagation_patterns(stage)
    
    def detect_propagation_patterns(self, stage: str) -> Dict[str, float]:
        """
        Detect patterns in how encodings propagate
        that might leak information
        """
        patterns = {}
        
        if stage not in self.propagation_graph:
            return patterns
        
        # Check for convergence (many inputs -> few outputs)
        for operation in self.propagation_graph[stage]:
            transforms = self.propagation_graph[stage][operation]
            
            input_diversity = len(set(t['input'] for t in transforms))
            output_diversity = len(set(t['output'] for t in transforms))
            
            if input_diversity > 0:
                convergence_ratio = output_diversity / input_diversity
                if convergence_ratio < 0.5:
                    patterns[f"{stage}:{operation}:convergence"] = convergence_ratio
        
        # Check for frequency amplification
        stage_dist = self.stage_distributions.get(stage, {})
        if stage_dist:
            total = sum(stage_dist.values())
            max_freq = max(stage_dist.values()) / total
            if max_freq > 0.1:  # More than 10% concentration
                patterns[f"{stage}:concentration"] = max_freq
        
        return patterns
    
    def adaptive_stage_encoding(self,
                               value: Any,
                               stage: str,
                               previous_encoding: Optional[bytes] = None) -> bytes:
        """
        Generate encoding based on stage-specific distribution
        """
        # Get distribution at this stage
        stage_dist = self.stage_distributions.get(stage, {})
        
        if not stage_dist:
            # First time at this stage, use default encoding
            return self.default_encode(value)
        
        # Find underrepresented encodings at this stage
        total = sum(stage_dist.values())
        mean_count = total / len(stage_dist) if stage_dist else 1
        
        # Generate encodings biased toward underrepresented regions
        candidate_encodings = []
        for _ in range(100):  # Generate candidates
            enc = self.generate_encoding(value)
            
            # Prefer encodings that are underrepresented
            current_count = stage_dist.get(enc, 0)
            if current_count < mean_count:
                candidate_encodings.append(enc)
                if len(candidate_encodings) >= 10:
                    break
        
        if not candidate_encodings:
            # All encodings overrepresented, generate fresh
            return self.generate_encoding(value)
        
        # Select from candidates
        return random.choice(candidate_encodings)
\end{lstlisting}

\subsection{Multi-Stage Frequency Mitigation}

Apply frequency normalization at multiple stages:

\begin{lstlisting}[language=Python, caption={Multi-stage frequency mitigation}]
class MultiStageFrequencyMitigation:
    """
    Mitigate frequency analysis across entire computation pipeline
    """
    
    def __init__(self, stages: List[str]):
        self.stages = stages
        self.stage_normalizers = {
            stage: AdaptiveFrequencyNormalizer()
            for stage in stages
        }
        self.global_tracker = DistributionTracker()
        
    def process_with_mitigation(self,
                               input_value: Any,
                               computation_graph: Dict) -> bytes:
        """
        Process through computation with frequency mitigation
        at each stage
        """
        current_encoding = None
        
        for stage in self.stages:
            # Get normalizer for this stage
            normalizer = self.stage_normalizers[stage]
            
            if stage == "input":
                # Initial encoding with frequency normalization
                current_encoding = normalizer.adaptive_encode(input_value)
                
            else:
                # Transform encoding through computation
                operation = computation_graph.get(stage, {}).get('operation')
                inputs = computation_graph.get(stage, {}).get('inputs', [])
                
                # Apply operation with frequency-aware encoding
                if operation and current_encoding:
                    # Track distribution before operation
                    self.global_tracker.observe_propagation(
                        f"{stage}_pre", current_encoding
                    )
                    
                    # Perform oblivious operation
                    result = self.oblivious_compute(
                        operation, current_encoding, inputs
                    )
                    
                    # Re-encode result with frequency normalization
                    current_encoding = normalizer.adaptive_encode(result)
                    
                    # Track distribution after operation
                    self.global_tracker.observe_propagation(
                        f"{stage}_post", current_encoding
                    )
        
        return current_encoding
    
    def analyze_frequency_leakage(self) -> Dict[str, float]:
        """
        Analyze potential frequency leakage across stages
        """
        analysis = {}
        
        for stage in self.stages:
            normalizer = self.stage_normalizers[stage]
            uniformity = normalizer.measure_uniformity()
            
            analysis[f"{stage}_uniformity"] = uniformity
            
            # Check if frequency patterns persist across stages
            if stage != self.stages[0]:
                prev_stage = self.stages[self.stages.index(stage) - 1]
                correlation = self.measure_stage_correlation(prev_stage, stage)
                analysis[f"{prev_stage}_to_{stage}_correlation"] = correlation
        
        # Overall system uniformity
        analysis["system_uniformity"] = np.mean([
            v for k, v in analysis.items() if "uniformity" in k
        ])
        
        return analysis
    
    def measure_stage_correlation(self, stage1: str, stage2: str) -> float:
        """
        Measure correlation between distributions at different stages
        Low correlation = good frequency mitigation
        """
        dist1 = self.global_tracker.intermediate_distributions.get(f"{stage1}_post", {})
        dist2 = self.global_tracker.intermediate_distributions.get(f"{stage2}_pre", {})
        
        if not dist1 or not dist2:
            return 0.0
        
        # Convert to frequency vectors
        all_encodings = set(dist1.keys()) | set(dist2.keys())
        
        vec1 = np.array([dist1.get(enc, 0) for enc in all_encodings])
        vec2 = np.array([dist2.get(enc, 0) for enc in all_encodings])
        
        # Normalize
        if vec1.sum() > 0:
            vec1 = vec1 / vec1.sum()
        if vec2.sum() > 0:
            vec2 = vec2 / vec2.sum()
        
        # Calculate correlation coefficient
        if len(vec1) > 1:
            return np.corrcoef(vec1, vec2)[0, 1]
        
        return 0.0
\end{lstlisting}

\subsection{Theoretical Analysis}

\begin{theorem}[Frequency Mitigation Bound]
Given input distribution $P_{\text{in}}$ with maximum frequency $p_{\max}$, the adaptive encoding scheme with $k(x) = \lceil 1/p(x) \rceil$ encodings per value $x$ achieves output distribution $P_{\text{out}}$ with:
$$\text{TV}(P_{\text{out}}, U) \leq \frac{1}{2}\sqrt{\frac{1}{k_{\min}}}$$
where $U$ is uniform distribution and $k_{\min} = \min_x k(x)$.
\end{theorem}

\begin{proof}[Proof Sketch]
Each value $x$ with frequency $p(x)$ gets $k(x) = \lceil 1/p(x) \rceil$ encodings. The expected number of times encoding $e_i^x$ appears is:
$$\mathbb{E}[\#e_i^x] = \frac{p(x) \cdot n}{k(x)} \approx \frac{n}{k(x)^2} \leq n$$
By concentration inequalities and uniformity of selection within encoding pools, the total variation distance from uniform is bounded by the reciprocal square root of the minimum encoding count.
\end{proof}

\subsection{Practical Implementation}

\begin{lstlisting}[language=Python, caption={Complete adaptive frequency mitigation system}]
class CompleteAdaptiveSystem:
    """Production-ready adaptive frequency mitigation"""
    
    def __init__(self, 
                 max_encodings: int = 1000,
                 rebalance_interval: int = 1000,
                 uniformity_target: float = 0.95):
        self.max_encodings = max_encodings
        self.rebalance_interval = rebalance_interval
        self.uniformity_target = uniformity_target
        
        # Components
        self.input_normalizer = AdaptiveFrequencyNormalizer(uniformity_target)
        self.propagation_tracker = PropagationAwareEncoder()
        self.multi_stage = MultiStageFrequencyMitigation([
            "input", "transform", "aggregate", "output"
        ])
        
        # Metrics
        self.operations_count = 0
        self.last_rebalance = 0
        
    def process_query(self, query: Any) -> bytes:
        """Process with full adaptive frequency mitigation"""
        
        self.operations_count += 1
        
        # Adaptive encoding based on observed frequency
        encoded = self.input_normalizer.adaptive_encode(query)
        
        # Track through computation
        result = self.multi_stage.process_with_mitigation(
            query,
            {"transform": {"operation": "oblivious_op"}}
        )
        
        # Periodic rebalancing
        if self.operations_count - self.last_rebalance > self.rebalance_interval:
            self.rebalance_all()
            self.last_rebalance = self.operations_count
        
        # Monitor and adjust
        metrics = self.multi_stage.analyze_frequency_leakage()
        if metrics["system_uniformity"] < self.uniformity_target:
            self.emergency_rebalance()
        
        return result
    
    def rebalance_all(self):
        """Periodic global rebalancing"""
        for stage in self.multi_stage.stages:
            self.multi_stage.stage_normalizers[stage].rebalance_encodings()
    
    def emergency_rebalance(self):
        """Emergency rebalancing when uniformity drops"""
        print(f"Emergency rebalance at operation {self.operations_count}")
        # Flush encoding pools and regenerate
        for normalizer in self.multi_stage.stage_normalizers.values():
            normalizer.encoding_pool.clear()
        self.rebalance_all()
\end{lstlisting}

\subsection{Key Insights}

This adaptive approach provides:

1. **Dynamic adjustment**: Encoding counts adapt to observed frequencies
2. **Propagation tracking**: Monitor how distributions change through computation
3. **Multi-stage mitigation**: Apply normalization at each computation stage
4. **Automatic rebalancing**: Detect and correct distribution skew
5. **Theoretical guarantees**: Bounded distance from uniform distribution

The system continuously learns and adapts, maintaining uniformity even as input distributions change over time.

\section{Implementation References}
\label{sec:implementation-refs}

The concepts presented in this chapter have been implemented in several complementary projects that demonstrate the practical application of oblivious computing principles:

\subsection{Algebraic Cipher Types}

The \texttt{algebraic\_cipher\_types} library provides a complete implementation of cipher types with monadic properties for noise injection:

\begin{lstlisting}[language=C++, caption={Cipher boolean type with false positive/negative rates}]
template <>
class cipher<bool>
{
public:
    using plain_value_type = bool;
    
    auto fpr() const { return bool_->fpr(); }  // False positive rate
    auto fnr() const { return bool_->fnr(); }  // False negative rate
    auto size() const { return bool_->size(); }
    
    // Decode with secret key
    auto try_convert(string_view secret) const {
        return bool_->try_convert(secret);
    }
    
private:
    // Type erasure for different cipher implementations
    struct concept {
        virtual double fnr() const = 0;
        virtual double fpr() const = 0;
        virtual size_t size() const = 0;
        virtual optional<bool> try_convert(string_view) const = 0;
    };
};
\end{lstlisting}

This implementation demonstrates:
\begin{itemize}
\item Type erasure for polymorphic cipher behavior
\item Tracking of error rates (FPR/FNR) through operations
\item Secret-key-based decoding only on trusted client
\end{itemize}

\subsection{Cipher Maps and Oblivious Functions}

The \texttt{cipher\_maps} project implements oblivious maps where:
\begin{itemize}
\item Undefined mappings return random oracle outputs
\item Both domain and codomain can be oblivious values: $f^* : X^* \mapsto Y^*$
\item Mappings are only learned upon request (lazy evaluation)
\end{itemize}

\begin{lstlisting}[language=Python, caption={Oblivious map with random oracle for undefined elements}]
class ObliviousMap:
    """
    Map where undefined elements return random oracle outputs
    Notation: f* = (f, C) where C is computational basis subset
    """
    
    def __getitem__(self, key: bytes) -> bytes:
        if key in self.defined_mappings:
            return self.defined_mappings[key]
        else:
            # Random oracle for undefined elements
            return self.random_oracle(key)
    
    def random_oracle(self, key: bytes) -> bytes:
        """Deterministic but unpredictable mapping"""
        return hashlib.sha256(self.seed + key).digest()
\end{lstlisting}

\subsection{Cipher Trapdoor Sets}

The \texttt{cipher\_trapdoor\_sets} implementation provides:
\begin{itemize}
\item Boolean algebra over cipher sets
\item Trapdoor functions for one-way transformations
\item Preservation of set operations in encrypted domain
\end{itemize}

\subsection{Practical Key-Value Store}

A complete oblivious key-value store demonstrates end-to-end system integration:

\begin{lstlisting}[language=C++, caption={Oblivious KVS with cipher operations}]
// From kvs.cpp - practical oblivious key-value store
class ObliviousKVS {
    // Store encrypted key-value pairs
    map<cipher<string>, cipher<string>> store;
    
    // Oblivious lookup - returns cipher value or noise
    cipher<string> lookup(cipher<string> key) {
        auto it = store.find(key);
        if (it != store.end()) {
            return it->second;
        }
        // Return noise for undefined keys
        return cipher<string>::noise();
    }
    
    // Oblivious AND operation example
    cipher<bool> oblivious_and(
        cipher<bool> a, 
        cipher<bool> b
    ) {
        // Operation on cipher values without decryption
        return cipher_and_operation(a, b);
    }
};
\end{lstlisting}

\subsection{Integration with Bernoulli Types}

These implementations integrate seamlessly with the Bernoulli type framework:

\begin{itemize}
\item \textbf{Error propagation}: Rate spans track uncertainty through compositions
\item \textbf{Space optimization}: Two-level hashing with seed search
\item \textbf{Frequency normalization}: Adaptive encoding based on observed distributions
\item \textbf{Complete obliviousness}: Server processes bytes without semantic understanding
\end{itemize}

\subsection{Available Implementations}

Complete working implementations are available in:
\begin{itemize}
\item \texttt{other/algebraic\_cipher\_types/} - Algebraic cipher type system
\item \texttt{other/cipher\_maps/} - Oblivious map constructions
\item \texttt{other/cipher\_trapdoor\_sets/} - Trapdoor set operations
\item \texttt{other/bernoulli\_data\_type/} - Core Bernoulli type implementations
\item \texttt{other/encrypted\_search\_*} - Various encrypted search systems
\end{itemize}

These implementations provide:
\begin{enumerate}
\item Production-ready C++ header-only libraries
\item Python reference implementations for prototyping
\item Comprehensive test suites verifying statistical properties
\item Benchmarks demonstrating practical performance
\end{enumerate}

The key insight from these implementations: \textit{Complete obliviousness is achievable in practice} with reasonable performance overhead, typically 2-10x compared to plaintext operations, while providing provable privacy guarantees that exact computation cannot achieve.

\section{Deep Dive: Advanced Oblivious Computing Concepts}
\label{sec:deep-dive-advanced}

Our exploration of the additional repositories reveals a comprehensive framework for oblivious computing that extends far beyond simple encrypted search. These projects demonstrate three fundamental pillars of complete oblivious systems:

\subsection{Oblivious Maps: Universal Function Approximators}

The \texttt{cipher\_maps} project introduces the concept of \textit{oblivious maps} as universal Bernoulli approximators:

\begin{definition}[Oblivious Map]
An oblivious map $f^* = (f, \mathcal{C})$ where $\mathcal{C}$ is the computational basis subset, satisfies:
\begin{itemize}
\item The function $f^* : X \mapsto Y$ is a Bernoulli approximation of $f$
\item If $x \in X$ is not in the domain of definition, $f^*(x)$ returns a random oracle output over $Y$
\item A particular mapping $y = f^*(x)$ may only be learned by applying $f^*$ to $x$
\end{itemize}
\end{definition}

\begin{lstlisting}[language=Python, caption={Oblivious map with random oracle behavior}]
class ObliviousMap:
    """
    Implements f* : X* -> Y* where both domain and codomain are oblivious
    Key properties:
    - Undefined elements return deterministic random oracle outputs
    - Mappings are learned lazily upon request
    - Error rates are tracked: error_rate(f_hat, x)
    """
    
    def __init__(self, secret_key: bytes):
        self.secret = secret_key
        self.defined_mappings = {}
        self.random_oracle_seed = hashlib.sha256(secret_key).digest()
    
    def __getitem__(self, encoded_x: bytes) -> bytes:
        """
        Returns encoded output for encoded input
        Undefined inputs get random oracle outputs
        """
        if encoded_x in self.defined_mappings:
            return self.defined_mappings[encoded_x]
        else:
            # Random oracle: deterministic but unpredictable
            return hashlib.sha256(
                self.random_oracle_seed + encoded_x
            ).digest()
    
    def error_rate(self, x: bytes) -> float:
        """
        Returns Pr{f_hat(x) != f(x)} for defined f
        For undefined x, returns 1.0 (always error)
        """
        return 0.0 if x in self.defined_mappings else 1.0
\end{lstlisting}

The theoretical paper proves that the \textit{singular hash map} (SHM) achieves the space-complexity lower bound for oblivious maps with arbitrary-length inputs and outputs, with false positive rate $\varepsilon$ and false negative rate $\eta$.

\subsection{Cipher Trapdoor Sets: Homomorphic Boolean Algebras}

The \texttt{cipher\_trapdoor\_sets} project develops a complete Boolean algebra over cipher sets that preserves operations while obscuring identities:

\begin{theorem}[Cipher Boolean Algebra Homomorphism]
Given the Boolean algebra $A = (\mathcal{P}(\{0,1\}^*), \cup, \cap, \complement, \emptyset, \{0,1\}^*)$ and $B = (\{0,1\}^m, \lor, \land, \neg, 0^m, 1^m)$, the homomorphism $F: A \mapsto B$ defined as:
\begin{equation}
F(\beta) = \begin{cases}
    h(\beta) & \beta \in \{0,1\}^* \\
    \lor & \beta = \cup \\
    \land & \beta = \cap \\
    \neg & \beta = \complement \\
    0^m & \beta = \emptyset \\
    1^m & \beta = \{0,1\}^*
\end{cases}
\end{equation}
preserves Boolean operations while providing one-way trapdoor properties through cryptographic hash function $h$.
\end{theorem}

The paper derives the false positive rate for these constructions:

\begin{theorem}[False Positive Rate]
The false positive rate is:
\begin{equation}
\text{FPR}(m,n) = \alpha^m(n) \text{ where } \alpha(n) = \left(1 - 2^{-(n+1)}\right)
\end{equation}
yielding a bit-rate per element:
\begin{equation}
b(n,\text{FPR}) = \frac{\log_2 \text{FPR}}{n \cdot \alpha(n)}
\end{equation}
\end{theorem}

\begin{lstlisting}[language=C++, caption={k-Disjoint Hash Set implementation}]
template<size_t k>
class DisjointHashSet {
    // k-DHS: bins m elements into m/k bins with k elements per bin
    // Special cases:
    //   k=1: Perfect Hash Filter
    //   k=m: Singular Hash Set (optimal space)
    
public:
    DisjointHashSet(size_t m, double target_fpr) 
        : num_elements(m), 
          bins(m/k), 
          elements_per_bin(k) {
        // Compute optimal parameters
        compute_hash_seeds(target_fpr);
    }
    
    bool contains(const bytes& encoded_element) {
        size_t bin_id = hash_to_bin(encoded_element);
        bytes bin_hash = compute_bin_hash(bin_id, encoded_element);
        
        // Check if hash indicates membership
        return check_membership_encoding(bin_hash);
    }
    
private:
    // Achieves information-theoretic lower bound
    // for space complexity when k=m (Singular Hash Set)
    size_t compute_optimal_space() {
        return -num_elements * log2(target_fpr) / log2(e);
    }
};
\end{lstlisting}

\subsection{Algebraic Cipher Types: Lifting Monoids to Cipher Monoids}

The \texttt{algebraic\_cipher\_types} project provides the mathematical foundation for lifting algebraic structures into the cipher domain:

\begin{definition}[Cipher Functor]
The cipher functor lifts a monoid $(S,*,e)$ to $c_A(S,*,e)$ where:
\begin{itemize}
\item $A \subseteq S$ is the subset of observable elements
\item $s: S \times \mathbb{N} \mapsto c_A S$ maps elements to their $k$-th cipher representation
\item $s': c_A S \mapsto S$ satisfies $s'(s(a,k)) = a$ for all $a \in S, k \in \mathbb{N}$
\item The operation $(c_A *): (c_A S, c_A S) \mapsto c_A S$ preserves associativity
\end{itemize}
\end{definition}

\begin{lstlisting}[language=C++, caption={Type-erased cipher values with monadic noise}]
template <typename T>
class cipher {
public:
    using plain_value_type = T;
    
    // Track error rates through operations
    double fpr() const { return impl_->fpr(); }
    double fnr() const { return impl_->fnr(); }
    
    // Decode only with secret key
    optional<T> try_decode(string_view secret) const {
        return impl_->try_decode(secret);
    }
    
    // Type erasure for polymorphic cipher types
private:
    struct concept {
        virtual double fpr() const = 0;
        virtual double fnr() const = 0;
        virtual optional<T> try_decode(string_view) const = 0;
    };
    
    template <typename CipherImpl>
    struct model : concept {
        // Concrete cipher implementation
        CipherImpl cipher_;
        
        double fpr() const override { 
            return compute_fpr(cipher_); 
        }
        double fnr() const override { 
            return compute_fnr(cipher_); 
        }
    };
    
    shared_ptr<const concept> impl_;
};

// Noisy cipher with level tracking for composition
template <typename T, unsigned int H, unsigned int L>
struct noisy_cipher {
    static constexpr auto secret_hash = H;
    static constexpr auto level = L;
    
    string code;  // Cipher encoding
    double fpr, fnr;  // Error rates
};

// Logical operations preserve level structure
template <unsigned int H, unsigned int L>
noisy_cipher<bool,H,L+1> logical_or(
    const noisy_cipher<bool,H,L>& x,
    const noisy_cipher<bool,H,L>& y
) {
    // Compose at next level with error propagation
    return compose_with_error_tracking(x, y, or_operation);
}
\end{lstlisting}

\subsection{Entropy Maximization for Encrypted Search}

The \texttt{encrypted\_search\_stream\_entropy\_maximization} project provides information-theoretic analysis of confidentiality:

\begin{theorem}[Maximum Entropy Distribution]
For an encrypted search system with $k$ search agents, the maximum entropy distribution has:
\begin{itemize}
\item Agent identities: $A_j \sim \text{Uniform}(k)$
\item Inter-arrival times: $T_j \sim \text{Geometric}(\lambda = 1/2)$
\item Trapdoors per query: $N_j \sim \text{Geometric}(\mu = 2)$
\item Trapdoor values: $Y_j \sim \text{Uniform}(0, 2^m - 1)$
\end{itemize}
This achieves expected compressed bit length:
\begin{equation}
\ell = \frac{1}{\lambda} + p + \mu(1 + m)
\end{equation}
where $p$ is the bits for agent encoding and $m$ is bits per trapdoor.
\end{theorem}

\subsection{Practical Implementation: Oblivious Key-Value Store}

The implementations demonstrate a complete oblivious KVS system:

\begin{lstlisting}[language=C++, caption={Complete oblivious KVS with cipher operations}]
class ObliviousKVS {
    // Store maps cipher keys to cipher values
    map<cipher<string>, cipher<string>> store;
    
    // Oblivious operations
    cipher<string> get(cipher<string> key) {
        auto it = store.find(key);
        if (it != store.end()) {
            return it->second;
        }
        // Return noise for undefined keys
        return cipher<string>::noise();
    }
    
    // Boolean operations on cipher values
    noisy_cipher<bool> contains_all(
        vector<cipher<string>> keys
    ) {
        noisy_cipher<bool> result = 
            noisy_cipher<bool>::true_value();
        
        for (const auto& key : keys) {
            auto exists = contains(key);
            result = logical_and(result, exists);
        }
        
        return result;  // Still encrypted!
    }
    
    // Oblivious function composition
    template<typename F>
    auto apply_oblivious(
        F encoded_function,
        cipher<string> encoded_input
    ) -> decltype(auto) {
        // Server applies function without knowing:
        // - What function it is
        // - What the input represents
        // - What the output means
        return encoded_function(encoded_input);
    }
};
\end{lstlisting}

\subsection{Key Mathematical Results}

\subsubsection{Space-Optimal Constructions}

The singular hash map achieves the information-theoretic lower bound:
\begin{equation}
\text{Space}(n, \varepsilon) = -n \cdot \frac{\log_2 \varepsilon}{1.44} \text{ bits}
\end{equation}

\subsubsection{Error Composition Through Operations}

For composed operations, error rates propagate as:
\begin{equation}
\text{FPR}(f \circ g) \leq \text{FPR}(f) + \text{FPR}(g) - \text{FPR}(f) \cdot \text{FPR}(g)
\end{equation}

\subsubsection{Entropy Bounds}

The entropy of an oblivious system is bounded by:
\begin{equation}
H(\text{Cipher}) \geq H(\text{Plain}) - I(\text{Plain}; \text{Observable})
\end{equation}
where $I$ is the mutual information between plaintext and observable ciphertext.

\subsection{Unified Framework Insights}

These projects together reveal a unified framework where:

\begin{enumerate}
\item \textbf{Universal Approximation}: Oblivious maps can approximate any function with controllable error rates
\item \textbf{Algebraic Preservation}: Boolean and monoid operations are preserved in the cipher domain
\item \textbf{Optimal Space}: Constructions achieve information-theoretic lower bounds
\item \textbf{Composability}: Operations compose with predictable error propagation
\item \textbf{Complete Obliviousness}: Server operates as pure $\text{bytes} \rightarrow \text{bytes}$ transformer
\end{enumerate}

The complete system provides:
\begin{itemize}
\item \textbf{Theoretical guarantees}: Proven optimal space and entropy bounds
\item \textbf{Practical implementations}: Working C++ and Python code
\item \textbf{Flexible abstractions}: Type erasure allows polymorphic cipher types
\item \textbf{Production readiness}: Error tracking, key management, and performance optimization
\end{itemize}

This represents a complete realization of the vision: computation on encrypted data where the server learns nothing beyond input/output sizes.

\section{Chapter Summary}

We've built a \textit{completely} oblivious computing system, progressing from naive search to true computational privacy. The complete framework includes:

\textbf{Foundation Techniques:}
\begin{itemize}
\item \textbf{Bloom filters}: Space efficiency and inherent uncertainty
\item \textbf{Frequency hiding}: Multiple encodings using 1/p(x) principle  
\item \textbf{Correlation breaking}: Tuple encoding for common pairs
\item \textbf{Noise injection}: Fake queries to obscure patterns
\item \textbf{Uniform encoding}: All queries look like random noise
\end{itemize}

\textbf{Advanced Oblivious Enhancements:}
\begin{itemize}
\item \textbf{Oblivious I/O}: Both inputs AND outputs remain encoded—server never sees plaintext
\item \textbf{Oblivious operations}: Operations themselves are encoded—server doesn't know if computing AND, OR, or XOR
\item \textbf{Adaptive frequency normalization}: Dynamic adjustment of encoding counts based on observed distributions
\item \textbf{Two-level perfect hashing}: Efficient constructions with seed search for near-perfect mappings
\item \textbf{Garbled circuits}: Practical implementation of oblivious function evaluation
\end{itemize}

The ultimate result: The server becomes a pure function evaluator: \texttt{bytes} $\rightarrow$ \texttt{bytes} with \textit{no semantic understanding} of what it's computing. This achieves true oblivious computing where the server learns nothing beyond the size of inputs and outputs. This is the full power of the Bernoulli-oblivious approach—by embracing both approximation and complete encoding, we achieve privacy guarantees that exact computation cannot provide.

\section{Further Reading}

\begin{itemize}
\item Song, D. et al. (2000). ``Practical techniques for searches on encrypted data''
\item Curtmola, R. et al. (2006). ``Searchable symmetric encryption: Improved definitions''
\item Cash, D. et al. (2013). ``Highly-scalable searchable symmetric encryption''
\item Pappas, V. et al. (2014). ``Blind seer: A scalable private DBMS''
\item Demertzis, I. et al. (2020). ``SEAL: Attack mitigation for encrypted databases''
\end{itemize}