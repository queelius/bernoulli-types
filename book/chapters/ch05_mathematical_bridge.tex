\chapter{Mathematical Foundations: A Bridge to Volume 2}
\label{ch:mathematical-bridge}

\begin{quote}
\textit{``Mathematics is the language in which God has written the universe.''} — Galileo Galilei

\textit{``And privacy is the dialect we're learning to speak.''} — This chapter
\end{quote}

\section*{Learning Objectives}
By the end of this chapter, you will:
\begin{itemize}
\item Understand the mathematical framework underlying Bernoulli types
\item Learn about error composition and propagation
\item See how probability theory guarantees privacy
\item Preview advanced constructions coming in Volume 2
\item Recognize when mathematical rigor is essential
\end{itemize}

\section{Why Mathematics Matters for Privacy}

You've built a working oblivious system. It seems private. But how can you be \textit{sure}? How do you know an adversary won't find a clever attack? How can you guarantee privacy will hold even as the system scales?

Mathematics provides the answer. It gives us:
\begin{itemize}
\item \textbf{Formal guarantees}: Proofs that privacy holds under specific assumptions
\item \textbf{Quantifiable bounds}: Exact error rates and privacy parameters
\item \textbf{Composition rules}: How privacy degrades with multiple operations
\item \textbf{Optimality results}: When we've achieved the best possible tradeoffs
\end{itemize}

This chapter bridges the practical foundations you've learned to the rigorous mathematical framework developed in Volume 2.

\section{The Probability Foundation}

\subsection{Random Variables and Bernoulli Types}

Every Bernoulli type is fundamentally a random variable:

\begin{definition}[Bernoulli Type as Random Variable]
A Bernoulli type $\mathcal{B}$ is a tuple $(X, \Omega, P, \varepsilon)$ where:
\begin{itemize}
\item $X$ is the true (latent) value
\item $\Omega$ is the space of possible observations
\item $P: X \times \Omega \to [0,1]$ is the observation probability
\item $\varepsilon \in [0,1]$ is the error bound
\end{itemize}
\end{definition}

For example, a Bernoulli Boolean:
\begin{itemize}
\item $X \in \{\text{true}, \text{false}\}$ (actual value)
\item $\Omega = \{\text{true}, \text{false}\}$ (observed value)
\item $P(\text{observed} | X) = \begin{cases}
    1 - \varepsilon & \text{if observed} = X \\
    \varepsilon & \text{if observed} \neq X
\end{cases}$
\end{itemize}

\subsection{Independence and Correlation}

Privacy requires independence between observations:

\begin{theorem}[Independence Preservation]
If observations $O_1, O_2$ are generated from independent Bernoulli types $\mathcal{B}_1, \mathcal{B}_2$, then:
$$P(O_1, O_2 | X_1, X_2) = P(O_1 | X_1) \cdot P(O_2 | X_2)$$
\end{theorem}

This independence is what prevents correlation attacks. When we use tuple encoding for pairs, we deliberately break independence to hide correlations—a controlled violation for privacy.

\section{Error Composition Theory}

\subsection{Basic Composition Rules}

When combining Bernoulli types, errors compose predictably:

\begin{theorem}[Error Composition]
Given Bernoulli types with error rates $\varepsilon_1, \varepsilon_2$:
\begin{align}
\text{Union: } \varepsilon_{A \cup B} &= \varepsilon_A + \varepsilon_B - \varepsilon_A \varepsilon_B \\
\text{Intersection: } \varepsilon_{A \cap B} &= \varepsilon_A \cdot \varepsilon_B \\
\text{Complement: } \varepsilon_{\neg A} &= \varepsilon_A \\
\text{Symmetric Difference: } \varepsilon_{A \triangle B} &= \varepsilon_A + \varepsilon_B - 2\varepsilon_A\varepsilon_B
\end{align}
\end{theorem}

\begin{proof}[Proof for Union]
The union has a false positive if either $A$ or $B$ has a false positive:
\begin{align}
P(\text{false positive in } A \cup B) &= P(\text{FP in } A \text{ or FP in } B) \\
&= P(\text{FP in } A) + P(\text{FP in } B) - P(\text{FP in both}) \\
&= \varepsilon_A + \varepsilon_B - \varepsilon_A \varepsilon_B
\end{align}
\end{proof}

\subsection{Cascading Operations}

For sequential operations, we need tighter bounds:

\begin{theorem}[Cascading Error Bounds]
For $n$ sequential operations with individual error rates $\varepsilon_i$:
$$\varepsilon_{\text{total}} \leq 1 - \prod_{i=1}^{n}(1 - \varepsilon_i) \leq \sum_{i=1}^{n} \varepsilon_i$$
\end{theorem}

The upper bound (sum) is tight when errors are small and independent.

\section{Information-Theoretic Privacy}

\subsection{Entropy and Uniformity}

Perfect privacy requires maximum entropy:

\begin{definition}[Maximum Entropy Principle]
A distribution $P$ over observations $\Omega$ achieves maximum entropy when:
$$H(P) = \log_2 |\Omega|$$
This occurs when $P$ is uniform: $P(\omega) = \frac{1}{|\Omega|}$ for all $\omega \in \Omega$.
\end{definition}

Our Bernoulli encodings achieve this by construction:

\begin{theorem}[Uniformity of Bernoulli Encoding]
If each value $v$ maps to $k$ encodings selected uniformly from space $\mathcal{H}$ with $|\mathcal{H}| = N$, and we select encodings uniformly at random, then as $k \to N/|V|$, the output distribution approaches uniform.
\end{theorem}

\subsection{Differential Privacy Connection}

Our framework relates to differential privacy but with key differences:

\begin{definition}[Differential Privacy]
A mechanism $\mathcal{M}$ is $(\epsilon, \delta)$-differentially private if for adjacent datasets $D, D'$:
$$P[\mathcal{M}(D) \in S] \leq e^\epsilon \cdot P[\mathcal{M}(D') \in S] + \delta$$
\end{definition}

\begin{proposition}[Bernoulli Types vs Differential Privacy]
Bernoulli types provide:
\begin{itemize}
\item \textbf{Computational} indistinguishability (not statistical)
\item \textbf{Bounded} error (not exponential mechanism)
\item \textbf{Composable} guarantees (errors don't explode)
\item \textbf{No privacy budget} (can query indefinitely)
\end{itemize}
\end{proposition}

\section{Optimal Constructions}

\subsection{Space-Privacy-Accuracy Tradeoffs}

There are fundamental limits to what's achievable:

\begin{theorem}[Optimality Bound]
For a system storing $n$ items with query accuracy $1-\varepsilon$ and perfect privacy, the minimum space required is:
$$S \geq n \cdot \frac{\log_2(1/\varepsilon)}{c}$$
where $c$ is a constant depending on the construction (typically $c \approx \ln^2 2$).
\end{theorem}

Bloom filters achieve this bound within a constant factor—they are near-optimal.

\subsection{The Impossibility Triangle}

We can formalize the privacy-functionality-efficiency tradeoff:

\begin{theorem}[Impossibility Result]
No system can simultaneously achieve:
\begin{enumerate}
\item Perfect privacy (zero information leakage)
\item Perfect functionality (exact results)
\item Sublinear space (better than storing everything)
\end{enumerate}
At most two of three are possible.
\end{theorem}

\begin{proof}[Proof Sketch]
Perfect privacy + perfect functionality requires the system to return exact results without revealing queries. This requires storing all possible results, requiring linear space.
\end{proof}

\section{Advanced Constructions (Preview of Volume 2)}

\subsection{Cuckoo Filters with Oblivious Access}

Beyond Bloom filters, cuckoo filters offer:
\begin{itemize}
\item Deletion support
\item Better cache locality
\item Lower space overhead
\end{itemize}

\begin{theorem}[Oblivious Cuckoo Filter]
A cuckoo filter with $b$ buckets and $f$ fingerprint bits achieves:
\begin{itemize}
\item False positive rate: $\varepsilon \approx 8b/2^f$
\item Space per item: $\lceil \log_2(1/\varepsilon) + 3 \rceil$ bits
\item Oblivious access with $O(1)$ memory accesses
\end{itemize}
\end{theorem}

\subsection{Locally Decodable Codes}

For private information retrieval with sublinear communication:

\begin{definition}[Locally Decodable Code]
An $(n, k, d, r)$-LDC encodes $k$-bit messages into $n$-bit codewords such that any bit can be recovered by reading only $r$ bits, even with $d$ errors.
\end{definition}

These enable PIR with $O(n^{1/r})$ communication complexity.

\subsection{Oblivious RAM (ORAM)}

For dynamic data with access pattern hiding:

\begin{theorem}[Path ORAM Bounds]
Path ORAM achieves:
\begin{itemize}
\item Storage: $O(N \log N)$ for $N$ items
\item Access complexity: $O(\log^2 N)$ with high probability
\item Perfect security against adaptive adversaries
\end{itemize}
\end{theorem}

\section{Security Proofs and Reductions}

\subsection{Computational Indistinguishability}

Our security relies on computational bounds:

\begin{definition}[Computational Indistinguishability]
Distributions $D_0, D_1$ are computationally indistinguishable if for all polynomial-time algorithms $A$:
$$\left| \Pr_{x \leftarrow D_0}[A(x) = 1] - \Pr_{x \leftarrow D_1}[A(x) = 1] \right| \leq \text{negl}(\lambda)$$
\end{definition}

\begin{theorem}[Oblivious Search Security]
Under the assumption that SHA-256 is a pseudorandom function, our oblivious search system achieves computational indistinguishability from random.
\end{theorem}

\subsection{Reduction to Cryptographic Primitives}

Security reduces to well-studied problems:

\begin{theorem}[Security Reduction]
If there exists a polynomial-time adversary $\mathcal{A}$ that breaks oblivious search privacy, then there exists a polynomial-time algorithm $\mathcal{B}$ that distinguishes SHA-256 from a random oracle.
\end{theorem}

\section{Practical Considerations}

\subsection{When to Apply Rigorous Analysis}

Use formal mathematical analysis when:
\begin{itemize}
\item \textbf{Security critical}: Financial, medical, government systems
\item \textbf{Novel constructions}: Unproven combinations of techniques
\item \textbf{Adversarial environment}: Sophisticated attackers expected
\item \textbf{Long-term deployment}: System must remain secure for years
\end{itemize}

\subsection{Common Mathematical Pitfalls}

\begin{enumerate}
\item \textbf{Assuming independence}: Operations may be correlated
\item \textbf{Ignoring composition}: Errors compound unexpectedly
\item \textbf{Wrong adversary model}: Underestimating capabilities
\item \textbf{Neglecting side channels}: Timing, power, cache attacks
\item \textbf{Asymptotic vs concrete}: $O(\log n)$ may be large in practice
\end{enumerate}

\section{Exercises}

\begin{enumerate}
\item \textbf{Error Analysis}: Calculate the false positive rate after 10 Boolean AND operations, each with individual error rate 0.001.

\item \textbf{Entropy Calculation}: Prove that uniform distribution maximizes entropy for a fixed alphabet size.

\item \textbf{Composition Bounds}: Derive tight bounds for the error rate of $(A \cup B) \cap (C \cup D)$ given individual error rates.

\item \textbf{Security Proof}: Sketch a proof that frequency hiding with $k$ encodings provides $k$-anonymity.

\item \textbf{Optimality}: Show that no data structure can achieve better than $\log_2(1/\varepsilon)$ bits per item for false positive rate $\varepsilon$.
\end{enumerate}

\section{Chapter Summary}

This chapter bridged practical implementation to mathematical foundations. Key insights:

\begin{itemize}
\item Bernoulli types are formally random variables with bounded error
\item Errors compose predictably through probability theory
\item Maximum entropy (uniformity) provides information-theoretic privacy
\item There are fundamental space-privacy-accuracy tradeoffs
\item Security reduces to well-studied cryptographic assumptions
\end{itemize}

The mathematics isn't just academic—it provides the confidence to deploy these systems in adversarial environments. Volume 2 will develop this framework fully, proving optimality results and exploring advanced constructions.

\section{What's Next: Volume 2 Preview}

Volume 2: Mathematical Framework will cover:

\begin{itemize}
\item \textbf{Chapter 6}: Measure theory and probability spaces
\item \textbf{Chapter 7}: Information theory and entropy
\item \textbf{Chapter 8}: Cryptographic foundations
\item \textbf{Chapter 9}: Optimal space bounds
\item \textbf{Chapter 10}: Advanced error composition
\item \textbf{Chapter 11}: Security proofs and reductions
\item \textbf{Chapter 12}: Connections to other privacy frameworks
\end{itemize}

For now, you have enough mathematical understanding to:
\begin{itemize}
\item Analyze your implementations
\item Understand security guarantees
\item Recognize when deeper theory is needed
\item Read research papers in this area
\end{itemize}

\section{Further Reading}

\begin{itemize}
\item Mitzenmacher, M. \& Upfal, E. (2017). \textit{Probability and Computing}
\item Goldreich, O. (2001). \textit{Foundations of Cryptography}
\item Vadhan, S. (2017). \textit{The Complexity of Differential Privacy}
\item Katz, J. \& Lindell, Y. (2020). \textit{Introduction to Modern Cryptography}
\item Cover, T. \& Thomas, J. (2006). \textit{Elements of Information Theory}
\end{itemize}