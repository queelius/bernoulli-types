\section{Implementation References}
\label{sec:implementation-refs}

The concepts presented in this chapter have been implemented in several complementary projects that demonstrate the practical application of oblivious computing principles:

\subsection{Algebraic Cipher Types}

The \texttt{algebraic\_cipher\_types} library provides a complete implementation of cipher types with monadic properties for noise injection:

\begin{lstlisting}[language=C++, caption={Cipher boolean type with false positive/negative rates}]
template <>
class cipher<bool>
{
public:
    using plain_value_type = bool;
    
    auto fpr() const { return bool_->fpr(); }  // False positive rate
    auto fnr() const { return bool_->fnr(); }  // False negative rate
    auto size() const { return bool_->size(); }
    
    // Decode with secret key
    auto try_convert(string_view secret) const {
        return bool_->try_convert(secret);
    }
    
private:
    // Type erasure for different cipher implementations
    struct concept {
        virtual double fnr() const = 0;
        virtual double fpr() const = 0;
        virtual size_t size() const = 0;
        virtual optional<bool> try_convert(string_view) const = 0;
    };
};
\end{lstlisting}

This implementation demonstrates:
\begin{itemize}
\item Type erasure for polymorphic cipher behavior
\item Tracking of error rates (FPR/FNR) through operations
\item Secret-key-based decoding only on trusted client
\end{itemize}

\subsection{Cipher Maps and Oblivious Functions}

The \texttt{cipher\_maps} project implements oblivious maps where:
\begin{itemize}
\item Undefined mappings return random oracle outputs
\item Both domain and codomain can be oblivious values: $f^* : X^* \mapsto Y^*$
\item Mappings are only learned upon request (lazy evaluation)
\end{itemize}

\begin{lstlisting}[language=Python, caption={Oblivious map with random oracle for undefined elements}]
class ObliviousMap:
    """
    Map where undefined elements return random oracle outputs
    Notation: f* = (f, C) where C is computational basis subset
    """
    
    def __getitem__(self, key: bytes) -> bytes:
        if key in self.defined_mappings:
            return self.defined_mappings[key]
        else:
            # Random oracle for undefined elements
            return self.random_oracle(key)
    
    def random_oracle(self, key: bytes) -> bytes:
        """Deterministic but unpredictable mapping"""
        return hashlib.sha256(self.seed + key).digest()
\end{lstlisting}

\subsection{Cipher Trapdoor Sets}

The \texttt{cipher\_trapdoor\_sets} implementation provides:
\begin{itemize}
\item Boolean algebra over cipher sets
\item Trapdoor functions for one-way transformations
\item Preservation of set operations in encrypted domain
\end{itemize}

\subsection{Practical Key-Value Store}

A complete oblivious key-value store demonstrates end-to-end system integration:

\begin{lstlisting}[language=C++, caption={Oblivious KVS with cipher operations}]
// From kvs.cpp - practical oblivious key-value store
class ObliviousKVS {
    // Store encrypted key-value pairs
    map<cipher<string>, cipher<string>> store;
    
    // Oblivious lookup - returns cipher value or noise
    cipher<string> lookup(cipher<string> key) {
        auto it = store.find(key);
        if (it != store.end()) {
            return it->second;
        }
        // Return noise for undefined keys
        return cipher<string>::noise();
    }
    
    // Oblivious AND operation example
    cipher<bool> oblivious_and(
        cipher<bool> a, 
        cipher<bool> b
    ) {
        // Operation on cipher values without decryption
        return cipher_and_operation(a, b);
    }
};
\end{lstlisting}

\subsection{Integration with Bernoulli Types}

These implementations integrate seamlessly with the Bernoulli type framework:

\begin{itemize}
\item \textbf{Error propagation}: Rate spans track uncertainty through compositions
\item \textbf{Space optimization}: Two-level hashing with seed search
\item \textbf{Frequency normalization}: Adaptive encoding based on observed distributions
\item \textbf{Complete obliviousness}: Server processes bytes without semantic understanding
\end{itemize}

\subsection{Available Implementations}

Complete working implementations are available in:
\begin{itemize}
\item \texttt{other/algebraic\_cipher\_types/} - Algebraic cipher type system
\item \texttt{other/cipher\_maps/} - Oblivious map constructions
\item \texttt{other/cipher\_trapdoor\_sets/} - Trapdoor set operations
\item \texttt{other/bernoulli\_data\_type/} - Core Bernoulli type implementations
\item \texttt{other/encrypted\_search\_*} - Various encrypted search systems
\end{itemize}

These implementations provide:
\begin{enumerate}
\item Production-ready C++ header-only libraries
\item Python reference implementations for prototyping
\item Comprehensive test suites verifying statistical properties
\item Benchmarks demonstrating practical performance
\end{enumerate}

The key insight from these implementations: \textit{Complete obliviousness is achievable in practice} with reasonable performance overhead, typically 2-10x compared to plaintext operations, while providing provable privacy guarantees that exact computation cannot achieve.