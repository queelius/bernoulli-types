\section{Fully Oblivious Operations}
\label{sec:fully-oblivious-ops}

\subsection{The Operation Leakage Problem}

Even with encoded inputs and outputs, if the server knows which operation is being performed, it learns information:

\begin{lstlisting}[language=Python, caption={Problem: Operations are observable}]
def server_process(encoded_a, encoded_b, operation="AND"):  # LEAK: operation visible!
    if operation == "AND":
        return self.and_operation(encoded_a, encoded_b)
    elif operation == "OR":
        return self.or_operation(encoded_a, encoded_b)
    # Server knows logical structure of computation!
\end{lstlisting}

\subsection{Oblivious Operations}

The solution: encode the operations themselves!

\begin{lstlisting}[language=Python, caption={Oblivious boolean operations}]
class ObliviousBoolean:
    """Oblivious representation of boolean values and operations"""
    
    def __init__(self, secret_key: bytes):
        self.secret_key = secret_key
        # Pre-compute operation encodings
        self.operation_encodings = self._generate_op_encodings()
        
    def encode_boolean(self, value: bool) -> bytes:
        """Encode boolean uniformly"""
        # Multiple encodings per value for uniformity
        nonce = random.randint(0, 100)
        data = f"{value}:{nonce}:{self.secret_key.hex()}".encode()
        return hashlib.sha256(data).digest()
    
    def encode_operation(self, op: str) -> bytes:
        """Operations are also encoded!"""
        # Server doesn't know if it's AND, OR, XOR, etc.
        nonce = random.randint(0, 100)
        data = f"{op}:{nonce}:{self.secret_key.hex()}".encode()
        return hashlib.sha256(data).digest()
    
    def oblivious_binary_op(self, 
                           encoded_op: bytes,
                           encoded_a: bytes, 
                           encoded_b: bytes) -> bytes:
        """
        Perform operation without knowing what it is!
        All inputs and outputs are encoded.
        """
        # Server executes without knowing operation type
        # Uses homomorphic-like property of encodings
        
        # Combine all encodings into single computation
        combined = self._combine_for_computation(
            encoded_op, encoded_a, encoded_b
        )
        
        # Result is still encoded
        return self._execute_oblivious(combined)
    
    def _execute_oblivious(self, combined: bytes) -> bytes:
        """
        Execute without revealing operation or values
        Uses pre-computed lookup table
        """
        # Hash to index in pre-computed results table
        index = int.from_bytes(combined[:8], 'big') % len(self.result_table)
        return self.result_table[index]
\end{lstlisting}

\subsection{Oblivious Pairs vs Pairs of Oblivious Values}

Critical distinction for correlation hiding:

\begin{lstlisting}[language=Python, caption={Two approaches to encoding pairs}]
class ObliviousPairEncoding:
    """Different encodings reveal different information"""
    
    def pair_of_oblivious(self, a: bool, b: bool) -> Tuple[bytes, bytes]:
        """
        Encode each value separately
        PROBLEM: Correlations visible through patterns
        """
        encoded_a = self.encode_boolean(a)
        encoded_b = self.encode_boolean(b)
        return (encoded_a, encoded_b)  # Two separate encodings
    
    def oblivious_pair(self, a: bool, b: bool) -> bytes:
        """
        Encode the pair as single unit
        BENEFIT: Correlation hidden!
        """
        # Single encoding for the pair
        pair_value = (a, b)
        data = f"{pair_value}:{self.secret_key.hex()}".encode()
        return hashlib.sha256(data).digest()  # One encoding for both!
    
    def demonstrate_difference(self):
        """Show why oblivious pairs hide correlations"""
        
        # Scenario: Always query (term1=True, term2=True) together
        correleted_queries = [(True, True)] * 100
        
        # Method 1: Pair of oblivious values
        separate_encodings = [
            self.pair_of_oblivious(a, b) 
            for a, b in correleted_queries
        ]
        # PROBLEM: Statistical analysis reveals both values usually True
        
        # Method 2: Oblivious pair
        pair_encodings = [
            self.oblivious_pair(a, b)
            for a, b in correleted_queries
        ]
        # BENEFIT: Just see random noise, no correlation visible!
\end{lstlisting}

\subsection{Universal Oblivious Computation}

The complete framework where everything is oblivious:

\begin{lstlisting}[language=Python, caption={Universal oblivious computation}]
class UniversalObliviousComputer:
    """
    Everything is oblivious:
    - Input values
    - Operations  
    - Intermediate results
    - Output values
    """
    
    def __init__(self, computation_space_size: int = 2**20):
        """
        Pre-compute results for all possible computations
        in the defined space
        """
        self.computation_table = self._build_computation_table(
            computation_space_size
        )
        
    def _build_computation_table(self, size: int) -> Dict[bytes, bytes]:
        """
        Pre-compute all possible results
        Maps: (encoded_op, encoded_inputs) -> encoded_output
        """
        table = {}
        
        # For each possible operation encoding
        for op in ["AND", "OR", "XOR", "NAND", "NOR"]:
            op_encodings = self._generate_encodings(op, count=100)
            
            # For each possible input combination
            for a in [True, False]:
                for b in [True, False]:
                    # Generate input encodings
                    a_encodings = self._generate_encodings(a, count=100)
                    b_encodings = self._generate_encodings(b, count=100)
                    
                    # Compute actual result
                    result = self._compute(op, a, b)
                    result_encoding = self._generate_encoding(result)
                    
                    # Store all encoding combinations
                    for op_enc in op_encodings:
                        for a_enc in a_encodings:
                            for b_enc in b_encodings:
                                key = self._combine_encodings(
                                    op_enc, a_enc, b_enc
                                )
                                table[key] = result_encoding
        
        return table
    
    def oblivious_compute(self,
                         encoded_operation: bytes,
                         encoded_inputs: bytes) -> bytes:
        """
        Server executes this:
        - Doesn't know the operation
        - Doesn't know the inputs
        - Doesn't know the output
        Just a lookup in pre-computed table!
        """
        key = self._combine_encodings(encoded_operation, encoded_inputs)
        
        # Simple table lookup - reveals nothing!
        return self.computation_table.get(
            key,
            self._encode_error()  # Encoded error value
        )
\end{lstlisting}

\subsection{Oblivious Function Composition}

Operations can be composed while maintaining obliviousness:

\begin{lstlisting}[language=Python, caption={Composing oblivious operations}]
class ObliviousFunctionComposition:
    """Compose functions without revealing structure"""
    
    def encode_function(self, func_description: str) -> bytes:
        """Encode entire function as single unit"""
        # E.g., "AND(OR(a,b), NOT(c))" encoded as one blob
        return hashlib.sha256(func_description.encode()).digest()
    
    def oblivious_compose(self, 
                         encoded_funcs: List[bytes],
                         encoded_inputs: bytes) -> bytes:
        """
        Compose multiple functions obliviously
        Server doesn't know composition structure
        """
        # Method 1: Encode composition pattern
        composition_encoding = self._encode_composition_pattern(encoded_funcs)
        
        # Method 2: Use garbled circuits approach
        garbled_circuit = self._create_garbled_circuit(
            composition_encoding,
            encoded_inputs
        )
        
        # Execute without knowing structure
        return self._evaluate_garbled(garbled_circuit)
    
    def oblivious_recursive(self,
                          encoded_func: bytes,
                          encoded_base: bytes,
                          encoded_recursive_case: bytes,
                          depth: int) -> bytes:
        """Even recursion can be oblivious!"""
        
        result = encoded_base
        for _ in range(depth):
            # Apply function obliviously
            result = self.oblivious_compute(
                encoded_func,
                self._combine_encodings(result, encoded_recursive_case)
            )
        
        return result  # Still encoded!
\end{lstlisting}

\subsection{Practical Implementation with Garbled Circuits}

For practical oblivious operations, we can use garbled circuit techniques:

\begin{lstlisting}[language=Python, caption={Garbled circuit approach}]
class GarbledObliviousComputation:
    """Use garbled circuits for oblivious operations"""
    
    def __init__(self):
        self.wire_labels = {}  # Maps wire -> (label_0, label_1)
        self.garbled_gates = {}  # Encrypted truth tables
        
    def garble_operation(self, operation: str) -> Dict[str, bytes]:
        """
        Create garbled circuit for operation
        Server can evaluate without knowing operation type
        """
        # Generate random labels for wires
        input_a_labels = (os.urandom(16), os.urandom(16))  # For 0, 1
        input_b_labels = (os.urandom(16), os.urandom(16))
        output_labels = (os.urandom(16), os.urandom(16))
        
        # Create garbled truth table
        garbled_table = []
        for a_val in [0, 1]:
            for b_val in [0, 1]:
                # Compute actual result
                if operation == "AND":
                    result = a_val & b_val
                elif operation == "OR":
                    result = a_val | b_val
                elif operation == "XOR":
                    result = a_val ^ b_val
                
                # Encrypt output label with input labels
                a_label = input_a_labels[a_val]
                b_label = input_b_labels[b_val]
                out_label = output_labels[result]
                
                # Double encryption
                encrypted = self._encrypt(
                    self._encrypt(out_label, a_label),
                    b_label
                )
                garbled_table.append(encrypted)
        
        # Shuffle table to hide operation
        random.shuffle(garbled_table)
        
        return {
            'table': garbled_table,
            'input_labels': (input_a_labels, input_b_labels),
            'output_labels': output_labels
        }
    
    def evaluate_garbled(self,
                        garbled_circuit: Dict,
                        input_a_label: bytes,
                        input_b_label: bytes) -> bytes:
        """
        Evaluate garbled circuit
        Server doesn't know:
        - What operation this implements
        - What values the labels represent
        - What the output means
        """
        # Try decrypting each table entry
        for encrypted_entry in garbled_circuit['table']:
            try:
                # Attempt double decryption
                output_label = self._decrypt(
                    self._decrypt(encrypted_entry, input_a_label),
                    input_b_label
                )
                return output_label  # Found valid output
            except:
                continue  # Wrong input labels for this entry
        
        raise ValueError("Invalid input labels")
\end{lstlisting}

\subsection{The Complete Oblivious System}

Putting it all together:

\begin{lstlisting}[language=Python, caption={Complete oblivious system}]
class CompletelyObliviousSystem:
    """
    Nothing is visible to the server:
    - Not the data
    - Not the operations
    - Not the structure
    - Not the results
    """
    
    def client_side_prepare(self, 
                          query: str,
                          operation: str,
                          correlation_aware: bool = True) -> bytes:
        """Client prepares oblivious computation"""
        
        if correlation_aware:
            # Encode correlated terms together
            encoded_input = self.encode_oblivious_tuple(
                self.extract_correlated_terms(query)
            )
        else:
            # Encode separately
            encoded_input = self.encode_separate_terms(query)
        
        # Encode the operation
        encoded_op = self.encode_operation(operation)
        
        # Combine into single oblivious computation request
        return self.create_oblivious_request(encoded_op, encoded_input)
    
    def server_side_process(self, oblivious_request: bytes) -> bytes:
        """
        Server processes without knowing anything
        Just a deterministic function: bytes -> bytes
        """
        # Could be:
        # - Table lookup
        # - Garbled circuit evaluation  
        # - Homomorphic computation
        # - ORAM access
        # Server doesn't know which!
        
        return self.oblivious_computation_engine.process(oblivious_request)
    
    def client_side_decode(self, 
                         oblivious_result: bytes,
                         expected_type: str) -> Any:
        """Client decodes result"""
        
        if expected_type == "boolean":
            return self.decode_oblivious_boolean(oblivious_result)
        elif expected_type == "tuple":
            return self.decode_oblivious_tuple(oblivious_result)
        elif expected_type == "list":
            return self.decode_oblivious_list(oblivious_result)
        else:
            return self.decode_generic(oblivious_result)
\end{lstlisting}

\subsection{Key Insight: Complete Obliviousness}

The crucial realization: **everything** can and should be oblivious:

1. **Values**: Encoded uniformly (Bernoulli types)
2. **Operations**: Encoded so server doesn't know AND vs OR vs XOR
3. **Composition**: Structure of computation hidden
4. **Correlations**: Tuple encoding hides relationships
5. **Control flow**: Even branches can be oblivious (compute both, select one)

The server becomes a pure function evaluator: `bytes → bytes` with no semantic understanding of what it's computing. This achieves true oblivious computing where the server learns nothing beyond the size of inputs and outputs.