\chapter*{Preface to Volume 1}
\addcontentsline{toc}{chapter}{Preface}

\section*{The Journey Begins}

In an age where data is the new oil and privacy is increasingly scarce, we face a fundamental challenge: how can we compute on sensitive information without revealing it? This volume introduces a radical approach—embracing imperfection to achieve perfection in privacy.

\section*{Why This Book Exists}

Traditional approaches to privacy-preserving computation force us into impossible corners:
\begin{itemize}
\item Homomorphic encryption provides perfect privacy but at 1000x computational cost
\item Secure multiparty computation requires complex protocols and trusted setups
\item Differential privacy adds noise but struggles with complex queries
\end{itemize}

This book presents a different path: \textbf{probabilistic privacy through Bernoulli types and oblivious computing}. By accepting small, bounded errors (like 0.1\% false positives), we achieve something remarkable—practical systems that reveal nothing about their queries.

\section*{What Makes This Approach Different}

Our framework rests on three key insights:

\begin{enumerate}
\item \textbf{Approximation enables privacy}: Exact computation inherently leaks information. Probabilistic data structures can achieve uniform representations indistinguishable from random noise.

\item \textbf{Composition preserves guarantees}: Unlike differential privacy where errors compound exponentially, our Bernoulli types compose predictably with bounded error rates.

\item \textbf{Practical efficiency matters}: A privacy system unused due to poor performance provides no privacy at all. Our methods achieve microsecond query times.
\end{enumerate}

\section*{Volume 1: Laying the Foundation}

This first volume establishes the conceptual and practical foundations:

\begin{description}
\item[Chapter 1] introduces probabilistic data structures through Bloom filters, showing how approximation saves space
\item[Chapter 2] explores hashing and its role in creating uniform representations
\item[Chapter 3] exposes the privacy problem—what adversaries learn from patterns
\item[Chapter 4] builds your first oblivious system with working code
\item[Chapter 5] bridges to the mathematical framework (preview of Volume 2)
\end{description}

\section*{Who Should Read This Book}

This volume is designed for:
\begin{itemize}
\item \textbf{Graduate students} in computer science, cryptography, or security
\item \textbf{Practitioners} building privacy-preserving systems
\item \textbf{Researchers} exploring new approaches to private computation
\item \textbf{Anyone} concerned about privacy in our digital age
\end{itemize}

\subsection*{Prerequisites}

Readers should have:
\begin{itemize}
\item Undergraduate-level understanding of algorithms and data structures
\item Basic probability theory (expected values, independence)
\item Programming experience (examples use Python)
\item Familiarity with hashing (helpful but not required)
\end{itemize}

Mathematical maturity is more important than specific knowledge—we develop the necessary theory as needed.

\section*{The Road Ahead}

This is Volume 1 of a planned three-volume series:

\begin{description}
\item[Volume 1: Foundations] (This book) Core concepts and first implementation
\item[Volume 2: Theory] Mathematical framework, security proofs, optimal constructions
\item[Volume 3: Systems] Distributed systems, performance engineering, real applications
\end{description}

Each volume stands alone but builds on previous material. Start here to understand the core ideas, then dive deeper based on your interests.

\section*{A Note on Terminology}

We introduce several new terms:
\begin{itemize}
\item \textbf{Bernoulli types}: Probabilistic data structures with bounded error rates
\item \textbf{Oblivious computing}: Computation that reveals nothing about its inputs
\item \textbf{Uniform encoding}: Representations indistinguishable from random noise
\end{itemize}

These aren't arbitrary—each captures a precise technical concept that existing terms don't quite cover.

\section*{How to Use This Book}

\subsection*{For Courses}
\begin{itemize}
\item \textbf{One semester}: Cover all five chapters with programming projects
\item \textbf{Module in privacy course}: Use Chapters 3-4 as standalone unit
\item \textbf{Seminar}: Each chapter can drive 1-2 weeks of discussion
\end{itemize}

\subsection*{For Self-Study}
\begin{itemize}
\item Read chapters in order—each builds on previous concepts
\item Implement the code examples—understanding comes through doing
\item Try the exercises—solutions available at \url{book.bernoulli-types.org}
\end{itemize}

\subsection*{For Practitioners}
\begin{itemize}
\item Start with Chapter 4 for immediate practical value
\item Reference Chapter 3 for threat model understanding
\item Use Chapter 5 to evaluate security guarantees
\end{itemize}

\section*{Code and Resources}

All code from this book is available at:
\begin{center}
\url{https://github.com/bernoulli-types/book-volume1}
\end{center}

The repository includes:
\begin{itemize}
\item Complete implementations
\item Extended examples
\item Test suites
\item Benchmark code
\item Jupyter notebooks
\end{itemize}

\section*{Acknowledgments}

This work builds on decades of research in probabilistic data structures, cryptography, and privacy. Special recognition goes to:

\begin{itemize}
\item Burton Bloom for the filter that started it all (1970)
\item The cryptography community for showing what's possible
\item The systems community for demanding practical solutions
\item Early readers who provided invaluable feedback
\end{itemize}

Special thanks to [specific individuals to be added].

\section*{Join the Conversation}

Privacy-preserving computation is a rapidly evolving field. Join our community:

\begin{itemize}
\item Discussion forum: \url{forum.bernoulli-types.org}
\item Updates and errata: \url{book.bernoulli-types.org}
\item Contact: \url{feedback@bernoulli-types.org}
\end{itemize}

\section*{A Personal Note}

Ten years ago, I faced a simple question: how can a journalist search leaked documents without revealing what they're investigating? Existing solutions were either impractical (homomorphic encryption) or insufficient (basic hashing). This book is the answer I wish I'd had then.

Privacy isn't about hiding wrongdoing—it's about preserving the space for thought, investigation, and growth without fear of observation. In building these systems, we're not just solving technical problems; we're defending fundamental human values.

The journey from exact to approximate, from deterministic to probabilistic, from observable to oblivious, isn't just a technical transformation. It's a new way of thinking about privacy, computation, and the tradeoffs we make.

Welcome to the world of probabilistic privacy. Let's build systems that compute without watching.

\vspace{1cm}
\begin{flushright}
\textit{Alexander Towell}\\
\textit{December 2024}
\end{flushright}