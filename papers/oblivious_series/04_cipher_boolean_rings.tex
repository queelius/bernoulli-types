\documentclass[11pt,final,hidelinks]{article}
\usepackage[utf8]{inputenc}
\usepackage[T1]{fontenc}
\usepackage{lmodern}
\usepackage[margin=1in]{geometry}
\usepackage[english]{babel}
\usepackage{graphicx}
\usepackage[activate={true,nocompatibility},final,tracking=true,kerning=true,spacing=true,factor=1100,stretch=10,shrink=10]{microtype}
\usepackage{mathtools}
\usepackage{amsmath}
\usepackage{amsthm}
\usepackage{amssymb}
\usepackage{algorithm2e}
\usepackage{booktabs}
\usepackage{hyperref}
\usepackage[square,numbers]{natbib}
\bibliographystyle{plainnat}
\usepackage{cleveref}
\usepackage{tikz}
\usetikzlibrary{arrows.meta,positioning,shapes.geometric}

% Include unified notation for oblivious computing
% Unified Notation for Oblivious Computing Series
% Focus: Precisely specifying WHICH parts are oblivious

% ============================================
% Basic Types (needed for Bernoulli references)
% ============================================

\providecommand{\Bool}{\mathbb{B}}
\providecommand{\True}{\mathtt{true}}
\providecommand{\False}{\mathtt{false}}
\providecommand{\Bernoulli}[2]{\mathcal{B}^{#2}(#1)}  % Bernoulli type constructor
\providecommand{\BF}{\mathsf{BF}}  % Bloom Filter notation
\providecommand{\Enc}[1]{\mathsf{Enc}(#1)}  % Encryption notation
\providecommand{\Universe}{\mathcal{U}}  % Universal set
\providecommand{\SI}[2]{\mathsf{SI}(#1, #2)}  % Secure Index
\providecommand{\MutInfo}[2]{I(#1 ; #2)}  % Mutual Information
\providecommand{\Info}[1]{H(#1)}  % Information/Entropy
\providecommand{\CondInfo}[2]{H(#1 | #2)}  % Conditional Entropy
\providecommand{\Prob}[1]{\mathbb{P}\left[#1\right]}  % Probability
\providecommand{\Adv}{\mathcal{A}}  % Adversary
\providecommand{\negl}{\mathsf{negl}}  % Negligible function
\providecommand{\Token}[1]{\mathsf{Token}(#1)}  % Tokenization function
\providecommand{\PRF}{\mathsf{PRF}}  % Pseudo-random function

% ============================================
% Core Oblivious Type Constructors
% ============================================

% Basic oblivious wrapper - makes entire type oblivious
\providecommand{\Obv}[1]{\mathcal{O}\langle #1 \rangle}

% Partial oblivious - only specific components are oblivious
% Examples:
%   (T, \Obv{U}) - second component oblivious
%   \Obv{(T, U)} - entire tuple oblivious (entangled)
%   (\Obv{T}, \Obv{U}) - both components independently oblivious

% ============================================
% Granularity Markers
% ============================================

% Component-level oblivious (independent)
\providecommand{\ObvComp}[1]{\mathcal{O}_c\langle #1 \rangle}

% Structure-level oblivious (entangled)
\providecommand{\ObvStruct}[1]{\mathcal{O}_s\langle #1 \rangle}

% Type-level oblivious (even the type itself is hidden)
\providecommand{\ObvType}[1]{\mathcal{O}_t\langle #1 \rangle}

% ============================================
% Oblivious Product Types
% ============================================

% Independent oblivious components
\providecommand{\ObvProd}[2]{(\Obv{#1}, \Obv{#2})}

% Entangled oblivious pair
\providecommand{\ObvPair}[2]{\Obv{(#1, #2)}}

% Mixed oblivious (first clear, second oblivious)
\providecommand{\MixedProd}[2]{(#1, \Obv{#2})}

% ============================================
% Oblivious Sum Types
% ============================================

% Tag is clear, value is oblivious
\providecommand{\ObvSum}[2]{#1 + \Obv{#2}}

% Both tag and value are oblivious
\providecommand{\ObvTagSum}[2]{\Obv{(#1 + #2)}}

% ============================================
% Oblivious Function Types
% ============================================

% Function with oblivious output
\providecommand{\ObvOut}[2]{#1 \to \Obv{#2}}

% Function with oblivious input
\providecommand{\ObvIn}[2]{\Obv{#1} \to #2}

% Fully oblivious function
\providecommand{\ObvFun}[2]{\Obv{(#1 \to #2)}}

% Oblivious function application
\providecommand{\ObvApp}[2]{\Obv{#1}(#2)}

% ============================================
% Oblivious Collections
% ============================================

% Oblivious set - membership is oblivious
\providecommand{\ObvSet}[1]{\Obv{\mathcal{P}(#1)}}

% Set of oblivious elements
\providecommand{\SetObv}[1]{\mathcal{P}(\Obv{#1})}

% Oblivious map - lookups are oblivious
\providecommand{\ObvMap}[2]{\Obv{(#1 \to #2)}}

% Map with oblivious values
\providecommand{\MapObv}[2]{#1 \to \Obv{#2}}

% ============================================
% Access Patterns and Leakage
% ============================================

% What leaks from accessing x
\providecommand{\Leak}[1]{\mathcal{L}(#1)}

% Access pattern for operation
\providecommand{\Pattern}[1]{\pi(#1)}

% Observable trace
\providecommand{\Trace}[1]{\tau(#1)}

% Hidden value
\providecommand{\Hidden}[1]{\mathsf{h}(#1)}

% Revealed/Observable value
\providecommand{\Reveal}[1]{\tilde{#1}}

% ============================================
% Leakage Specifications
% ============================================

% No leakage
\providecommand{\NoLeak}{\bot}

% Size leakage only
\providecommand{\SizeLeak}{\mathsf{size}}

% Pattern leakage
\providecommand{\PatternLeak}{\mathsf{pattern}}

% Full leakage
\providecommand{\FullLeak}{\top}

% ============================================
% Examples of Notation Usage
% ============================================

% Example 1: Oblivious map returning tuple
% \ObvMap{K}{(T, U)} - entire map is oblivious, returns clear tuple
% K \to \Obv{(T, U)} - clear map, returns oblivious tuple
% K \to (\Obv{T}, \Obv{U}) - clear map, returns tuple with both components oblivious
% \ObvMap{K}{(\Obv{T}, U)} - oblivious map, returns tuple with first component oblivious

% Example 2: Nested oblivious structures
% \Obv{\Obv{T}} - double oblivious (observation of an observation)
% \ObvSet{\ObvPair{T}{U}} - oblivious set of oblivious pairs
% \SetObv{(T, U)} - clear set of oblivious tuples

% ============================================
% Composition Rules
% ============================================

% Leakage composition (sequential)
\providecommand{\LeakSeq}[2]{\Leak{#1} \oplus \Leak{#2}}

% Leakage composition (parallel)
\providecommand{\LeakPar}[2]{\Leak{#1} \parallel \Leak{#2}}

% Leakage bound
\providecommand{\LeakBound}[2]{\Leak{#1} \leq #2}

% ============================================
% Security Definitions
% ============================================

% Indistinguishability
\providecommand{\Indist}{\approx}
\providecommand{\CompIndist}{\approx_c}
\providecommand{\StatIndist}{\approx_s}

% Security parameter
\providecommand{\SecParam}{\lambda}

% Negligible function
\providecommand{\Negl}[1]{\mathsf{negl}(#1)}

% ============================================
% Common Patterns
% ============================================

% Searchable encryption pattern
\providecommand{\SearchEnc}[2]{\ObvMap{#1}{#2}}

% PIR pattern (oblivious index, clear value)
\providecommand{\PIR}[2]{\ObvIn{#1}{#2}}

% ORAM pattern (oblivious address, oblivious value)
\providecommand{\ORAM}[2]{\ObvFun{#1}{#2}}

% ============================================
% Notation Guide Box
% ============================================

\providecommand{\ObliviousNotationGuide}{%
\begin{center}
\fbox{
\begin{minipage}{0.9\textwidth}
\textbf{Oblivious Type Notation Guide}\\[0.5em]
\begin{tabular}{ll}
\textbf{Notation} & \textbf{Meaning} \\
\hline
$\Obv{T}$ & Type $T$ is oblivious \\
$(T, \Obv{U})$ & Tuple with second component oblivious \\
$\Obv{(T, U)}$ & Entire tuple is oblivious (entangled) \\
$(\Obv{T}, \Obv{U})$ & Both components independently oblivious \\
$\ObvMap{K}{V}$ & Oblivious map (lookups don't leak) \\
$K \to \Obv{V}$ & Clear map with oblivious values \\
$\Leak{x} = \NoLeak$ & Operation on $x$ leaks nothing \\
$\Leak{x} = \SizeLeak$ & Operation on $x$ leaks size only \\
\end{tabular}
\end{minipage}
}
\end{center}
}


% Additional notation for Boolean rings
\newcommand{\XOR}{\oplus}
\newcommand{\AND}{\land}
\newcommand{\HashRing}[1]{\mathcal{H}(#1)}
\newcommand{\SymDiff}{\triangle}
\newcommand{\Cat}{\parallel}
\newcommand{\BitString}{\{0,1\}^m}
\newcommand{\PowerSet}[1]{\mathcal{P}(#1)}
\newcommand{\id}{\mathsf{id}}

% Theorem environments
\newtheorem{theorem}{Theorem}[section]
\newtheorem{lemma}[theorem]{Lemma}
\newtheorem{proposition}[theorem]{Proposition}
\newtheorem{corollary}[theorem]{Corollary}
\newtheorem{definition}[theorem]{Definition}
\newtheorem{example}[theorem]{Example}
\newtheorem{remark}[theorem]{Remark}
\newtheorem{construction}[theorem]{Construction}

\title{Cipher Boolean Rings: XOR-Based Oblivious Sets for Identity Testing}
\author{
    Alexander Towell\\
    \texttt{atowell@siue.edu}
}
\date{\today}

\begin{document}
\maketitle

\begin{abstract}
We present a fundamentally different approach to oblivious sets based on Boolean rings and XOR operations, departing from the hash-based Bloom filter paradigm. Instead of testing membership through multiple hash functions setting bits in an array, we represent sets as single hash values computed by XORing the hashes of all elements. This construction supports only equality testing—determining if two sets are identical—not membership queries. The result is indistinguishable from random noise, providing strong privacy properties. We formalize this as a homomorphism from the Boolean ring of sets with symmetric difference to the Boolean ring of bit strings with XOR. This approach trades the flexibility of membership testing for perfect hiding of set structure, space optimality (one hash per set regardless of size), and elegant algebraic properties. We show this is particularly suited for scenarios requiring set equality verification rather than membership testing, such as detecting duplicate datasets, verifying synchronized states, or comparing encrypted collections.
\end{abstract}

\ObliviousNotationGuide

\section{Introduction}

\subsection{A Radical Departure from Bloom Filters}

Traditional approaches to oblivious sets (Bloom filters, cuckoo filters) focus on membership testing: "Is element $x$ in set $S$?" We present an orthogonal approach that answers a different question: "Are sets $S_1$ and $S_2$ equal?"

\begin{example}[The XOR Construction]
Given set $S = \{x, y, z\}$:
\begin{itemize}
    \item Bloom filter: Set bits at positions $h_1(x), h_2(x), \ldots$ for each element
    \item XOR approach: Compute single value $h(x) \XOR h(y) \XOR h(z)$
\end{itemize}
The XOR result:
\begin{itemize}
    \item Looks completely random
    \item Same size regardless of $|S|$
    \item Cannot extract individual elements
    \item Can only test equality with another XOR-hash
\end{itemize}
\end{example}

\subsection{Why XOR?}

The XOR operation has unique properties that make it ideal for oblivious sets:

\begin{enumerate}
    \item \textbf{Commutativity}: $a \XOR b = b \XOR a$ (order doesn't matter)
    \item \textbf{Associativity}: $(a \XOR b) \XOR c = a \XOR (b \XOR c)$ (grouping doesn't matter)
    \item \textbf{Self-inverse}: $a \XOR a = 0$ (elements cancel themselves)
    \item \textbf{Identity}: $a \XOR 0 = a$ (empty set is identity)
\end{enumerate}

These properties mean sets with the same elements produce the same hash regardless of insertion order or duplicates.

\subsection{The Trade-off}

\begin{center}
\begin{tabular}{lll}
\toprule
\textbf{Property} & \textbf{Bloom Filter} & \textbf{XOR Hash} \\
\midrule
Membership test & Yes & No \\
Equality test & Indirect & Direct \\
Space per set & $O(n)$ & $O(1)$ \\
Reveals size & Approximately & No \\
Set operations & Union, intersection & Symmetric difference \\
Information hiding & Partial & Complete \\
\bottomrule
\end{tabular}
\end{center}

\section{Mathematical Framework}

\subsection{Boolean Rings}

\begin{definition}[Boolean Ring]
A Boolean ring $(R, +, \cdot, 0, 1)$ satisfies:
\begin{itemize}
    \item $(R, +, 0)$ is an abelian group
    \item $(R, \cdot, 1)$ is a monoid
    \item Multiplication distributes over addition
    \item $a \cdot a = a$ for all $a \in R$ (idempotent)
\end{itemize}
\end{definition}

\begin{example}[Set Boolean Ring]
The power set forms a Boolean ring:
\begin{equation}
(\PowerSet{U}, \SymDiff, \cap, \overline{}, \emptyset)
\end{equation}
where $\SymDiff$ is symmetric difference: $A \SymDiff B = (A \setminus B) \cup (B \setminus A)$.
\end{example}

\begin{example}[Bit String Boolean Ring]
Bit strings form a Boolean ring:
\begin{equation}
(\BitString, \XOR, \AND, \id, 0^m)
\end{equation}
where operations are bitwise.
\end{example}

\subsection{The XOR Homomorphism}

\begin{definition}[XOR Set Hash]
Define $G: \PowerSet{U} \to \BitString$ by:
\begin{equation}
G(S) = \bigoplus_{x \in S} h(x)
\end{equation}
where $h: U \to \BitString$ is a hash function and $\bigoplus$ is XOR.
\end{definition}

\begin{theorem}[Homomorphism Properties]
$G$ is a homomorphism preserving:
\begin{align}
G(A \SymDiff B) &= G(A) \XOR G(B)\\
G(A \cap B) &= G(A) \AND G(B) \text{ (only for special hash families)}\\
G(\emptyset) &= 0^m
\end{align}
\end{theorem}

\begin{proof}
For symmetric difference:
\begin{align}
G(A \SymDiff B) &= \bigoplus_{x \in A \SymDiff B} h(x)\\
&= \bigoplus_{x \in A \setminus B} h(x) \XOR \bigoplus_{x \in B \setminus A} h(x)\\
&= G(A) \XOR G(B)
\end{align}
The self-inverse property of XOR handles the cancellation of common elements.
\end{proof}

\subsection{Properties of XOR Hashing}

\begin{theorem}[Uniqueness]
For random oracle $h$:
\begin{equation}
\Prob{G(A) = G(B) | A \neq B} = 2^{-m}
\end{equation}
where $m$ is the hash output size in bits.
\end{theorem}

\begin{proof}
Since $h$ is a random oracle, $G(A)$ and $G(B)$ are independent uniform random strings when $A \neq B$. The probability they are equal is $2^{-m}$.
\end{proof}

\begin{corollary}[Birthday Paradox]
With $n$ different sets, expect a collision when $n \approx 2^{m/2}$.
\end{corollary}

\section{Oblivious Properties}

\subsection{Information Hiding}

\begin{theorem}[Computational Indistinguishability]
For polynomial-time adversary $\Adv$:
\begin{equation}
\left|\Prob{\Adv(G(S)) = 1} - \Prob{\Adv(r) = 1}\right| \leq \negl(\lambda)
\end{equation}
where $r \leftarrow \BitString$ is random and $\lambda$ is security parameter.
\end{theorem}

This means the XOR hash of a set is indistinguishable from random noise.

\begin{theorem}[No Information Leakage]
Given $G(S)$, the following are information-theoretically hidden:
\begin{itemize}
    \item The size $|S|$
    \item The elements of $S$
    \item Any proper subset of $S$
    \item The structure of $S$
\end{itemize}
\end{theorem}

\begin{proof}
For any set $S$ with $|S| = n$, there are exactly $2^m$ possible XOR-hash values. Each appears for exactly $2^{|U|-m}$ different sets (when $|U| > m$). Thus $G(S)$ reveals nothing about $S$ beyond ruling out sets with different hashes.
\end{proof}

\subsection{Oblivious Equality Testing}

\begin{construction}[Oblivious Set Equality]
To test if hidden sets $\Obv{S_1}$ and $\Obv{S_2}$ are equal:
\begin{enumerate}
    \item Compute $g_1 = G(S_1)$, $g_2 = G(S_2)$
    \item Return $\Obv{(g_1 == g_2)}$
\end{enumerate}
False positive rate: $2^{-m}$.
\end{construction}

\section{Algebraic Operations}

\subsection{Symmetric Difference}

\begin{theorem}[Efficient Symmetric Difference]
\begin{equation}
G(A \SymDiff B) = G(A) \XOR G(B)
\end{equation}
Computed in $O(1)$ time regardless of set sizes.
\end{theorem}

\begin{example}[Set Updates]
To add element $x$ to set $S$:
\begin{equation}
G(S \cup \{x\}) = \begin{cases}
G(S) \XOR h(x) & \text{if } x \notin S\\
G(S) & \text{if } x \in S
\end{cases}
\end{equation}
But we cannot determine which case applies!
\end{example}

\subsection{Multiset Handling}

\begin{remark}[Duplicate Elimination]
XOR automatically handles duplicates:
\begin{equation}
G(\{x, x, y\}) = h(x) \XOR h(x) \XOR h(y) = 0 \XOR h(y) = h(y) = G(\{y\})
\end{equation}
This can be a feature or a bug depending on the application.
\end{remark}

\subsection{Set Algebra Limitations}

\begin{theorem}[Cannot Compute Union/Intersection]
Given only $G(A)$ and $G(B)$, cannot compute:
\begin{itemize}
    \item $G(A \cup B)$ (would need $G(A \cap B)$)
    \item $G(A \cap B)$ (would need individual elements)
    \item $G(A \setminus B)$ (would need membership info)
\end{itemize}
\end{theorem}

\section{Applications}

\subsection{Distributed Set Synchronization}

\begin{example}[Efficient Sync Detection]
Nodes maintaining replicated sets:
\begin{enumerate}
    \item Each node maintains XOR-hash of its set
    \item Periodically exchange hashes ($m$ bits)
    \item If hashes differ, sets are out of sync
    \item If hashes match, sets are equal (with high probability)
\end{enumerate}
Bandwidth: $O(m)$ instead of $O(n)$ for $n$ elements.
\end{example}

\subsection{Deduplication Detection}

\begin{example}[Dataset Deduplication]
\begin{verbatim}
struct Dataset {
    XORHash content_hash;  // G(all_records)
    size_t record_count;   // For additional validation
};

bool are_duplicates(Dataset d1, Dataset d2) {
    return d1.content_hash == d2.content_hash;
}
\end{verbatim}
Identifies duplicate datasets without comparing records.
\end{example}

\subsection{Privacy-Preserving Set Comparison}

\begin{construction}[Private Equality Test]
Alice and Bob test if $S_A = S_B$:
\begin{enumerate}
    \item Alice sends $G(S_A) \XOR r$ for random $r$
    \item Bob sends $G(S_B) \XOR s$ for random $s$  
    \item Use secure computation to check equality
    \item Neither party learns the other's set
\end{enumerate}
\end{construction}

\subsection{Blockchain State Verification}

\begin{example}[Merkle Tree Alternative]
Instead of Merkle tree for state:
\begin{itemize}
    \item State = set of key-value pairs
    \item State hash = XOR of hash(key||value)
    \item Updates: XOR old and new hashes
    \item Constant-time updates regardless of state size
\end{itemize}
Trade-off: Cannot prove individual elements, only entire state equality.
\end{example}

\section{Variants and Extensions}

\subsection{Weighted XOR Hashing}

\begin{construction}[Element Multiplicity]
Track element counts using multiple hashes:
\begin{equation}
G_k(S) = \bigoplus_{x \in S} h_k(x)^{c(x)}
\end{equation}
where $c(x)$ is count of $x$ and $h_k$ are independent hashes.
\end{construction}

\subsection{Hybrid Constructions}

\begin{construction}[XOR + Bloom Hybrid]
Maintain both:
\begin{itemize}
    \item XOR hash for equality testing
    \item Bloom filter for membership testing
    \item Space: $O(1) + O(n)$
    \item Best of both worlds at higher cost
\end{itemize}
\end{construction}

\subsection{Incremental XOR Hashing}

\begin{construction}[Sliding Window Sets]
For time-series data:
\begin{enumerate}
    \item Maintain XOR hash of current window
    \item Add new: $G' = G \XOR h(x_{new})$
    \item Remove old: $G' = G \XOR h(x_{old})$
    \item Constant time window updates
\end{enumerate}
\end{construction}

\section{Security Analysis}

\subsection{Collision Resistance}

\begin{theorem}[Birthday Attack]
Finding two different sets with same XOR-hash requires:
\begin{equation}
O(2^{m/2}) \text{ hash evaluations}
\end{equation}
\end{theorem}

\begin{corollary}[Security Parameter]
For $\lambda$-bit security, need $m \geq 2\lambda$.
\end{corollary}

\subsection{Preimage Resistance}

\begin{theorem}[Cannot Invert XOR Hash]
Given $g = G(S)$, finding any $S'$ with $G(S') = g$ requires:
\begin{equation}
O(2^m) \text{ operations (brute force)}
\end{equation}
assuming $h$ is a random oracle.
\end{theorem}

\subsection{Algebraic Attacks}

\begin{remark}[Linear Structure]
The linearity of XOR enables certain attacks:
\begin{itemize}
    \item Given $G(A)$, $G(B)$, can compute $G(A \SymDiff B)$
    \item Cannot determine $A$, $B$, or $A \SymDiff B$ individually
    \item Linear algebra over $GF(2)$ applies
\end{itemize}
\end{remark}

\section{Performance Analysis}

\subsection{Time Complexity}

\begin{center}
\begin{tabular}{lcc}
\toprule
\textbf{Operation} & \textbf{XOR Hash} & \textbf{Bloom Filter} \\
\midrule
Insert element & $O(1)$ & $O(k)$ \\
Delete element & $O(1)$ & Not supported \\
Test membership & Not supported & $O(k)$ \\
Test equality & $O(1)$ & $O(m)$ \\
Symmetric difference & $O(1)$ & Not direct \\
\bottomrule
\end{tabular}
\end{center}

\subsection{Space Complexity}

\begin{theorem}[Space Optimality]
XOR hashing achieves:
\begin{itemize}
    \item $m$ bits per set (constant)
    \item Independent of set size
    \item Optimal for equality testing
    \item Cannot be improved information-theoretically
\end{itemize}
\end{theorem}

\section{Limitations}

\subsection{Fundamental Constraints}

\begin{enumerate}
    \item \textbf{No membership testing}: Cannot check if $x \in S$
    \item \textbf{No size information}: Cannot determine $|S|$
    \item \textbf{No subset relations}: Cannot check $S_1 \subseteq S_2$
    \item \textbf{No element extraction}: Cannot recover any elements
    \item \textbf{Multiset blindness}: Cannot distinguish $\{x\}$ from $\{x, x\}$
\end{enumerate}

\subsection{When Not to Use XOR Hashing}

\begin{itemize}
    \item Need membership queries
    \item Need to enumerate elements
    \item Need to support range queries
    \item Need to preserve element multiplicity
    \item Need proof of inclusion/exclusion
\end{itemize}

\section{Theoretical Connections}

\subsection{Relationship to Linear Sketching}

XOR hashing is a linear sketch over $GF(2)$:
\begin{equation}
G(S) = \sum_{x \in S} h(x) \mod 2
\end{equation}
This connects to:
\begin{itemize}
    \item Compressed sensing
    \item Linear error-correcting codes
    \item Syndrome decoding
\end{itemize}

\subsection{Category Theory Perspective}

\begin{theorem}[Functorial Properties]
XOR hashing defines a functor:
\begin{itemize}
    \item From: Category of sets with symmetric difference
    \item To: Category of bit strings with XOR
    \item Preserves: Identity and composition
\end{itemize}
\end{theorem}

\section{Conclusions}

XOR-based Boolean rings provide a radically different approach to oblivious sets:

\textbf{Key Insights:}
\begin{itemize}
    \item \textbf{Different question}: Equality testing vs. membership testing
    \item \textbf{Perfect hiding}: Result indistinguishable from random
    \item \textbf{Space optimal}: Constant size regardless of set size
    \item \textbf{Algebraic elegance}: Clean homomorphism properties
\end{itemize}

\textbf{The Trade-off:}
\begin{itemize}
    \item Gain: Perfect obliviousness, space efficiency, symmetric difference
    \item Lose: Membership testing, size information, traditional set operations
\end{itemize}

\textbf{Design Philosophy:}
This represents a different point in the design space:
\begin{itemize}
    \item Bloom filters: "Test membership approximately"
    \item XOR hashing: "Test equality exactly (up to collision probability)"
\end{itemize}

Both approaches produce oblivious types, but answer fundamentally different questions. The choice depends on the application's core requirement: Do you need to know if an element is in a set, or if two sets are the same?

\bibliography{references}

\end{document}