% Unified Notation for Bernoulli Types Research
% =============================================
%
% This file provides consistent notation across all papers based on the
% latent/observed framework that is central to Bernoulli types.
%
% Core Principle: Distinguish between latent (true) and observed (noisy) values

% ========== GENERAL NOTATION ==========

% Latent values (no special decoration - these are the "true" values)
% - x, y, z for elements
% - S, T, U for sets  
% - f, g, h for functions

% Observed values (use tilde to indicate observation/approximation)
% - \obs{x}, \obs{y}, \obs{z} for observed elements
% - \obs{S}, \obs{T}, \obs{U} for observed sets
% - \obs{f}, \obs{g}, \obs{h} for observed functions

\newcommand{\obs}[1]{\widetilde{#1}}  % Universal observation operator

% Alternative: Use \latent and \observed for clarity in definitions
\newcommand{\latent}[1]{#1}
\newcommand{\observed}[1]{\widetilde{#1}}

% ========== BERNOULLI TYPES ==========

% Bernoulli type constructor: B_T^{(k)} following the refined framework
\newcommand{\BType}[1]{B_{#1}}  % Basic Bernoulli type B_T
\newcommand{\BTypeOrder}[2]{B_{#1}^{(#2)}}  % Bernoulli type with order B_T^{(k)}

% Specific Bernoulli types
\newcommand{\BBool}{B_{\mathrm{bool}}}  % Bernoulli Boolean
\newcommand{\BBoolOrder}[1]{B_{\mathrm{bool}}^{(#1)}}  % Bernoulli Boolean with order
\newcommand{\BSet}[1]{B_{#1 \mapsto \mathrm{bool}}}  % Bernoulli set (set indicator function)
\newcommand{\BMap}[2]{B_{#1 \mapsto #2}}  % Bernoulli map from type 1 to type 2

% Latent value notation - B_T(x) represents observation of latent x
\newcommand{\BValue}[2]{B_{#1}(#2)}  % B_T(x) - Bernoulli observation of latent x
\newcommand{\BValueOrder}[3]{B_{#1}^{(#2)}(#3)}  % B_T^{(k)}(x) - with explicit order

% ========== ERROR RATES ==========

% General error terminology (for any type)
\newcommand{\errorrate}{\epsilon}  % Generic error rate
\newcommand{\missrate}{\delta}     % Miss rate (fail to detect when present)
\newcommand{\spuriousrate}{\alpha} % Spurious rate (detect when absent)
\newcommand{\confusionrate}{\gamma} % Confusion between different values

% Boolean-specific terminology (only use for Boolean/binary contexts)
\newcommand{\fprate}{\alpha}  % False positive rate (Boolean only!)
\newcommand{\fnrate}{\beta}   % False negative rate (Boolean only!)
\newcommand{\tprate}{\tau}    % True positive rate (Boolean only!)
\newcommand{\tnrate}{\rho}    % True negative rate (Boolean only!)
\newcommand{\FPR}{\mathsf{FPR}}  % False positive rate (text form)
\newcommand{\FNR}{\mathsf{FNR}}  % False negative rate (text form)

% General approximation error for type T
\newcommand{\ApproxError}[2]{\epsilon_{#1 \to #2}}  % Error from type T1 to T2
\newcommand{\TypeConfusion}[3]{\gamma_{#1}(#2 \to #3)}  % Confusion in type T: value1 -> value2

% Error rate notation for specific structures
\newcommand{\withError}[3]{#1_{#2,#3}}  % e.g., \withError{S}{\spuriousrate}{\missrate}

% ========== ERROR MODEL ORDERS ==========

% Order-2 error model (uniform error rates)
\newcommand{\OrderTwo}{\text{Order-2}}
\newcommand{\UniformError}[2]{\epsilon_{#1}, \delta_{#2}}  % Uniform rates

% Order-|U| error model (element-specific error rates)
\newcommand{\OrderU}{\text{Order-}|\Universe|}
\newcommand{\ElementError}[1]{\text{error}(#1)}  % Element-specific error function
\newcommand{\ElementSpuriousRate}[1]{\spuriousrate_{#1}}   % Element-specific spurious rate
\newcommand{\ElementMissRate}[1]{\missrate_{#1}}           % Element-specific miss rate
% Boolean-specific (only use when appropriate)
\newcommand{\ElementFPRate}[1]{\fprate_{#1}}   % Element-specific false positive (Boolean only!)
\newcommand{\ElementFNRate}[1]{\fnrate_{#1}}   % Element-specific false negative (Boolean only!)

% ========== PROBABILITY AND EXPECTATION ==========

\newcommand{\Prob}[1]{\mathbb{P}\left[#1\right]}
\newcommand{\ProbCond}[2]{\mathbb{P}\left[#1 \mid #2\right]}  % Conditional probability
\newcommand{\Expect}[1]{\mathbb{E}\left[#1\right]}
\newcommand{\Var}[1]{\mathrm{Var}\left[#1\right]}

% ========== INFORMATION THEORY ==========

\newcommand{\Entropy}[1]{H(#1)}  % General entropy H(X)
\newcommand{\ConditionalEntropy}[2]{H(#1 \mid #2)}  % Conditional entropy H(X|Y)
\newcommand{\MatrixEntropy}[1]{H(#1)}  % Matrix entropy H(Q)
\newcommand{\MutualInfo}[2]{I(#1; #2)}  % Mutual information I(X;Y)
\newcommand{\KLDiv}[2]{D_{\mathrm{KL}}(#1 \parallel #2)}  % KL divergence

% ========== SET NOTATION ==========

% Basic sets
\newcommand{\Universe}{\mathcal{U}}  % Universal set
\newcommand{\PowerSet}[1]{\mathcal{P}(#1)}  % Power set
\newcommand{\EmptySet}{\emptyset}

% Set operations (no special commands needed, use standard LaTeX)
% \cup for union
% \cap for intersection  
% \setminus for difference
% \triangle for symmetric difference

% Set operations
\newcommand{\SetUnion}{\cup}
\newcommand{\SetIntersection}{\cap}
\newcommand{\SetComplement}[1]{\overline{#1}}
\newcommand{\Complement}[1]{\overline{#1}}
\newcommand{\PS}[1]{\mathcal{P}(#1)}  % Alternate power set notation

% Cardinality
\newcommand{\Card}[1]{\lvert#1\rvert}

% Membership indicator
\newcommand{\Indicator}[1]{\mathbf{1}_{#1}}

% ========== TYPE NOTATION ==========

\newcommand{\Type}[1]{\mathtt{#1}}

% ========== BOOLEAN VALUES ==========

\newcommand{\Bool}{\mathbb{B}}
\newcommand{\True}{\mathtt{true}}
\newcommand{\False}{\mathtt{false}}

% ========== CHANNEL/CONFUSION MATRIX ==========

% Channel notation: observed | latent
\newcommand{\Channel}[2]{\Prob{\text{observe } #1 \mid \text{latent } #2}}

% Confusion matrix entry
\newcommand{\Confusion}[2]{q_{#1,#2}}  % q_ij = P(observe j | latent i)

% ========== FUNCTION NOTATION ==========

% Function composition (use standard \circ)
% Approximate/observed function application
\newcommand{\ApproxApply}[2]{\obs{#1}(#2)}  % \ApproxApply{f}{x}

% ========== USAGE EXAMPLES ==========

% Example 1: Bernoulli Boolean
% Latent: x \in \Bool
% Observed: \obs{x} \in \Bernoulli{\Bool}{2}
% Relationship: \Channel{\obs{x} = \True}{x = \False} = \fprate

% Example 2: Bernoulli Set  
% Latent: S \subseteq U
% Observed: \obs{S} observes S with rates (\fprate, \fnrate)
% Query: \Channel{x \in \obs{S}}{x \notin S} = \fprate

% Example 3: Composition
% Latent: f \circ g
% Observed: \obs{f} \circ \obs{g}
% Error: \errorrate_{\obs{f} \circ \obs{g}} = \errorrate_f + \errorrate_g - \errorrate_f \errorrate_g

% ========== DEPRECATED NOTATION ==========
% The following should be replaced:
% \ASet{S} → \obs{S}
% \PASet{S} → \obs{S}^+ or \withError{S}{\fprate}{0}  
% \NASet{S} → \obs{S}^- or \withError{S}{0}{\fnrate}
% \Set{S} → S (no decoration needed for latent)

% ========== NOTATION TABLE (for inclusion) ==========

\newcommand{\BernoulliNotationTable}{%
\begin{tabular}{@{}ll@{}}
\textbf{Symbol} & \textbf{Meaning} \\
\midrule
$\obs{x}$ & Observed value corresponding to latent $x$ \\
$\obs{S}$ & Observed (approximate) set for latent $S$ \\
$\obs{f}$ & Observed map approximating latent $f$ \\
$\spuriousrate$ & Spurious rate: $\Prob\{\obs{x}\in\obs{S} \mid x\notin S\}$ \\
$\missrate$ & Miss rate: $\Prob\{\obs{x}\notin\obs{S} \mid x\in S\}$ \\
$\confusionrate$ & Confusion rate between different values \\
$\ElementError{x}$ & Element-specific error function for $x$ \\
$\ElementSpuriousRate{x}$ & Spurious rate specific to element $x$ \\
$\ElementMissRate{x}$ & Miss rate specific to element $x$ \\
$\errorrate$ & Generic error rate $\Prob\{\obs{f}(x)\neq f(x)\}$ \\
$\ApproxError{T_1}{T_2}$ & Approximation error from type $T_1$ to $T_2$ \\
$\TypeConfusion{T}{v_1}{v_2}$ & Confusion in type $T$: value $v_1 \to v_2$ \\
$\fprate, \fnrate$ & False pos./neg. rates (Boolean contexts only) \\
$Q$ & Confusion matrix: $Q_{ij}=\Prob\{\text{obs } t_j\mid \text{lat } t_i\}$ \\
$W(y\mid x)$ & Channel kernel from latent $x$ to observed $y$ \\
$\Entropy{Q}$ & Matrix entropy of confusion matrix $Q$ \\
$\ConditionalEntropy{\text{lat}}{\text{obs}}$ & Conditional entropy of latent given observed \\
$\MutualInfo{X}{Y}$ & Mutual information between $X$ and $Y$ \\
$\Indicator{S}$ & Set indicator function $U\to\Bool$ \\
$\Complement{S}$ & Set complement \\
$\Card{S}$ & Cardinality of a set $S$ \\
\end{tabular}%
}

\newcommand{\NotationSection}{%
\section*{Notation}
This section summarizes symbols used throughout the paper. See also the shared cheat sheet for formulas and assumptions.

\medskip
\noindent\BernoulliNotationTable
}
