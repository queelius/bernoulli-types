\documentclass[11pt,final,hidelinks]{article}
\usepackage[utf8]{inputenc}
\usepackage[T1]{fontenc}
\usepackage{lmodern}
\usepackage[margin=1in]{geometry}
\usepackage[english]{babel}
\usepackage{graphicx}
\usepackage[activate={true,nocompatibility},final,tracking=true,kerning=true,spacing=true,factor=1100,stretch=10,shrink=10]{microtype}
\usepackage{mathtools}
\usepackage{amsmath}
\usepackage{amsthm}
\usepackage{amssymb}
% \usepackage{minted} % Disabled - using verbatim instead
\usepackage{booktabs}
\usepackage{hyperref}
\usepackage[square,numbers]{natbib}
\bibliographystyle{plainnat}
\usepackage{cleveref}

% Include unified notation definitions shared across the Bernoulli series
% Unified Notation for Bernoulli Types Research
% =============================================
%
% This file provides consistent notation across all papers based on the
% latent/observed framework that is central to Bernoulli types.
%
% Core Principle: Distinguish between latent (true) and observed (noisy) values

% ========== GENERAL NOTATION ==========

% Latent values (no special decoration - these are the "true" values)
% - x, y, z for elements
% - S, T, U for sets  
% - f, g, h for functions

% Observed values (use tilde to indicate observation/approximation)
% - \obs{x}, \obs{y}, \obs{z} for observed elements
% - \obs{S}, \obs{T}, \obs{U} for observed sets
% - \obs{f}, \obs{g}, \obs{h} for observed functions

\newcommand{\obs}[1]{\widetilde{#1}}  % Universal observation operator

% Alternative: Use \latent and \observed for clarity in definitions
\newcommand{\latent}[1]{#1}
\newcommand{\observed}[1]{\widetilde{#1}}

% ========== BERNOULLI TYPES ==========

% Bernoulli type constructor: B_T^{(k)} following the refined framework
\newcommand{\BType}[1]{B_{#1}}  % Basic Bernoulli type B_T
\newcommand{\BTypeOrder}[2]{B_{#1}^{(#2)}}  % Bernoulli type with order B_T^{(k)}

% Specific Bernoulli types
\newcommand{\BBool}{B_{\mathrm{bool}}}  % Bernoulli Boolean
\newcommand{\BBoolOrder}[1]{B_{\mathrm{bool}}^{(#1)}}  % Bernoulli Boolean with order
\newcommand{\BSet}[1]{B_{#1 \mapsto \mathrm{bool}}}  % Bernoulli set (set indicator function)
\newcommand{\BMap}[2]{B_{#1 \mapsto #2}}  % Bernoulli map from type 1 to type 2

% Latent value notation - B_T(x) represents observation of latent x
\newcommand{\BValue}[2]{B_{#1}(#2)}  % B_T(x) - Bernoulli observation of latent x
\newcommand{\BValueOrder}[3]{B_{#1}^{(#2)}(#3)}  % B_T^{(k)}(x) - with explicit order

% ========== ERROR RATES ==========

% General error terminology (for any type)
\newcommand{\errorrate}{\epsilon}  % Generic error rate
\newcommand{\missrate}{\delta}     % Miss rate (fail to detect when present)
\newcommand{\spuriousrate}{\alpha} % Spurious rate (detect when absent)
\newcommand{\confusionrate}{\gamma} % Confusion between different values

% Boolean-specific terminology (only use for Boolean/binary contexts)
\newcommand{\fprate}{\alpha}  % False positive rate (Boolean only!)
\newcommand{\fnrate}{\beta}   % False negative rate (Boolean only!)
\newcommand{\tprate}{\tau}    % True positive rate (Boolean only!)
\newcommand{\tnrate}{\rho}    % True negative rate (Boolean only!)
\newcommand{\FPR}{\mathsf{FPR}}  % False positive rate (text form)
\newcommand{\FNR}{\mathsf{FNR}}  % False negative rate (text form)

% General approximation error for type T
\newcommand{\ApproxError}[2]{\epsilon_{#1 \to #2}}  % Error from type T1 to T2
\newcommand{\TypeConfusion}[3]{\gamma_{#1}(#2 \to #3)}  % Confusion in type T: value1 -> value2

% Error rate notation for specific structures
\newcommand{\withError}[3]{#1_{#2,#3}}  % e.g., \withError{S}{\spuriousrate}{\missrate}

% ========== ERROR MODEL ORDERS ==========

% Order-2 error model (uniform error rates)
\newcommand{\OrderTwo}{\text{Order-2}}
\newcommand{\UniformError}[2]{\epsilon_{#1}, \delta_{#2}}  % Uniform rates

% Order-|U| error model (element-specific error rates)
\newcommand{\OrderU}{\text{Order-}|\Universe|}
\newcommand{\ElementError}[1]{\text{error}(#1)}  % Element-specific error function
\newcommand{\ElementSpuriousRate}[1]{\spuriousrate_{#1}}   % Element-specific spurious rate
\newcommand{\ElementMissRate}[1]{\missrate_{#1}}           % Element-specific miss rate
% Boolean-specific (only use when appropriate)
\newcommand{\ElementFPRate}[1]{\fprate_{#1}}   % Element-specific false positive (Boolean only!)
\newcommand{\ElementFNRate}[1]{\fnrate_{#1}}   % Element-specific false negative (Boolean only!)

% ========== PROBABILITY AND EXPECTATION ==========

\newcommand{\Prob}[1]{\mathbb{P}\left[#1\right]}
\newcommand{\ProbCond}[2]{\mathbb{P}\left[#1 \mid #2\right]}  % Conditional probability
\newcommand{\Expect}[1]{\mathbb{E}\left[#1\right]}
\newcommand{\Var}[1]{\mathrm{Var}\left[#1\right]}

% ========== INFORMATION THEORY ==========

\newcommand{\Entropy}[1]{H(#1)}  % General entropy H(X)
\newcommand{\ConditionalEntropy}[2]{H(#1 \mid #2)}  % Conditional entropy H(X|Y)
\newcommand{\MatrixEntropy}[1]{H(#1)}  % Matrix entropy H(Q)
\newcommand{\MutualInfo}[2]{I(#1; #2)}  % Mutual information I(X;Y)
\newcommand{\KLDiv}[2]{D_{\mathrm{KL}}(#1 \parallel #2)}  % KL divergence

% ========== SET NOTATION ==========

% Basic sets
\newcommand{\Universe}{\mathcal{U}}  % Universal set
\newcommand{\PowerSet}[1]{\mathcal{P}(#1)}  % Power set
\newcommand{\EmptySet}{\emptyset}

% Set operations (no special commands needed, use standard LaTeX)
% \cup for union
% \cap for intersection  
% \setminus for difference
% \triangle for symmetric difference

% Set operations
\newcommand{\SetUnion}{\cup}
\newcommand{\SetIntersection}{\cap}
\newcommand{\SetComplement}[1]{\overline{#1}}
\newcommand{\Complement}[1]{\overline{#1}}
\newcommand{\PS}[1]{\mathcal{P}(#1)}  % Alternate power set notation

% Cardinality
\newcommand{\Card}[1]{\lvert#1\rvert}

% Membership indicator
\newcommand{\Indicator}[1]{\mathbf{1}_{#1}}

% ========== TYPE NOTATION ==========

\newcommand{\Type}[1]{\mathtt{#1}}

% ========== BOOLEAN VALUES ==========

\newcommand{\Bool}{\mathbb{B}}
\newcommand{\True}{\mathtt{true}}
\newcommand{\False}{\mathtt{false}}

% ========== CHANNEL/CONFUSION MATRIX ==========

% Channel notation: observed | latent
\newcommand{\Channel}[2]{\Prob{\text{observe } #1 \mid \text{latent } #2}}

% Confusion matrix entry
\newcommand{\Confusion}[2]{q_{#1,#2}}  % q_ij = P(observe j | latent i)

% ========== FUNCTION NOTATION ==========

% Function composition (use standard \circ)
% Approximate/observed function application
\newcommand{\ApproxApply}[2]{\obs{#1}(#2)}  % \ApproxApply{f}{x}

% ========== USAGE EXAMPLES ==========

% Example 1: Bernoulli Boolean
% Latent: x \in \Bool
% Observed: \obs{x} \in \Bernoulli{\Bool}{2}
% Relationship: \Channel{\obs{x} = \True}{x = \False} = \fprate

% Example 2: Bernoulli Set  
% Latent: S \subseteq U
% Observed: \obs{S} observes S with rates (\fprate, \fnrate)
% Query: \Channel{x \in \obs{S}}{x \notin S} = \fprate

% Example 3: Composition
% Latent: f \circ g
% Observed: \obs{f} \circ \obs{g}
% Error: \errorrate_{\obs{f} \circ \obs{g}} = \errorrate_f + \errorrate_g - \errorrate_f \errorrate_g

% ========== DEPRECATED NOTATION ==========
% The following should be replaced:
% \ASet{S} → \obs{S}
% \PASet{S} → \obs{S}^+ or \withError{S}{\fprate}{0}  
% \NASet{S} → \obs{S}^- or \withError{S}{0}{\fnrate}
% \Set{S} → S (no decoration needed for latent)

% ========== NOTATION TABLE (for inclusion) ==========

\newcommand{\BernoulliNotationTable}{%
\begin{tabular}{@{}ll@{}}
\textbf{Symbol} & \textbf{Meaning} \\
\midrule
$\obs{x}$ & Observed value corresponding to latent $x$ \\
$\obs{S}$ & Observed (approximate) set for latent $S$ \\
$\obs{f}$ & Observed map approximating latent $f$ \\
$\spuriousrate$ & Spurious rate: $\Prob\{\obs{x}\in\obs{S} \mid x\notin S\}$ \\
$\missrate$ & Miss rate: $\Prob\{\obs{x}\notin\obs{S} \mid x\in S\}$ \\
$\confusionrate$ & Confusion rate between different values \\
$\ElementError{x}$ & Element-specific error function for $x$ \\
$\ElementSpuriousRate{x}$ & Spurious rate specific to element $x$ \\
$\ElementMissRate{x}$ & Miss rate specific to element $x$ \\
$\errorrate$ & Generic error rate $\Prob\{\obs{f}(x)\neq f(x)\}$ \\
$\ApproxError{T_1}{T_2}$ & Approximation error from type $T_1$ to $T_2$ \\
$\TypeConfusion{T}{v_1}{v_2}$ & Confusion in type $T$: value $v_1 \to v_2$ \\
$\fprate, \fnrate$ & False pos./neg. rates (Boolean contexts only) \\
$Q$ & Confusion matrix: $Q_{ij}=\Prob\{\text{obs } t_j\mid \text{lat } t_i\}$ \\
$W(y\mid x)$ & Channel kernel from latent $x$ to observed $y$ \\
$\Entropy{Q}$ & Matrix entropy of confusion matrix $Q$ \\
$\ConditionalEntropy{\text{lat}}{\text{obs}}$ & Conditional entropy of latent given observed \\
$\MutualInfo{X}{Y}$ & Mutual information between $X$ and $Y$ \\
$\Indicator{S}$ & Set indicator function $U\to\Bool$ \\
$\Complement{S}$ & Set complement \\
$\Card{S}$ & Cardinality of a set $S$ \\
\end{tabular}%
}

\newcommand{\NotationSection}{%
\section*{Notation}
This section summarizes symbols used throughout the paper. See also the shared cheat sheet for formulas and assumptions.

\medskip
\noindent\BernoulliNotationTable
}

\usepackage{tikz}
\usetikzlibrary{arrows.meta,positioning,shapes}

% Theorem environments
\newtheorem{theorem}{Theorem}[section]
\newtheorem{lemma}[theorem]{Lemma}
\newtheorem{proposition}[theorem]{Proposition}
\newtheorem{corollary}[theorem]{Corollary}
\newtheorem{definition}[theorem]{Definition}
\newtheorem{example}[theorem]{Example}
\newtheorem{remark}[theorem]{Remark}

% Unified notation - defined in unified_notation.tex

% Type and code notation
\newcommand{\bernoulli}[2]{\mathcal{B}\langle #1, #2 \rangle}
% Probability commands - defined in unified_notation.tex
\newcommand{\equals}{==}
\newcommand{\notequals}{!=}
\newcommand{\assign}{=}
\newcommand{\code}[1]{\texttt{#1}}

\title{Regular Types in the Bernoulli Model: When Equality Isn't Equal}
\author{
    Alexander Towell\\
    \texttt{atowell@siue.edu}
}
\date{\today}

\begin{document}
\maketitle
\NotationSection

\begin{abstract}
We explore the fundamental tension between regular types—which assume perfect observation of equality—and Bernoulli types, where we can only observe latent equality through a noisy channel. Regular types require that equality be reflexive, symmetric, and transitive, but these axioms assume we can perfectly observe whether two values are equal. In reality, equality comparison is itself a computation that observes the latent mathematical fact of equality. Bernoulli types make this explicit: when we write \code{a == b}, we are not accessing the latent truth but rather observing it through potentially faulty comparison operations. This shift from assuming perfect equality to observing approximate equality enables new programming paradigms for distributed systems (where different nodes may observe different equalities), privacy-preserving computation (where equality is deliberately obscured), and fault-tolerant algorithms (where hardware errors affect observations). We present a type system that distinguishes latent equality from observed equality, enabling safe integration of Bernoulli types with traditional programming.
\end{abstract}

\section{Introduction}

\paragraph{Scope and organization.}  Positioned as Part~4 of our Bernoulli series, this paper revisits the concept of regular types under the latent/observed paradigm.  Building on the foundations (Part~1: confusion matrices and Bayesian inference), the set algebra (Part~2: error propagation laws), and the function framework (Part~3: composition of observed functions), we examine how equality and other regular type properties must be reinterpreted when observations are imperfect.  Parts~5--7 explore applications to search, implementation techniques, and statistical analysis.

Alexander Stepanov's concept of regular types forms the foundation of generic programming \cite{stepanov2014}. A regular type supports:
\begin{itemize}
    \item \textbf{Copy construction}: \code{T a(b);} creates \code{a} as a copy of \code{b}
    \item \textbf{Assignment}: \code{a = b;} makes \code{a} equal to \code{b}
    \item \textbf{Equality}: \code{a == b} returns \code{true} iff \code{a} and \code{b} are equal
    \item \textbf{Destruction}: Objects can be destroyed
\end{itemize}

These operations must satisfy semantic constraints. Most critically, equality must be:
\begin{itemize}
    \item \textbf{Reflexive}: \code{a == a}
    \item \textbf{Symmetric}: \code{a == b} implies \code{b == a}
    \item \textbf{Transitive}: \code{a == b} and \code{b == c} implies \code{a == c}
\end{itemize}

Bernoulli types reveal that these constraints assume perfect observation of equality. In reality:
\begin{itemize}
    \item \textbf{Latent}: Two values are either mathematically equal or not
    \item \textbf{Observed}: We can only observe this equality through computation
\end{itemize}

When \code{operator==} returns \code{bernoulli<bool>}, it acknowledges this fundamental gap:
\begin{verbatim}
bernoulli<int> a(42);
assert(a == a);  // Observing latent equality; observation might be incorrect!
\end{verbatim}

This paper argues that recognizing the latent/observed distinction in equality is not a deficiency but a more honest model of computation.

\section{Background: Regular Types}

\subsection{The Regular Type Concept}

Stepanov defines regular types as those that behave like built-in types:

\begin{definition}[Regular Type]
A type \code{T} is regular if it supports:
\begin{enumerate}
    \item Default construction: \code{T a;}
    \item Copy construction: \code{T a(b);}
    \item Move construction: \code{T a(std::move(b));}
    \item Copy assignment: \code{a = b;}
    \item Move assignment: \code{a = std::move(b);}
    \item Equality: \code{a == b;}
    \item Inequality: \code{a != b;}
\end{enumerate}
with natural semantics.
\end{definition}

\subsection{Why Regularity Matters}

Regular types enable generic programming:

\begin{verbatim}
template<typename T>
void sort(std::vector<T>& v) {
    // Requires T to be regular
    // Uses copy, assignment, and comparison
}
\end{verbatim}

The STL assumes regularity for types stored in containers. This assumption enables efficient algorithms and predictable behavior.

\subsection{The Equality Problem}

Equality is fundamental to regularity. It enables:
\begin{itemize}
    \item Container lookups: \code{std::find}, \code{std::map}
    \item Algorithm correctness: \code{std::unique}, \code{std::set}
    \item Reasoning about code: Equational reasoning
\end{itemize}

But what if equality itself is uncertain?

\section{Bernoulli Types and Regularity}

\subsection{Probabilistic Equality}

For Bernoulli types, equality returns a Bernoulli boolean:

\begin{verbatim}
template<typename T>
class bernoulli {
    T value;
    double error_rate;
public:
    bernoulli<bool> operator==(const bernoulli<T>& other) const {
        // Returns true with probability related to actual equality
        return sample_bernoulli_bool(compute_equality_probability(value, other.value));
    }
};
\end{verbatim}

\subsection{The Latent/Observed Gap in Equality Axioms}

\subsubsection{Reflexivity: Observing Self-Equality}

Latent reflexivity always holds: a value is mathematically equal to itself. But observation introduces error:

\begin{verbatim}
bernoulli<int> x(42, 0.1);  // 10% error rate
auto result = (x == x);      // Observing latent self-equality
\end{verbatim}

The gap between latent and observed arises from:
\begin{itemize}
    \item \textbf{Measurement uncertainty}: Physical sensors observing the same value twice
    \item \textbf{Transmission errors}: Network corrupting equality checks
    \item \textbf{Privacy mechanisms}: Deliberately noisy observations for security
    \item \textbf{Hardware faults}: Bit flips during comparison
\end{itemize}

Mathematically:
\begin{equation}
\ProbCond{\obs{(x = x)} = \True}{\text{latent } x = x} = 1 - \epsilon
\end{equation}

\subsubsection{Symmetry: Independent Observations}

Latent symmetry holds: if $a = b$ then $b = a$. However, each observation is independent:

\begin{equation}
\ProbCond{\obs{(a = b)}}{\text{latent } a = b} = \ProbCond{\obs{(b = a)}}{\text{latent } b = a}
\end{equation}

But individual observations may differ:
\begin{verbatim}
if (a == b) {      // First observation of latent equality
    assert(b == a);  // Second independent observation - might differ!
}
\end{verbatim}

This reflects reality: asking "is A equal to B?" twice may yield different answers due to measurement noise, timing, or observer differences.

\subsubsection{Transitivity: Compounding Observation Errors}

Transitivity suffers most severely. Even when latent transitivity holds ($a = b \land b = c \implies a = c$), observations compound errors:

\begin{verbatim}
bernoulli<int> a(1), b(1), c(1);  // Latently all equal
if (a == b && b == c) {            // Two successful observations
    assert(a == c);  // Third independent observation - often fails!
}
\end{verbatim}

For error rate $\epsilon$, the probability of observing transitivity when it latently holds:
\begin{equation}
\ProbCond{\obs{(a = c)} \mid \obs{(a = b)} \land \obs{(b = c)} \land \text{latent } a = b = c} = 1 - \epsilon
\end{equation}

The key insight: we're not observing "transitivity" but rather three independent observations of latent equalities.

\subsection{Cascading Effects}

The probabilistic nature of equality affects all derived operations:

\begin{itemize}
    \item \textbf{Containers}: \code{std::set<bernoulli<T>>} may contain "duplicates"
    \item \textbf{Algorithms}: \code{std::sort} may not produce a total order
    \item \textbf{Invariants}: Class invariants become probabilistic
\end{itemize}

\section{Compositional vs Direct Approximation in Regular Types}

A fundamental design choice emerges when applying Bernoulli approximation to composite types: should we approximate each component independently (compositional) or approximate the composite type as a unit (direct)?

\subsection{Two Approaches to Composite Type Approximation}

\begin{definition}[Compositional Approximation]
For a product type $A \times B$, \emph{compositional approximation} creates:
$$\text{Pair of Bernoulli values}: (\BValue{A}{a}, \BValue{B}{b})$$
where each component is independently approximated.
\end{definition}

\begin{definition}[Direct Approximation]
For a product type $A \times B$, \emph{direct approximation} creates:
$$\text{Bernoulli pair}: \BValue{A \times B}{(a,b)}$$
where the pair itself is approximated as a single unit.
\end{definition}

\begin{theorem}[Structural Differences]
Compositional and direct approximation exhibit fundamentally different structural properties:

\textbf{Compositional}: 
\begin{itemize}
    \item Confusion matrix is Kronecker product of component matrices
    \item Independent error structure between components
    \item Higher-order due to independent parameters
\end{itemize}

\textbf{Direct}:
\begin{itemize}
    \item Single unified confusion matrix over product space
    \item Can exhibit correlation between component errors
    \item May have different error patterns than component-wise errors
\end{itemize}
\end{theorem}

\begin{example}[Regular Type with Compositional Approximation]
For a struct with multiple fields:
\begin{verbatim}
struct Person {
    bernoulli<string> name;  // Independent approximation
    bernoulli<int> age;      // Independent approximation
};

// Equality comparison requires both field comparisons to succeed
bernoull<bool> operator==(const Person& p1, const Person& p2) {
    return (p1.name == p2.name) && (p1.age == p2.age);
    // Error rate: 1 - (1-ε_name)(1-ε_age)
}
\end{verbatim}
\end{example}

\begin{example}[Regular Type with Direct Approximation]
Alternatively, approximate the entire struct:
\begin{verbatim}
struct LatentPerson {
    string name;
    int age;
};

using BernoulliPerson = bernoulli<LatentPerson>;

// Single approximation error for entire object
BernoulliPerson person = approximate_person(latent_person);
// Error structure may correlate name and age errors
\end{verbatim}
\end{example}

\subsection{Design Implications for Regular Types}

\begin{theorem}[Regularity Under Compositional Approximation]
Under compositional approximation, regular type axioms become:
\begin{itemize}
    \item \textbf{Reflexivity}: $\Prob{\obs{a} == \obs{a}} = \prod_{i} (1-\epsilon_i)$ where $\epsilon_i$ are component error rates
    \item \textbf{Transitivity**: Requires three independent observations, each with compound error rate
\end{itemize}
\end{theorem}

\begin{theorem}[Regularity Under Direct Approximation]
Under direct approximation, regular type axioms have different structure:
\begin{itemize}
    \item \textbf{Error correlation}: Components errors can be correlated in ways impossible with compositional approach
    \item \textbf{Lower effective error rates}: May achieve better overall approximation quality
    \item \textbf{Implementation complexity**: Requires algorithm-specific design for each composite type
\end{itemize}
\end{theorem}

\subsection{Practical Guidance}

\begin{remark}[When to Use Each Approach]
\textbf{Compositional approximation} is preferred when:
\begin{itemize}
    \item Components are logically independent
    \item Implementing separate approximations is straightforward
    \item Error budgets can be allocated per-component
\end{itemize}

\textbf{Direct approximation} is preferred when:
\begin{itemize}
    \item Components have natural correlation structure
    \item A single approximation algorithm can handle the entire type
    \item Lower overall error rates are critical
\end{itemize}
\end{remark}

\section{New Programming Paradigms

\subsection{Probabilistic Contracts}

Traditional contracts assume deterministic predicates:
\begin{verbatim}
void insert(std::set<T>& s, const T& value) {
    // Postcondition: s.count(value) == 1
}
\end{verbatim}

With Bernoulli types, contracts become probabilistic:
\begin{verbatim}
void insert(bernoulli_set<T>& s, const T& value) {
    // Postcondition: P[value in s] >= 1 - epsilon
}
\end{verbatim}

\subsection{Eventually-Consistent Data Structures}

Bernoulli types naturally model eventually-consistent systems:

\begin{example}[Distributed Set]
In a distributed system with network partitions:
\begin{verbatim}
class distributed_set {
    bernoulli_set<T> local_view;
    
    bernoulli<bool> contains(const T& value) {
        // Returns probabilistic result based on:
        // - Local information
        // - Network reliability
        // - Synchronization lag
    }
};
\end{verbatim}
\end{example}

\subsection{Byzantine Fault Tolerance}

Bernoulli types can model Byzantine failures:

\begin{verbatim}
template<typename T>
class byzantine_value {
    std::vector<bernoulli<T>> replicas;
    
    bernoulli<T> read() {
        // Majority voting with probabilistic agreement
        return majority_vote(replicas);
    }
};
\end{verbatim}

\section{Practical Considerations}

\subsection{When to Use Bernoulli Types}

Bernoulli types are appropriate when:
\begin{itemize}
    \item \textbf{Uncertainty is inherent}: Sensor data, network communication
    \item \textbf{Approximation is acceptable}: Large-scale analytics, caching
    \item \textbf{Privacy is required}: Differential privacy, secure computation
    \item \textbf{Fault tolerance is critical}: Distributed systems, blockchain
\end{itemize}

\subsection{Mixing Deterministic and Probabilistic Code}

Safe integration requires careful boundaries:

\begin{verbatim}
// Convert Bernoulli to deterministic with threshold
template<typename T>
std::optional<T> to_deterministic(const bernoulli<T>& b, double confidence) {
    if (b.confidence() >= confidence) {
        return b.expected_value();
    }
    return std::nullopt;
}

// Lift deterministic to Bernoulli
template<typename T>
bernoulli<T> to_bernoulli(const T& value) {
    return bernoulli<T>(value, 0.0);  // Zero error rate
}
\end{verbatim}

\subsection{Type System Extensions}

We propose type system extensions for safety:

\begin{verbatim}
// Type-level error bounds
template<typename T, double MaxError>
class bounded_bernoulli {
    static_assert(MaxError >= 0.0 && MaxError <= 1.0);
    // ...
};

// Concepts for Bernoulli-aware algorithms
template<typename T>
concept BernoulliComparable = requires(T a, T b) {
    { a == b } -> std::convertible_to<bernoulli<bool>>;
};
\end{verbatim}

\section{Case Studies}

\subsection{Distributed Consensus}

Traditional consensus algorithms assume reliable equality checks. With Bernoulli types, we model realistic networks:

\begin{example}[Probabilistic Paxos]
\begin{verbatim}
class probabilistic_paxos {
    struct proposal {
        int ballot;
        bernoulli<value_type> value;
    };
    
    bernoulli<bool> is_majority(const std::vector<vote>& votes) {
        // Account for probabilistic equality in vote counting
        int agrees = 0, total = 0;
        for (const auto& v : votes) {
            auto match = (v.value == proposed_value);
            agrees += match.probability_true();
            total++;
        }
        return bernoulli<bool>(agrees > total/2.0);
    }
};
\end{verbatim}
\end{example}

\subsection{Privacy-Preserving Equality}

Differential privacy requires adding noise to queries:

\begin{example}[Private Database]
\begin{verbatim}
class private_database {
    double epsilon;  // Privacy parameter
    
    bernoulli<bool> equals(const record& a, const record& b) {
        bool actual = (a.id == b.id);
        // Add Laplace noise for differential privacy
        double noise = laplace_noise(1.0 / epsilon);
        double probability = sigmoid(actual ? 1.0 + noise : noise);
        return bernoulli<bool>::from_probability(probability);
    }
};
\end{verbatim}
\end{example}

\subsection{Approximate Deduplication}

Large-scale systems often need approximate deduplication:

\begin{example}[Content-Addressable Storage]
\begin{verbatim}
class approximate_cas {
    struct hash_result {
        bernoulli<hash_type> hash;
        double collision_probability;
    };
    
    void store(const data& content) {
        auto h = hash(content);
        if (probably_unique(h)) {
            backing_store[h.expected()] = content;
        } else {
            // Handle probable duplicate
            merge_or_create_new(h, content);
        }
    }
};
\end{verbatim}
\end{example}

\section{Implementation Strategies}

\subsection{Efficient Bernoulli Booleans}

\begin{verbatim}
class bernoulli_bool {
    bool value;
    float confidence;  // Stored as fixed-point for efficiency
    
public:
    bernoulli_bool operator&&(const bernoulli_bool& other) const {
        // Fuzzy logic operations
        return bernoulli_bool{
            value && other.value,
            confidence * other.confidence
        };
    }
    
    explicit operator bool() const {
        // Threshold conversion for legacy code
        return confidence > 0.5 ? value : sample_bernoulli(confidence);
    }
};
\end{verbatim}

\subsection{Container Adaptations}

Standard containers need adaptation for Bernoulli types:

\begin{verbatim}
template<typename T>
class bernoulli_set {
    using comparator = std::function<bernoulli<bool>(const T&, const T&)>;
    
    struct node {
        T value;
        double insertion_confidence;
        std::vector<std::pair<T, double>> possible_duplicates;
    };
    
    void insert(const T& value) {
        // Probabilistic insertion with duplicate tracking
        for (auto& n : nodes) {
            auto eq = compare(value, n.value);
            if (eq.probability_true() > duplicate_threshold) {
                n.possible_duplicates.push_back({value, eq.probability_true()});
                return;
            }
        }
        nodes.push_back({value, 1.0, {}});
    }
};
\end{verbatim}

\section{Related Work}

\subsection{Probabilistic Programming}

Languages like Church \cite{goodman2008} and Stan \cite{carpenter2017} support probabilistic computation but don't address the regular type concept directly.

\subsection{Fuzzy Logic and Computing}

Fuzzy set theory \cite{zadeh1965} provides mathematical foundations for approximate membership, which Bernoulli types extend to general computation.

\subsection{Eventual Consistency}

CRDTs \cite{shapiro2011} achieve eventual consistency through commutative operations, while Bernoulli types model the uncertainty during convergence.

\subsection{Approximate Computing}

Work on approximate hardware \cite{han2013} and quality-of-service programming \cite{hoffmann2011} shares goals but focuses on different abstraction levels.

\section{Future Directions}

\subsection{Language Integration}

Future programming languages could provide first-class support:

\begin{verbatim}
// Hypothetical syntax
probable<int> x = 42 +/- 0.1;  // Value with uncertainty
if probably (x == 42) {       // Probabilistic branching
    // Execute with high probability
} else maybe {               // Lower probability branch
    // Handle uncertainty
}
\end{verbatim}

\subsection{Verification Tools}

Static analysis for probabilistic properties:
\begin{itemize}
    \item Probability of invariant violations
    \item Error propagation analysis
    \item Convergence guarantees
\end{itemize}

\subsection{Hardware Support}

Probabilistic computing primitives in hardware:
\begin{itemize}
    \item Native Bernoulli boolean operations
    \item Approximate equality instructions
    \item Stochastic arithmetic units
\end{itemize}

\section{Philosophical Implications}

\subsection{The Nature of Computational Truth}

Bernoulli types reveal that computation doesn't access truth directly—it observes it. The latent/observed distinction is not a limitation but a fundamental aspect of physical computation. Even "deterministic" systems are simply those where observation error is negligible.

\subsection{Identity Through Observation}

The apparent violation of reflexivity ($x == x$ may be false) doesn't mean identity is broken—it means we can only observe identity, not access it directly. Consider:
\begin{itemize}
    \item \textbf{Latent}: A value is always identical to itself
    \item \textbf{Observed}: Each observation of self-identity may differ
    \item \textbf{Reality}: Different parts of a distributed system may observe different identities
\end{itemize}

\subsection{From Determinism to Observation}

Traditional programming assumes perfect observation—that our code directly manipulates latent mathematical objects. Bernoulli types acknowledge that all computation is mediated by observation:
\begin{itemize}
    \item Physical hardware observes mathematical operations
    \item Networks observe messages through noisy channels  
    \item Sensors observe the physical world imperfectly
    \item Even "pure" functions observe their inputs through the lens of representation
\end{itemize}

This isn't the end of determinism but its proper contextualization: determinism is the special case where observation is perfect.

\section{Appendix: Limitations and Interop Guidelines}

\begin{proposition}[Non-regularity under noisy equality]
If a type's equality operator returns an observed Boolean with nonzero error (i.e., $\Prob\{\obs{(a=b)}\neq (a=b)\}>0$ for some $a,b$), then the type is not regular in the classical sense: observed equality is not guaranteed to be reflexive, symmetric, or transitive on every evaluation.
\end{proposition}

\paragraph{Design guidance.}
\begin{itemize}
  \item \textbf{Surface latent vs observed}: Expose both a latent-spec notion of equality and an observed comparator returning a probability or error interval.
  \item \textbf{Stabilize by repetition}: For critical checks, aggregate multiple observations (majority vote) and bound error via concentration.
  \item \textbf{Contain scope}: Restrict observed equality to local algorithmic steps; publish only stabilized results across module boundaries.
  \item \textbf{Document assumptions}: Record independence/memoryless assumptions and expected error rates at API boundaries.
\end{itemize}

\section{Fundamental Limits: Rank and Asymptotic Indistinguishability}

Even with perfect knowledge of confusion matrix parameters, some latent values remain asymptotically indistinguishable due to the rank structure of the observation process.

\begin{theorem}[Asymptotic Indistinguishability in Regular Types]
Consider the confusion matrix $Q$ for observing equality between regular type values. If $\text{rank}(Q) < |T|$ where $T$ is the type's value space, then there exist latent configurations that remain indistinguishable even with unlimited observations.
\end{theorem}

This has profound implications for regular types:

\begin{itemize}
    \item \textbf{Perfect parameter knowledge isn't enough}: Even knowing all error rates $\{\epsilon_{i,j}\}$ exactly, some latent equalities cannot be distinguished
    \item \textbf{Fundamental privacy bounds}: The rank deficiency creates inherent privacy—some information remains protected regardless of computational resources
    \item \textbf{Asymptotic convergence limits}: No amount of repeated observation can resolve certain ambiguities if they lie in the null space of the confusion matrix
\end{itemize}

\begin{remark}[Connection to Oblivious Computing]
This rank-based analysis reveals why certain oblivious computation protocols provide unconditional security: the observation process is designed to be rank-deficient, ensuring that private inputs remain indistinguishable even with unlimited computational power applied to the observed outputs.
\end{remark}

For regular types, this means that some equality relationships are fundamentally unobservable, creating a hierarchy of knowable vs unknowable relationships that persists regardless of how perfect our measurement apparatus becomes.

\section{Conclusions}

Regular types assume perfect observation of equality—that when we write \code{a == b}, we access the latent mathematical truth. Bernoulli types acknowledge that equality comparison is itself an observation through a potentially noisy channel. This shift from assumed perfect observation to explicit approximate observation has profound implications:

\begin{itemize}
    \item \textbf{Honesty}: Acknowledges that all computation observes latent truth imperfectly
    \item \textbf{Robustness}: Systems expecting observation errors handle real-world failures gracefully
    \item \textbf{Scalability}: Accepting approximate observations enables better scaling
    \item \textbf{Privacy}: The gap between latent and observed can protect sensitive information
    \item \textbf{Distributed systems}: Different nodes may observe different equalities—a feature, not a bug
\end{itemize}

The apparent violation of regularity axioms reveals a deeper truth: those axioms assume we can perfectly observe mathematical relationships. In reality, every equality check, every comparison, every computation is an observation of latent mathematical truth through the lens of physical implementation.

By making the latent/observed distinction explicit, Bernoulli types don't break programming—they make it more honest about the nature of computation. As systems grow in scale and complexity, embracing the fundamental gap between what is mathematically true and what we can observe becomes not just useful but necessary.

\bibliography{references}

% Shared one-page cheat sheet at end for quick reference
% Shared appendix: Cheat Sheet for Bernoulli Series
\section*{Appendix: Bernoulli Cheat Sheet}

% Symbols
\paragraph{Symbols}
\begin{tabular}{@{}ll@{}}
$\fprate$ & False positive rate $\Prob\{\obs{x}\in\obs{S}\mid x\notin S\}$ \\
$\fnrate$ & False negative rate $\Prob\{\obs{x}\notin\obs{S}\mid x\in S\}$ \\
$\tprate$ & True positive rate $1-\fnrate$; $\tnrate$: true negative rate $1-\fprate$ \\
$\epsilon$ & Pointwise error for maps: $\epsilon(x)=\Prob\{\obs{f}(x)\neq f(x)\}$ \\
$Q$ & Confusion matrix: $Q_{ij}=\Prob\{\text{obs } t_j\mid \text{lat } t_i\}$ \\
$W(y\mid x)$ & Channel kernel for pointwise observation \\
\end{tabular}

% Set formulas
\paragraph{Bernoulli sets (independent sub-queries)}
\begin{align*}
\fprate_{A\cap B} &= \fprate_A\,\fprate_B, & \fnrate_{A\cap B} &= 1-(1-\fnrate_A)(1-\fnrate_B), \\
\fprate_{A\cup B} &= 1-(1-\fprate_A)(1-\fprate_B), & \fnrate_{A\cup B} &= \fnrate_A\,\fnrate_B, \\
\fprate_{\overline{A}} &= \fnrate_A, & \fnrate_{\overline{A}} &= \fprate_A.
\end{align*}

% Maps formulas
\paragraph{Bernoulli maps}
Composition bound: $\Prob\{\obs{(g\circ f)}\neq g\circ f\} \le 1-(1-\epsilon_f)(1-\epsilon_g)$.  
Channel composition: $W_{g\circ f}(z\mid y')=\sum_y W_g(z\mid y)W_f(y\mid y')$.  
BSC composition: $\epsilon_{g\circ f}=\epsilon_f+\epsilon_g-\epsilon_f\epsilon_g$.

% Bayes / MV
\paragraph{Bayes and aggregation}
Posterior membership: $\displaystyle \Prob\{x\in S\mid x\in \obs{S}\} = \frac{\pi(1-\fnrate)}{\pi(1-\fnrate)+(1-\pi)\fprate}$.  
Majority vote (i.i.d., $\epsilon<\tfrac12$): $\Prob\{\hat{x}_k\neq x\} \le e^{-2(\tfrac12-\epsilon)^2 k}$.

% Intervals
\paragraph{Intervals}
If $\fprate\in[\underline{\alpha},\overline{\alpha}]$ and $\fnrate\in[\underline{\beta},\overline{\beta}]$, propagate by endpoint evaluation due to monotonicity.

% Assumptions
\paragraph{Assumptions}
Unless stated otherwise: (i) memoryless channels (independent across items), (ii) finite alphabets for confusion-matrix arguments, (iii) independence between sub-queries when using product-form laws, (iv) priors specified when applying Bayes.



\end{document}
